% !TeX root = ../document.tex

\chapter{广义相对论基础}
\begin{xiti}
	\item 试证弯曲时空麦氏方程 $\tensor{\nabla}{^a} \tensor{F}{_a_b} = - 4\pi \tensor{J}{_b}$ 蕴含电荷守恒定律,即 $\Nabla{a} \tensor{J}{^a} = 0$ 。注:$\tensor{\nabla}{^a} \tensor{F}{_a_b} = - 4\pi \tensor{J}{_b}$ 等价于式 (7-2-8) 而非 (7-2-9) ,故本题表明式 (7-2-8) 而非式 (7-2-9) 可推出电荷守恒。
	
	\begin{zm}
		\begin{equation*}
		-4 \pi \Nabla{a} \tensor{J}{^a} = \Nabla{a} \Nabla{b} \tensor{F}{^b^a} = 0.
		\end{equation*}
	\end{zm}
    
    \item 试证 $\displaystyle \Fd{\tensor{\omega}{_a}} = \Dd{\tensor{\omega}{_a}} + \left(\tensor{A}{_a} \wedge \tensor{Z}{_b}\right) \tensor{\omega}{^b} \quad \forall \tensor{\omega}{_a} \in \F_{G}(0,1).$
    
    \begin{zm}
    	$\forall \tensor{v}{^a} \in \F_{G}(1,0),$
    	\begin{align*}
    	    \tensor{v}{^a} \Fd{\tensor{\omega}{_a}} &= \Fd{\left(\tensor{v}{^a}\tensor{\omega}{_a}\right)} - \tensor{\omega}{_a} \Fd{\tensor{v}{^a}}\\
    	    &= \tensor{v}{^a} \Dd{\tensor{\omega}{_a}} + \tensor{\omega}{_a} \Dd{\tensor{v}{^a}} - \tensor{\omega}{_a} \left( \Dd{\tensor{v}{^a}} + 2 \tensor{A}{^{[a}} \tensor{Z}{^{b]}} \tensor{v}{_b} \right)\\
    	    &= \tensor{v}{^a} \left( \Dd{\tensor{\omega}{_a}} - 2 \tensor{A}{_{[b}} \tensor{Z}{_{a]}} \tensor{\omega}{^b} \right)\\
    	    &= \tensor{v}{^a} \left( \Dd{\tensor{\omega}{_a}} + \tensor{A}{_{a}} \wedge \tensor{Z}{_{b}} \tensor{\omega}{^b} \right).
    	\end{align*}
    \end{zm}
    
    \item 试证费米导数性质3.
    
    \begin{zm}
    	性质3如下:
    	\begin{Property}
    		若 $\tensor{w}{^a}$ 是 $G(\tau)$ 上的空间矢量场(对线上各点 $\tensor{w}{^a} \tensor{Z}{_a} = 0$ ),则 \[ \Fdd{\tensor{w}{^a}} = \tensor{h}{^a_b} \left( \Ddd{\tensor{w}{^b}} \right), \] 其中 $\tensor{h}{^a_b} = \tensor{g}{_a_b} + \tensor{Z}{_a} \tensor{Z}{_b}$ ,$\tensor{h}{^a_b} = \tensor{g}{^a^c} \tensor{h}{_c_b}$ 是 $G(\tau)$ 上各点的投影映射。
    	\end{Property}
    	\begin{Proof}
    		$\tensor{h}{^a_b} = \tensor{g}{^a^c} \left( \tensor{g}{_c_b} + \tensor{Z}{_c} \tensor{Z}{_b} \right) = \tensor{\delta}{^a_b} + \tensor{Z}{^a} \tensor{Z}{_b},$
    		\begin{align*}
    		\tensor{h}{^a_b} \Dd{\tensor{w}{^b}} &= \left( \tensor{\delta}{^a_b} + \tensor{Z}{^a} \tensor{Z}{_b} \right) \Dd{\tensor{w}{^b}}\\
    		&= \Dd{\tensor{w}{^a}} + \tensor{Z}{^a} \left( \Dd{\left( \tensor{Z}{_b} \tensor{w}{^b} \right)} - \tensor{w}{^b} \Dd{\tensor{Z}{_b}} \right)\\
    		&= \Dd{\tensor{w}{^a}} - \tensor{Z}{^a} \tensor{A}{^b} \tensor{w}{_b}\\
    		&= \Dd{\tensor{w}{^a}} + \left(\tensor{A}{^a} \tensor{Z}{^b} - \tensor{Z}{^a} \tensor{A}{^b}\right) \tensor{w}{_b}\\
    		&= \Fd{\tensor{w}{^a}}.
    		\end{align*}
    	\end{Proof}
    \end{zm}
    
    \item 试证类时线 $G(\tau)$ 上长度不变(且非零)的矢量场必经受时空转动。提示:令 $\tensor{u}{^a} \equiv \Ddd{\tensor{v}{^a}}$,则 $\tensor{u}{_a} \tensor{v}{^a}=0$。先证:无论 $\tensor{v}{_a}\tensor{v}{^a}$ 为零与否,总有 $G(\tau)$ 上矢量场 $\tensor{{v'}}{^a}$ 使 $\tensor{{v'}}{_a} \tensor{v}{^a} = 1$ 。再验证 $\tensor{v}{^a}$ 经受以 $\tensor{\Omega}{_a_b} \equiv 2 \tensor{{v'}}{_{[a}} \tensor{u}{_{b]}}$ 为角速度 2 形式的时空转动。
    
	\begin{zm}
		\begin{enumerate}
			\item 记 $\displaystyle \tensor{u}{^a} = \Dd{\tensor{v}{^a}}$ ,则 $\displaystyle \Dd{\left(\tensor{v}{_a}\tensor{v}{^a}\right)} = 2\tensor{u}{_a} \tensor{v}{^a}=0$ 。
			\item 若 $\tensor{v}{^a} \tensor{v}{_a} \neq 0$,令
			\begin{equation*}
				\tensor{{v'}}{^a} = \frac{\tensor{v}{^a}}{\tensor{v}{^b} \tensor{v}{^b}},
			\end{equation*}
			若 $\tensor{v}{^a} \tensor{v}{_a} = 0$,则 $\tensor{Z}{^a} \tensor{v}{_a}$ 不为零,因为与类时矢量内积为零则为类空矢量。于是定义
			\begin{equation*}
				\tensor{{v'}}{^a} = \frac{\tensor{Z}{^a}}{\tensor{Z}{^b} \tensor{v}{_b}}.
			\end{equation*}
			\item 定义 $\tensor{\Omega}{_a_b} = 2 \tensor{{v'}}{_{[a}} \tensor{u}{_{b]}}$,则
			\begin{equation*}
				- \tensor{\Omega}{^a^b} \tensor{v}{_b} = \tensor{u}{^a} = \Dd{\tensor{v}{^a}},
			\end{equation*}
			故 $\tensor{v}{^a}$ 经受以 $\tensor{\Omega}{_a_b}$ 为角速度 2 形式的时空转动。
		\end{enumerate}
	\end{zm}
	
	\item 设 $\left\{ T,X,Y,Z \right\}$ 为闵氏时空的洛伦兹坐标系,曲线 $G(\tau)$ 的参数表达式为
	\begin{equation*}
		T = A^{-1} \sinh A\tau \qc X = A^{-1} \cosh A\tau \qc Y=Z=0 \qc \text{(其中$A$ 为常数)}
	\end{equation*}
	\begin{enumerate}[label=(\alph*)]
		\item 试证 $G(\tau)$ 是类时双曲线(即图(6-43)\footnote{即本文档图~\hyperlink{t6}{6.13}} 中的 $G$)
	\end{enumerate}
    
\end{xiti}