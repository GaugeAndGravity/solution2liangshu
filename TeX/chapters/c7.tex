% !TeX root = ../document.tex

\chapter{广义相对论基础}
\begin{xiti}
	\item 试证弯曲时空麦氏方程 $\tensor{\nabla}{^a} \tensor{F}{_a_b} = - 4\pi \tensor{J}{_b}$ 蕴含电荷守恒定律,即 $\Nabla{a} \tensor{J}{^a} = 0$ 。注:$\tensor{\nabla}{^a} \tensor{F}{_a_b} = - 4\pi \tensor{J}{_b}$ 等价于式 (7-2-8) 而非 (7-2-9) ,故本题表明式 (7-2-8) 而非式 (7-2-9) 可推出电荷守恒。
	
	\begin{zm}
		\begin{equation*}
		-4 \pi \Nabla{a} \tensor{J}{^a} = \Nabla{a} \Nabla{b} \tensor{F}{^b^a} = 0.
		\end{equation*}
	\end{zm}
    
    \item 试证 $\displaystyle \Fd{\tensor{\omega}{_a}} = \Dd{\tensor{\omega}{_a}} + \left(\tensor{A}{_a} \wedge \tensor{Z}{_b}\right) \tensor{\omega}{^b} \quad \forall \tensor{\omega}{_a} \in \F_{G}(0,1).$
    
    \begin{zm}
    	$\forall \tensor{v}{^a} \in \F_{G}(1,0),$
    	\begin{align*}
    	    \tensor{v}{^a} \Fd{\tensor{\omega}{_a}} &= \Fd{\left(\tensor{v}{^a}\tensor{\omega}{_a}\right)} - \tensor{\omega}{_a} \Fd{\tensor{v}{^a}}\\
    	    &= \tensor{v}{^a} \Dd{\tensor{\omega}{_a}} + \tensor{\omega}{_a} \Dd{\tensor{v}{^a}} - \tensor{\omega}{_a} \left( \Dd{\tensor{v}{^a}} + 2 \tensor{A}{^{[a}} \tensor{Z}{^{b]}} \tensor{v}{_b} \right)\\
    	    &= \tensor{v}{^a} \left( \Dd{\tensor{\omega}{_a}} - 2 \tensor{A}{_{[b}} \tensor{Z}{_{a]}} \tensor{\omega}{^b} \right)\\
    	    &= \tensor{v}{^a} \left( \Dd{\tensor{\omega}{_a}} + \tensor{A}{_{a}} \wedge \tensor{Z}{_{b}} \tensor{\omega}{^b} \right).
    	\end{align*}
    \end{zm}
    
    \item 试证费米导数性质3.\label{prob-7.3}
    
    \begin{zm}
    	性质3如下:
    	\begin{Property}
    		若 $\tensor{w}{^a}$ 是 $G(\tau)$ 上的空间矢量场(对线上各点 $\tensor{w}{^a} \tensor{Z}{_a} = 0$ ),则 \[ \Fdd{\tensor{w}{^a}} = \tensor{h}{^a_b} \left( \Ddd{\tensor{w}{^b}} \right), \] 其中 $\tensor{h}{^a_b} = \tensor{g}{_a_b} + \tensor{Z}{_a} \tensor{Z}{_b}$ ,$\tensor{h}{^a_b} = \tensor{g}{^a^c} \tensor{h}{_c_b}$ 是 $G(\tau)$ 上各点的投影映射。
    	\end{Property}
    	\begin{Proof}
    		$\tensor{h}{^a_b} = \tensor{g}{^a^c} \left( \tensor{g}{_c_b} + \tensor{Z}{_c} \tensor{Z}{_b} \right) = \tensor{\delta}{^a_b} + \tensor{Z}{^a} \tensor{Z}{_b},$
    		\begin{align*}
    		\tensor{h}{^a_b} \Dd{\tensor{w}{^b}} &= \left( \tensor{\delta}{^a_b} + \tensor{Z}{^a} \tensor{Z}{_b} \right) \Dd{\tensor{w}{^b}}\\
    		&= \Dd{\tensor{w}{^a}} + \tensor{Z}{^a} \left( \Dd{\left( \tensor{Z}{_b} \tensor{w}{^b} \right)} - \tensor{w}{^b} \Dd{\tensor{Z}{_b}} \right)\\
    		&= \Dd{\tensor{w}{^a}} - \tensor{Z}{^a} \tensor{A}{^b} \tensor{w}{_b}\\
    		&= \Dd{\tensor{w}{^a}} + \left(\tensor{A}{^a} \tensor{Z}{^b} - \tensor{Z}{^a} \tensor{A}{^b}\right) \tensor{w}{_b}\\
    		&= \Fd{\tensor{w}{^a}}.
    		\end{align*}
    	\end{Proof}
    \end{zm}
    
    \item 试证类时线 $G(\tau)$ 上长度不变(且非零)的矢量场必经受时空转动。提示:令 $\tensor{u}{^a} \equiv \Ddd{\tensor{v}{^a}}$,则 $\tensor{u}{_a} \tensor{v}{^a}=0$。先证:无论 $\tensor{v}{_a}\tensor{v}{^a}$ 为零与否,总有 $G(\tau)$ 上矢量场 $\tensor{{v'}}{^a}$ 使 $\tensor{{v'}}{_a} \tensor{v}{^a} = 1$ 。再验证 $\tensor{v}{^a}$ 经受以 $\tensor{\Omega}{_a_b} \equiv 2 \tensor{{v'}}{_{[a}} \tensor{u}{_{b]}}$ 为角速度 2 形式的时空转动。
    
	\begin{zm}
		\begin{enumerate}
			\item 记 $\displaystyle \tensor{u}{^a} = \Dd{\tensor{v}{^a}}$ ,则 $\displaystyle \Dd{\left(\tensor{v}{_a}\tensor{v}{^a}\right)} = 2\tensor{u}{_a} \tensor{v}{^a}=0$ 。
			\item 若 $\tensor{v}{^a} \tensor{v}{_a} \neq 0$,令
			\begin{equation*}
				\tensor{{v'}}{^a} = \frac{\tensor{v}{^a}}{\tensor{v}{^b} \tensor{v}{^b}},
			\end{equation*}
			若 $\tensor{v}{^a} \tensor{v}{_a} = 0$,则 $\tensor{Z}{^a} \tensor{v}{_a}$ 不为零,因为与类时矢量内积为零则为类空矢量。于是定义
			\begin{equation*}
				\tensor{{v'}}{^a} = \frac{\tensor{Z}{^a}}{\tensor{Z}{^b} \tensor{v}{_b}}.
			\end{equation*}
			\item 定义 $\tensor{\Omega}{_a_b} = 2 \tensor{{v'}}{_{[a}} \tensor{u}{_{b]}}$,则
			\begin{equation*}
				- \tensor{\Omega}{^a^b} \tensor{v}{_b} = \tensor{u}{^a} = \Dd{\tensor{v}{^a}},
			\end{equation*}
			故 $\tensor{v}{^a}$ 经受以 $\tensor{\Omega}{_a_b}$ 为角速度 2 形式的时空转动。
		\end{enumerate}
	\end{zm}
	
	\item 设 $\left\{ T,X,Y,Z \right\}$ 为闵氏时空的洛伦兹坐标系,曲线 $G(\tau)$ 的参数表达式为
	\begin{equation*}
		T = A^{-1} \sinh A\tau \qc X = A^{-1} \cosh A\tau \qc Y=Z=0 \qc \text{(其中$A$ 为常数)}
	\end{equation*}
	\begin{enumerate}[label=(\alph*)]
		\item 试证 $G(\tau)$ 是类时双曲线(即图(6-43)\footnote{即本文档图~\hyperlink{t6}{6.13}} 中的 $G$),$\tau$ 是固有时,$A$ 是 $G(\tau)$ 的 4 加速 $\tensor{A}{^a}$ 的长度。
		\item 试证从 $\left\{ T,X,Y,Z \right\}$ 坐标系原点 $o$ 出发的与 $G(\tau)$ 有交的任一半直线 $\mu(s)$ 都与 $G(\tau)$ 正交。
		\item 设(b)中的 $\mu(s)$ 的参数 $s$ 是 $\mu$ 的线长,随着 $\mu(s)$ 取遍所有从 $o$ 出发并与 $G(\tau)$ 有交的半直线,便得 $G(\tau)$ 上的一个空间矢量场 $\tensor{w}{^a} \equiv \tensor{\left( \pdv*{s} \right)}{^a}$,试证 $\tensor{w}{^a}$ 沿 $G(\tau)$ 费移。
		\item 令 $\tensor{Z}{^a} \equiv \tensor{\left( \pdv*{\tau} \right)}{^a}$,选 $\left\{ \tensor{Z}{^a}, \tensor{w}{^a}, \tensor{\left( \pdv*{Y} \right)}{^a}, \tensor{\left( \pdv*{Z} \right)}{^a} \right\}$ 为 $G(\tau)$ 上的正交归一 4 标架场,求出 $G(\tau)$ 的固有坐标系 $\left\{ t, x, y, z \right\}$ 并指出其坐标域。
		
		答: $T=\left( A^{-1} + x \right) \sinh At \qc X= \left( A^{-1} + x \right) \cosh At \qc Y=y \qc Z=z$。

		\item 写出闵氏时空在上述固有坐标系中的线元表达式。计算闵氏度规在该系的克氏符,验证它满足引理 7-4-3,即式 (7-4-10)\footnote{正文 (7-4-10) 为
		\begin{equation*}
			\begin{gathered}
				\ChristoffelSymbol{0}{0}{0} = \ChristoffelSymbol{\sigma}{i}{j} = 0 \qc \ChristoffelSymbol{0}{0}{i} = \ChristoffelSymbol{0}{i}{0} = \ChristoffelSymbol{i}{0}{0} = \tensor{\hat{A}}{_i},\\
				\ChristoffelSymbol{i}{0}{j} = \ChristoffelSymbol{i}{j}{0} = - \tensor{\omega}{^k} \tensor{\varepsilon}{_0_k_i_j} \qc \sigma = 0,1,2,3; \quad i,j,k = 1,2,3.
			\end{gathered}
		\end{equation*}
		}。
	\end{enumerate}

		\begin{zm}
			\begin{enumerate}[label=(\alph*)]
				\item 由 $\cosh^2 x - \sinh^2 x = 1$ 知 $\left( A X \right)^2 - \left( A T \right)^2 = 1$,故这是渐近线为 $T=\pm X$ 的双曲线。
				
				以 $\tau$ 为参数,
				\begin{equation*}
					\tensor{\left( \pdv{\tau} \right)}{^a} = \cosh(A\tau) \tensor{\left( \pdv{T} \right)}{^a} + \sinh(A\tau) \tensor{\left( \pdv{X} \right)}{^a},
				\end{equation*}
				则
				\begin{equation*}
					\tensor{\eta}{_a_b} \tensor{\left( \pdv{\tau} \right)}{^a} \tensor{\left( \pdv{\tau} \right)}{^b} = - \cosh[2](A\tau) + \sinh[2](A\tau) = -1,
				\end{equation*}
				即切矢归一,$\tau$ 为固有时。

				将 $\tensor{\left( \pdv{\tau} \right)}{^a}$ 延拓为
				\begin{equation*}
					\tensor{Z}{^a} = A X \tensor{\left( \pdv{T} \right)}{^a} + A T \tensor{\left( \pdv{X} \right)}{^a},
				\end{equation*}
				容易算得观者四加速为
				\begin{equation*}
					\begin{split}
						\tensor{\hat{A}}{^a} &= \left. \tensor{Z}{^b} \Nabla{b} \tensor{Z}{^a}\right|_{G(\tau)}\\
						&= \left. A^2 T \tensor{\left( \pdv{T} \right)}{^a} + A^2 X \tensor{\left( \pdv{X} \right)}{^a} \right|_{G(\tau)}\\
						&= A \sinh(A\tau) \tensor{\left( \pdv{T} \right)}{^a} + A \cosh(A\tau) \tensor{\left( \pdv{X} \right)}{^a},
					\end{split}
				\end{equation*}
				则
				\begin{equation*}
					\tensor{\eta}{_a_b} \tensor{\hat{A}}{^a} \tensor{\hat{A}}{^b} = A^2 \left( \cosh[2](A\tau) - \sinh[2](A\tau) \right) = A^2,
				\end{equation*}
				即四加速的模长为 $A$。

				\item 与 $G$ 交于 $\tau$ 处的 $\mu$ 的方程为
				\begin{equation*}
					T = \tanh(A\tau) X,
				\end{equation*}
				故其在 $G(\tau)$ 处的切矢正比于
				\begin{equation*}
					\tensor{\left( \pdv{s} \right)}{^a} = \sinh(A\tau)\tensor{\left( \pdv{T} \right)}{^a} + \cosh(A\tau) \tensor{\left( \pdv{X} \right)}{^a},
				\end{equation*}
				可算得
				\begin{equation*}
					\tensor{\left( \pdv{s} \right)}{^a} \tensor{\left( \pdv{\tau} \right)}{_a} = - \cosh(A\tau) \sinh(A\tau) + \cosh(A\tau) \sinh(A\tau) =0.
				\end{equation*}
				\item 在 (b) 中给出的 $\tensor{\left( \pdv*{s} \right)}{^a}$ 已经是归一的,因而就是 $\tensor{w}{^a}$。
				由(b) 和习题~\hyperref[prob-7.3]{3},知
				\begin{equation*}
					\begin{split}
						\Fd{\tensor{w}{^a}} &= \tensor{h}{^a_b} \Dd{\tensor{w}{^b}}\\
						&= \tensor{h}{^a_b} \tensor{Z}{^c} \Nabla{c} \tensor{w}{^b}\\
						&= \tensor{h}{^a_b} \left( A \cosh(A\tau) \tensor{\left( \pdv{T} \right)}{^b} + A \sinh(A\tau) \tensor{\left( \pdv{X} \right)}{^b} \right)\\
						&= A \tensor{h}{^a_b} \tensor{Z}{^b}\\
						&= 0,
					\end{split}
				\end{equation*}
				其中 $\tensor{Z}{^a} = \tensor{\left( \pdv*{\tau} \right)}{^a}$,故 $\tensor{w}{^a}$ 沿 $G(\tau)$ 费移。
				\item 以 $G(0)$ 为坐标原点,$\left\{ t,0,0,0\right\}$ 对应的点为 $G(t)$,即
				\begin{equation*}
					T= A^{-1} \sinh A t \qc X= A^{-1} \cosh A t \qc Y = Z =0,
				\end{equation*}
				而此点处
				\begin{equation*}
					\begin{split}
						&x\tensor{w}{^a} + y \tensor{\left( \pdv{Y} \right)}{^a} + z \tensor{\left( \pdv{Z} \right)}{^a}\\
						={}& x \sinh(At) \tensor{\left( \pdv{T} \right)}{^a} + x \cosh(At) \tensor{\left( \pdv{X} \right)}{^a} + y \tensor{\left( \pdv{Y} \right)}{^a} + z \tensor{\left( \pdv{Z} \right)}{^a},
					\end{split}
				\end{equation*}
				沿此矢量决定的测地线(直线)走参数为1的距离,即
				\begin{equation*}
					\Delta T = x \sinh(At) \qc \Delta X = x \cosh(At) \qc \Delta Y = y \qc \Delta Z = z \tensor{\left( \pdv{Z} \right)}{^a},
				\end{equation*}
				故 $\left\{ t,x,y,z \right\}$ 对应的点为
				\begin{equation}
					T=\left( A^{-1} + x \right) \sinh At \qc X= \left( A^{-1} + x \right) \cosh At \qc Y=y \qc Z=z. \label{eq-txyz2TXYZ}
				\end{equation}
				\item 计算得
				\begin{equation*}
					\begin{split}
						\dd{s}^2 ={}& - \dd{T}^2 + \dd{X}^2 + \dd{Y}^2 + \dd{Z}^2\\
						={}& - \left[ \left( 1 + A x \right) \cosh(A t) \dd{t} + \sinh(A t) \dd{x} \right]^2\\
						&{}+ \left[ \left( 1 + A x \right) \sinh(A t) \dd{t} + \cosh(A t) \dd{x} \right]^2 + \dd{y}^2 + \dd{z}^2\\
						={}& - \left( 1+ A x \right)^2 \dd{t}^2 + \dd{x}^2 + \dd{y}^2 + \dd{z}^2,
					\end{split}
				\end{equation*}
				容易算得非零克氏符为
				\begin{equation*}
					\ChristoffelSymbol{t}{t}{x} = \ChristoffelSymbol{t}{x}{t} = \frac{A}{1+A x} \qc \ChristoffelSymbol{x}{t}{t} = A \left( 1+Ax \right),
				\end{equation*}
				在线上时
				\begin{equation*}
					\ChristoffelSymbol{t}{t}{x} = \ChristoffelSymbol{t}{x}{t} =  \ChristoffelSymbol{x}{t}{t} = A,
				\end{equation*}
				% 而观者的四加速为
				% \begin{equation*}
				% 	\begin{split}
				% 		\tensor{\hat{A}}{^a} &= \tensor{Z}{^b} \Nabla{b} \tensor{Z}{^a}\\
				% 		&= \left( A X \tensor{\left( \pdv{T} \right)}{^b} + A T \tensor{\left( \pdv{X} \right)}{^b} \right) \Nabla{b} \left( A X \tensor{\left( \pdv{T} \right)}{^a} + A T \tensor{\left( \pdv{X} \right)}{^a} \right)\\
				% 		&= A^2 X \tensor{\left( \pdv{X} \right)}{^a} + A^2 T \tensor{\left( \pdv{T} \right)}{^a},
				% 	\end{split}
				% \end{equation*}
				对~\eqref{eq-txyz2TXYZ} 反解得坐标变换
				\begin{equation*}
					t = A^{-1} \tanh[-1](\frac{T}{X}) \qc x = \sqrt{X^2 - T^2} - A^{-1} \qc y = Y \qc z = Z,
				\end{equation*}
				故
				\begin{equation*}
					\begin{split}
						\tensor{\left( \pdv{T} \right)}{^a} &= \pdv{t}{T} \tensor{\left( \pdv{t} \right)}{^a} + \pdv{x}{T} \tensor{\left( \pdv{x} \right)}{^a}\\
						&= \frac{X}{A\left( X^2 - T^2 \right)} \tensor{\left( \pdv{t} \right)}{^a} - \frac{T}{\sqrt{X^2 - T^2}} \tensor{\left( \pdv{x} \right)}{^a}\\
						&= \frac{\cosh At}{1+Ax} \tensor{\left( \pdv{t} \right)}{^a} - \sinh(At) \tensor{\left( \pdv{x} \right)}{^a},\\
						\tensor{\left( \pdv{X} \right)}{^a} &= \pdv{t}{X} \tensor{\left( \pdv{t} \right)}{^a} + \pdv{x}{X} \tensor{\left( \pdv{x} \right)}{^a}\\
						&= \frac{T}{A\left( T^2 - X^2 \right)} \tensor{\left( \pdv{t} \right)}{^a} + \frac{X}{\sqrt{X^2 - T^2}} \tensor{\left( \pdv{x} \right)}{^a}\\
						&= - \frac{\sinh At}{1+Ax} \tensor{\left( \pdv{t} \right)}{^a} + \cosh(At) \tensor{\left( \pdv{x} \right)}{^a},
					\end{split}
				\end{equation*}
				故
				\begin{equation*}
					\begin{split}
						\tensor{\hat{A}}{^a} ={}& A^2 X \tensor{\left( \pdv{X} \right)}{^a} + A^2 T \tensor{\left( \pdv{T} \right)}{^a}\\
						={}& A \left( 1 + A x \right) \cosh(At) \left( - \frac{\sinh At}{1+Ax} \tensor{\left( \pdv{t} \right)}{^a} + \cosh(At) \tensor{\left( \pdv{x} \right)}{^a} \right)\\
						&{} + A\left( 1+Ax \right) \sinh(At) \left( \frac{\cosh At}{1+Ax} \tensor{\left( \pdv{t} \right)}{^a} - \sinh(At) \tensor{\left( \pdv{x} \right)}{^a} \right)\\
						={}& A \left( 1+ Ax \right) \tensor{\left( \pdv{x} \right)}{^a},
					\end{split}
				\end{equation*}
				在线上有
				\begin{equation*}
					\tensor{\hat{A}}{^a} = A \tensor{\left( \pdv{x} \right)}{^a},
				\end{equation*}
				满足引理,证毕。
			\end{enumerate}
		\end{zm}
    
\end{xiti}