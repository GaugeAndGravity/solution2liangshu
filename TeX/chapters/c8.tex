% !TeX root = ../document.tex

\chapter{爱因斯坦方程的求解}

\begin{xiti}
    \item 试证命题 8-1-1。
    
    \begin{zm}
        正文命题8-1-1为
        \begin{Proposition}
            设 $\tensor{\xi}{^a} = \tensor{\left( \pdv*{t} \right)}{^a} $ 是 Killing 矢量场,$\Sigma_0 = \left\{ p \in M \mid t(p) = 0 \right\}$ 是处处与 $\tensor{\xi}{^a}$ 正交的超曲面,则 超曲面 $\Sigma_{t_1} = \left\{ p \in M \mid t(p) = t_1 \right\}$ 也处处与 $\tensor{\xi}{^a}$ 正交。
        \end{Proposition}

        \begin{Proof}
            设矢量场 $\tensor{\xi}{^a}$ 生成的单参微分同胚群 为 $\phi_t$,则 $\Sigma_{t_1} = \phi_{t_1} \left[ \Sigma_0 \right]$,任取 $p \in \Sigma_0$,$q = \phi_{t_1}(p) \in \Sigma_{t_1}$,以及 $q$ 点处 $\Sigma_{t_1}$ 的任意切矢量 $\tensor{v}{^a} \in \TB[q]{\Sigma_{t_1}}$,则 $\tensor{u}{^a} = \left( \phi_{-t_1} \right)_* \tensor{v}{^a} \in \TBx[p]{\Sigma_0}$,故
            \begin{equation*}
                \begin{split}
                    \left.\tensor{g}{_a_b}\right|_{q} \tensor{v}{^a} \left.\tensor{\xi}{^b}\right|_q &= \left( \phi_{t_1} \right)_* \left( \left. \tensor{g}{_a_b} \right|_{p} \right) \tensor{v}{^a} \left( \phi_{t_1} \right)_* \left( \left. \tensor{\xi}{^a} \right|_p \right)\\
                    &= \left( \phi_{t_1} \right)_* \left( \left. \tensor{g}{_a_b} \right|_p \tensor{u}{^a} \left. \tensor{\xi}{^b} \right|_p \right)\\
                    &= \left( \phi_{t_1} \right)_* 0\\
                    &= 0,
                \end{split}
            \end{equation*}
            故 $\Sigma_{t_1}$ 也与 $\tensor{\xi}{^a}$ 正交。
        \end{Proof}  
    \end{zm}
    
    \item 设 $\gamma(r)$ 是图~\hyperlink{8-6}{8-6}~中 $\Sigma_{t}$ 上从 $p_1$ 到 $p_2$ 的、$\theta$ 和 $\varphi$ 都为常数的曲线(以径向坐标 $r$ 为曲线参数),试证 $\gamma(r)$ 是(非仿射参数化的)测地线。提示:用式(5-7-2)。
    
    \begin{figure}[htpb!]
        \centering
        \begin{tikzpicture}[thick]
            \draw (0,0) ellipse [x radius=1,y radius=.2];
            \draw (0,0) ellipse [x radius=2.1, y radius=0.4];
            \draw (0,0) ellipse [x radius=3, y radius=0.74];
            \draw (0,0) ellipse [x radius=4, y radius=1.2];
            \filldraw[fill=white,color=white] (-4.5,1.5) -- (4.5,1.5) -- (4.5,-1.5) -- (-4.5,-1.5) -- cycle (-2.5,1.11) -- (-4.1,-1.11) -- (4.1,-1.11) -- (2.5,1.09) -- cycle;
            \draw (-2.5,1.11) -- (-4.1,-1.11) -- (4.1,-1.11) -- (2.5,1.09) -- cycle;
            \draw[dashed] (0,0) -- (3.3,0);
            \draw[-{Stealth[length=0.4cm,width'=0pt 0.3,inset=1pt]}] (1,0) -- (1.99,0);
            \node (n) at (1.5,-0.32) {\colorbox{white}{$\smash[b]{\tensor{n}{^a}}$}};
            \node (Sigma) at (3.3,-1.3) {$\Sigma$};
            \fill[fill=white] (2.15,0.4) -- (2.15,1.2) -- (2.05,1.2) -- (2.05,0.4) -- cycle;
            \fill[fill=white] (-2.15,0.4) -- (-2.15,1.2) -- (-2.05,1.2) -- (-2.05,0.4) -- cycle;
            \draw[-{Stealth[length=0.5cm,width'=0pt 0.3,inset=1pt]}] (2.1,0) -- (2.1,1.7);
            \draw[-{Stealth[length=0.5cm,width'=0pt 0.3,inset=1pt]}] (-2.1,0) -- (-2.1,1.7);
            \node[left] (xi1) at (2.1,1.5) {$\tensor{\xi}{^a}$};
            \node[right] (xi2) at (-2.05,1.5) {$\tensor{\xi}{^a}$};
        \end{tikzpicture}
        \caption{正文图 8-6}\hypertarget{8-6}{}
    \end{figure}

    \begin{zm}
        $\gamma(r)$ 即为 $r$ 坐标线,故切矢为 $\tensor{\dot{\gamma}}{^a} = \tensor{\left( \pdv{r} \right)}{^a}$。正文 (8-3-4) 列出了球对称度规 (8-3-2) 式
        \begin{equation*}
            {\dd{s}}^2 = - \e{2A(r)} \dd{t}^2 + \e{2B(r)} \dd{r}^2 + r^2 \left( \dd{\theta}^2 + \sin^2 \theta \dd{\varphi}^2 \right) \tag{8-3-2}
        \end{equation*}
        的所有非零克氏符为
        \begin{equation*}
            \begin{aligned}
                \ChristoffelSymbol{0}{0}{1} &= \ChristoffelSymbol{0}{1}{0} = A', & \ChristoffelSymbol{1}{0}{0} &= A' \e{2(A-B)}, & \ChristoffelSymbol{1}{1}{1} &= B',\\
                \ChristoffelSymbol{1}{2}{2} &= - r \e{-2B}, & \ChristoffelSymbol{1}{3}{3} &= - r \sin^2 \theta \e{-2B}, & \ChristoffelSymbol{2}{1}{2} &= \ChristoffelSymbol{2}{2}{1} = 1/r,\\
                \ChristoffelSymbol{2}{3}{3} &= - \sin \theta \cos \theta, & \ChristoffelSymbol{3}{1}{3} &= \ChristoffelSymbol{3}{3}{1} = 1/r, & \ChristoffelSymbol{3}{2}{3} &= \ChristoffelSymbol{3}{3}{2} = \cot \theta,
            \end{aligned}
            \tag{8-3-4}
        \end{equation*}
        故
        \begin{equation*}
            \begin{split}
                \tensor{\dot{\gamma}}{^a} \Nabla{a} \tensor{\dot{\gamma}}{^b} &= \tensor{\left( \pdv{r} \right)}{^a} \Nabla{a} \tensor{\left( \pdv{r} \right)}{^b}\\
                &= \ChristoffelSymbol{\mu}{1}{1} \tensor{\left( \pdv{x^\mu} \right)}{^b}\\
                &= \ChristoffelSymbol{1}{1}{1} \tensor{\left( \pdv{r} \right)}{^b}\\
                &= \dv{B}{r} \tensor{\left( \pdv{r} \right)}{^b},
            \end{split}
        \end{equation*}
        故 $\gamma(r)$ 是测地线。
    \end{zm}
    
\end{xiti}
