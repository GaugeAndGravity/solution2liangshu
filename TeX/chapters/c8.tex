% !TeX root = ../document.tex

\chapter{爱因斯坦方程的求解}

\begin{xiti}
	\item 试证命题 8-1-1。

	\begin{zm}
		正文命题8-1-1为
		\begin{Proposition}
			设 $\tensor{\xi}{^a} = \tensor{\left( \pdv*{t} \right)}{^a} $ 是 Killing 矢量场,$\Sigma_0 = \left\{ p \in M \mid t(p) = 0 \right\}$ 是处处与 $\tensor{\xi}{^a}$ 正交的超曲面,则 超曲面 $\Sigma_{t_1} = \left\{ p \in M \mid t(p) = t_1 \right\}$ 也处处与 $\tensor{\xi}{^a}$ 正交。
		\end{Proposition}

		\begin{Proof}
			设矢量场 $\tensor{\xi}{^a}$ 生成的单参微分同胚群 为 $\phi_t$,则 $\Sigma_{t_1} = \phi_{t_1} \left[ \Sigma_0 \right]$,任取 $p \in \Sigma_0$,$q = \phi_{t_1}(p) \in \Sigma_{t_1}$,以及 $q$ 点处 $\Sigma_{t_1}$ 的任意切矢量 $\tensor{v}{^a} \in \TB[q]{\Sigma_{t_1}}$,则 $\tensor{u}{^a} = \left( \phi_{-t_1} \right)_* \tensor{v}{^a} \in \TBx[p]{\Sigma_0}$,故
			\begin{equation*}
				\begin{split}
					\left.\tensor{g}{_a_b}\right|_{q} \tensor{v}{^a} \left.\tensor{\xi}{^b}\right|_q &= \left( \phi_{t_1} \right)_* \left( \left. \tensor{g}{_a_b} \right|_{p} \right) \tensor{v}{^a} \left( \phi_{t_1} \right)_* \left( \left. \tensor{\xi}{^a} \right|_p \right)\\
					&= \left( \phi_{t_1} \right)_* \left( \left. \tensor{g}{_a_b} \right|_p \tensor{u}{^a} \left. \tensor{\xi}{^b} \right|_p \right)\\
					&= \left( \phi_{t_1} \right)_* 0\\
					&= 0,
				\end{split}
			\end{equation*}
			故 $\Sigma_{t_1}$ 也与 $\tensor{\xi}{^a}$ 正交。
		\end{Proof}
	\end{zm}

	\item 设 $\gamma(r)$ 是图~\hyperlink{8-6}{8-6}~中 $\Sigma_{t}$ 上从 $p_1$ 到 $p_2$ 的、$\theta$ 和 $\varphi$ 都为常数的曲线(以径向坐标 $r$ 为曲线参数),试证 $\gamma(r)$ 是(非仿射参数化的)测地线。提示:用式(5-7-2)。

	\begin{figure}[htpb!]
		\centering
		\begin{tikzpicture}[thick]
			\draw (0,0) ellipse [x radius=1,y radius=.2];
			\draw (0,0) ellipse [x radius=2.1, y radius=0.4];
			\draw (0,0) ellipse [x radius=3, y radius=0.74];
			\draw (0,0) ellipse [x radius=4, y radius=1.2];
			\filldraw[fill=white,color=white] (-4.5,1.5) -- (4.5,1.5) -- (4.5,-1.5) -- (-4.5,-1.5) -- cycle (-2.5,1.11) -- (-4.1,-1.11) -- (4.1,-1.11) -- (2.5,1.09) -- cycle;
			\draw (-2.5,1.11) -- (-4.1,-1.11) -- (4.1,-1.11) -- (2.5,1.09) -- cycle;
			\draw[dashed] (0,0) -- (3.3,0);
			\draw[-{Stealth[length=0.4cm,width'=0pt 0.3,inset=1pt]}] (1,0) -- (1.99,0);
			\node (n) at (1.5,-0.32) {\colorbox{white}{$\smash[b]{\tensor{n}{^a}}$}};
			\node (Sigma) at (3.3,-1.3) {$\Sigma$};
			\fill[fill=white] (2.15,0.4) -- (2.15,1.2) -- (2.05,1.2) -- (2.05,0.4) -- cycle;
			\fill[fill=white] (-2.15,0.4) -- (-2.15,1.2) -- (-2.05,1.2) -- (-2.05,0.4) -- cycle;
			\draw[-{Stealth[length=0.5cm,width'=0pt 0.3,inset=1pt]}] (2.1,0) -- (2.1,1.7);
			\draw[-{Stealth[length=0.5cm,width'=0pt 0.3,inset=1pt]}] (-2.1,0) -- (-2.1,1.7);
			\node[left] (xi1) at (2.1,1.5) {$\tensor{\xi}{^a}$};
			\node[right] (xi2) at (-2.05,1.5) {$\tensor{\xi}{^a}$};
		\end{tikzpicture}
		\caption{正文图 8-6}\hypertarget{8-6}{}
	\end{figure}

	\begin{zm}
		$\gamma(r)$ 即为 $r$ 坐标线,故切矢为 $\tensor{\dot{\gamma}}{^a} = \tensor{\left( \pdv{r} \right)}{^a}$。正文 (8-3-4) 列出了球对称度规 (8-3-2) 式
		\begin{equation*}
			{\dd{s}}^2 = - \e{2A(r)} \dd{t}^2 + \e{2B(r)} \dd{r}^2 + r^2 \left( \dd{\theta}^2 + \sin^2 \theta \dd{\varphi}^2 \right) \tag{8-3-2}
		\end{equation*}
		的所有非零克氏符为
		\begin{equation*}
			\begin{aligned}
				\ChristoffelSymbol{0}{0}{1} &= \ChristoffelSymbol{0}{1}{0} = A', & \ChristoffelSymbol{1}{0}{0} &= A' \e{2(A-B)}, & \ChristoffelSymbol{1}{1}{1} &= B',\\
				\ChristoffelSymbol{1}{2}{2} &= - r \e{-2B}, & \ChristoffelSymbol{1}{3}{3} &= - r \sin^2 \theta \e{-2B}, & \ChristoffelSymbol{2}{1}{2} &= \ChristoffelSymbol{2}{2}{1} = 1/r,\\
				\ChristoffelSymbol{2}{3}{3} &= - \sin \theta \cos \theta, & \ChristoffelSymbol{3}{1}{3} &= \ChristoffelSymbol{3}{3}{1} = 1/r, & \ChristoffelSymbol{3}{2}{3} &= \ChristoffelSymbol{3}{3}{2} = \cot \theta,
			\end{aligned}
			\tag{8-3-4}
		\end{equation*}
		故
		\begin{equation*}
			\begin{split}
				\tensor{\dot{\gamma}}{^a} \Nabla{a} \tensor{\dot{\gamma}}{^b} &= \tensor{\left( \pdv{r} \right)}{^a} \Nabla{a} \tensor{\left( \pdv{r} \right)}{^b}\\
				&= \ChristoffelSymbol{\mu}{1}{1} \tensor{\left( \pdv{x^\mu} \right)}{^b}\\
				&= \ChristoffelSymbol{1}{1}{1} \tensor{\left( \pdv{r} \right)}{^b}\\
				&= \dv{B}{r} \tensor{\left( \pdv{r} \right)}{^b},
			\end{split}
		\end{equation*}
		故 $\gamma(r)$ 是测地线。
	\end{zm}

	\item 设 $\tensor{\xi}{^a}$ 是稳态时空的类时 Killing 矢量场,$\chi \equiv \left( - \tensor{g}{_a_b} \tensor{\xi}{^a} \tensor{\xi}{^b} \right)^{1/2}$。
	\begin{enumerate}[label = (\alph*)]
		\item 试证 $\chi$ 在 $\tensor{\xi}{^a}$ 的积分曲线上为常数;
		\item 试证稳态观者的 4 加速 $\tensor{A}{^a} = \tensor{\nabla}{^a} (\ln \chi)$。提示:利用 Killing 方程 $\tensor{\nabla}{^{(a}} \tensor{\xi}{^{b)}} = 0$ 和 (a) 的结果。
	\end{enumerate}

	\begin{zm}
		\begin{enumerate}[label = (\alph*)]
			\item 易知
			\begin{equation*}
				\begin{split}
					\Ld{\xi} \chi^2 &= - \Ld{\xi} \left( \tensor{g}{_a_b} \tensor{\xi}{^a} \tensor{\xi}{^b} \right)\\
					&= - \left( \Ld{\xi} \tensor{g}{_a_b} \right) \tensor{\xi}{^a} \tensor{\xi}{^b} - 2 \tensor{g}{_a_b} \tensor{\xi}{^a} \left( \Ld{\xi} \tensor{\xi}{^b} \right)\\
					&= 0,
				\end{split}
			\end{equation*}
			故 $\chi$ 沿 $\xi$ 积分曲线为定值。
			\item 稳态观者 $\tensor{Z}{^a} = \tensor{\xi}{^a}/\chi$,故
			\begin{align*}
				\tensor{A}{^a} &= \tensor{Z}{^b} \Nabla{b} \tensor{A}{^a}\\
				&= \frac{1}{\chi} \tensor{\xi}{^b} \Nabla{b} \left( \frac{\tensor{\xi}{^a}}{\chi} \right)\\
				&= \frac{1}{\chi^2} \tensor{\xi}{_b} \tensor{\nabla}{^b} \tensor{\xi}{^a}\\
				&= - \frac{1}{\chi^2} \tensor{\xi}{_b} \tensor{\nabla}{^a} \tensor{\xi}{^b}\\
				&= - \frac{1}{2 \chi^2} \tensor{\nabla}{^a} \left( \tensor{\xi}{_b} \tensor{\xi}{^b} \right)\\
				&= \frac{1}{2 \chi^2} \tensor{\nabla}{^a} \chi^2\\
				&= \frac{1}{\chi} \tensor{\nabla}{^a} \chi\\
				&= \tensor{\nabla}{^a} \left( \ln \chi \right).
			\end{align*}
		\end{enumerate}
	\end{zm}

	\item 试证:(a) 电磁场能动张量的迹为零,即 $T\equiv \tensor{g}{^a^b} \tensor{T}{_a_b} = 0$;(b) 电磁真空时空的标量曲率 $R=0$。

	\begin{zm}
		\begin{enumerate}[label = (\alph*)]
			\item 由
			\begin{equation*}
				\tensor{T}{_a_b} = \frac{1}{4\pi} \left( \tensor{F}{_a_c} \tensor{F}{_b^c} - \frac{1}{4} \tensor{g}{_a_b} \tensor{F}{_c_d} \tensor{F}{^c^d} \right)
			\end{equation*}
			知
			\begin{equation*}
				\begin{split}
					T &= \tensor{g}{^a^b} \tensor{T}{_a_b}\\
					&= \frac{1}{4\pi} \left( \tensor{F}{_a_c} \tensor{F}{^a^c} - \tensor{F}{_c_d} \tensor{F}{^c^d} \right)\\
					&= 0.
				\end{split}
			\end{equation*}
			\item 对场方程取迹得
			\begin{equation*}
				8\pi T = R - 2 R = -R,
			\end{equation*}
			故对于电磁真空时空,由于 $T=0$,知 $R = 0$。
		\end{enumerate}
	\end{zm}

	\item 试证式 (8-4-7) 和 (8-4-28)\footnote{正文 (8-4-7) 为
	\begin{equation*}
		\tensor{\Sigma}{_a_b} \tensor{\Sigma}{^a^b} = 2 \left( \tensor{F}{_a_b} \tensor{F}{^a^b} + \ii \tensor{F}{_a_b} \tensor[^*]{F}{^a^b} \right), \tag{8-4-7}
	\end{equation*}
	正文 (8-4-28) 为
	\begin{equation*}
		\tensor{E}{^a} = \frac{Q}{r^2} \tensor{\left( e_1 \right)}{^a} \qc \tensor{B}{^a} = 0 \quad \text{[其中 $\tensor{\left( e_1 \right)}{^a} \equiv f^{1/2} \tensor{\left( \pdv*{r} \right)}{^a}$]}. \tag{8-4-28}
	\end{equation*}}。

	\begin{zm}
		先证 (8-4-7) 式:
		\begin{equation*}
			\begin{split}
				\tensor{\Sigma}{_a_b} \tensor{\Sigma}{^a^b} &= \left( \tensor{F}{_a_b} + \ii \tensor[^*]{F}{_a_b} \right) \left( \tensor{F}{^a^b} + \ii \tensor[^*]{F}{^a^b} \right)\\
				&= \tensor{F}{_a_b} \tensor{F}{^a^b} - \tensor[^*]{F}{_a_b} \tensor[^*]{F}{^a^b} + 2\ii \tensor{F}{_a_b} \tensor[^*]{F}{^a^b},
			\end{split}
		\end{equation*}
		而
		\begin{equation}
			\begin{split}
				\tensor[^*]{F}{_a_b} \tensor[^*]{F}{^a^b} &= \frac{1}{2} \tensor{F}{^c^d} \tensor{\varepsilon}{_c_d_a_b} \times \frac{1}{2} \tensor{F}{_e_f} \tensor{\varepsilon}{^e^f^a^b}\\
				&= - \frac{1}{4} \times 2! \times 2! \tensor{\delta}{^{[e}_c} \tensor{\delta}{^{f]}_d} \tensor{F}{^c^d} \tensor{F}{_e_f}\\
				&= - \tensor{F}{^a^b} \tensor{F}{_a_b},
			\end{split}\label{eq-F*F*}
		\end{equation}
		故
		\begin{equation}
			\begin{split}
				\tensor{\Sigma}{_a_b} \tensor{\Sigma}{^a^b} = 2 \left( \tensor{F}{_a_b} \tensor{F}{^a^b} + \ii \tensor{F}{_a_b} \tensor[^*]{F}{^a^b} \right).\label{eq-SS}
			\end{split}
		\end{equation}
		而由式 (8-4-27)
		\begin{equation*}
			\tensor{F}{_a_b} = - \frac{Q}{r^2} \tensor{\left( \dd{t} \right)}{_a} \wedge \tensor{\left( \dd{r} \right)}{_b}
		\end{equation*}
		以及
		\begin{equation*}
			\tensor{Z}{^a} = f^{-1/2} \tensor{\left( \pdv*{t} \right)}{^a}
		\end{equation*}
		得
		\begin{equation*}
			\begin{split}
				\tensor{E}{_a} &= \tensor{F}{_a_b} \tensor{Z}{^b}\\
				&= \frac{Q}{r^2} \tensor{\left( \dd{r} \right)}{_a}\\
				&= \frac{Q}{r^2} \tensor{\left( e^1 \right)}{_a},\\
				\tensor{B}{_a} &= - \tensor[^*]{F}{_a_b} \tensor{Z}{^b}\\
				&= \frac{1}{2} \tensor{F}{^c^d} \tensor{\varepsilon}{_c_d_a_b} \tensor{Z}{^b}\\
				&= - \frac{Q}{r^2} \tensor{\varepsilon}{_c_d_a_b} \tensor{\left( e_0 \right)}{^c} \tensor{\left( e_1 \right)}{^d} \tensor{\left( e_0 \right)}{^b}\\
				&= 0,
			\end{split}
		\end{equation*}
		其中 $\tensor{\left( e_0 \right)}{^a} = \tensor{Z}{^a}$ 是归一化的第 0 基矢。
	\end{zm}

	\item 设 $\tensor{F}{_a_b}$ 是任意时空中的2形式场,$\tensor[^*]{F}{_a_b}$ 是 $\tensor{F}{_a_b}$ 的对偶 2 形式场, $\alpha \in \left[ 0,2\pi \right]$ 为常实数,则 $\tensor{{F'}}{_a_b} \equiv \tensor{F}{_a_b} \cos \alpha - \tensor[^*]{F}{_a_b} \sin \alpha$ 称为 $\tensor{F}{_a_b}$ 的、角度为 $\alpha$ 的一个\textbf{对偶转动}(duality rotation)。
	\begin{enumerate}[label = (\alph*)]
		\item 试证 $\tensor{F}{_a_b}$ 为无源电磁场当且仅当 $\tensor{{F'}}{_a_b}$ 为无源电磁场[证明很易。若用麦氏方程的外微分表达式 (7-2-7') 和 (7-2-8') \footnote{这里为typo,应为 (7-2-4') 和 (7-2-5'):
		\begin{gather*}
			\dd{\tensor[^*]{\!\form{F}}{}} = 4 \pi \tensor[^*]{\!\form{J}}{}, \tag{7-2-4'}\\
			\dd{\form{F}} = 0. \tag{7-2-5'}
		\end{gather*}}甚至一望便知。]。
		\item 试证电磁场 $\tensor{F}{_a_b}$ 和 $\tensor{{F'}}{_a_b}$ 有相同能动张量。提示:用 $\tensor{T}{_a_b}$ 的对称表示式 (6-6-24') \footnote{
			这里为typo,应为 (6-6-28'):
			\begin{equation*}
				\tensor{T}{_a_b} = \frac{1}{8\pi} \left( \tensor{F}{_a_c} \tensor{F}{_b^c} + \tensor[^*]{F}{_a_c} \tensor[^*]{F}{_b^c} \right). \tag{6-6-28'}
			\end{equation*}
		}可简化证明。
		\item 令 $M\equiv 2 \tensor{F}{_a_b} \tensor{F}{^a^b}$,$N\equiv 2 \tensor{F}{_a_b} \tensor[^*]{F}{^a^b}$ ,$M' \equiv 2 \tensor{{F'}}{_a_b} \tensor{{F'}}{^a^b}$,$N' \equiv 2 \tensor{{F'}}{_a_b} \tensor[^*]{{F'}}{^a^b}$,试证
		\begin{equation*}
			M' = M \cos 2\alpha - N \sin 2\alpha \qc N' = M \sin 2\alpha + N \cos 2\alpha.
		\end{equation*}
		\item 令 $\tensor{\Sigma}{_a_b} \equiv \tensor{F}{_a_b} + \ii \tensor[^*]{F}{_a_b}$,$\tensor{{\Sigma'}}{_a_b} \equiv \tensor{{F'}}{_a_b} + \ii \tensor[^*]{{F'}}{_a_b}$,则 $K \equiv \tensor{\Sigma}{_a_b} \tensor{\Sigma}{^a^b}$ 和 $K' \equiv \tensor{{\Sigma'}}{_a_b} \tensor{{\Sigma'}}{^a^b}$ 为复标量场,故在每一时空点的 $K$ 和 $K'$ 相当于复平面上的两个矢量。试用 (c) 的结果证明矢量 $K'$ 是矢量 $K$ 逆时针转 $2\alpha$ 角的结果(即 $\abs{K} = \abs{K'}$,$K'$ 和 $K$ 的辐角差为 $2\alpha$。)。
		\item 设 $\left( \myvec{E}, \myvec{B} \right)$ 和 $\left( \myvec{E}' , \myvec{B}' \right)$ 是瞬时观者分别测 $\tensor{F}{_a_b}$ 和 $\tensor{{F'}}{_a_b}$ 所得的电场和磁场,试证
		\begin{equation*}
			\myvec{E}' = \myvec{E} \cos \alpha + \myvec{B} \sin \alpha \qc \myvec{B}' = - \myvec{E} + \myvec{B} \cos \alpha.
		\end{equation*}
	\end{enumerate}
	注:对偶转动的进一步物理意义见本书下册及 \cite{Jackson1975}。

	\begin{zm}
		\begin{enumerate}[label = (\alph*)]
			\item 无源麦氏方程为
			\begin{gather*}
				\dd{\tensor[^*]{\!\form{F}}{}} = 0,\\
				\dd{\form{F}} = 0,
			\end{gather*}
			则结论显然。
			\item 由
			\begin{equation*}
				\tensor[^*^*]{\form{F}}{} = - \form{F}
			\end{equation*}
			知
			\begin{equation*}
				\begin{split}
					\tensor{{T'}}{_a_b} ={}& \frac{1}{8\pi} \left( \tensor{{F'}}{_a_c} \tensor{{F'}}{_b^c} + \tensor[^*]{{F'}}{_a_c} \tensor[^*]{{F'}}{_b^c} \right)\\
					={}& \frac{1}{8\pi} \big[ \left( \tensor{F}{_a_c} \cos \alpha - \tensor[^*]{F}{_a_c} \sin \alpha \right) \left( \tensor{F}{_b^c} \cos \alpha - \tensor[^*]{F}{_b^c} \sin \alpha \right)\\
					&\phantom{\frac{1}{8\pi} \big[} {}+ \left( \tensor[^*]{F}{_a_c} \cos \alpha + \tensor{F}{_a_c} \sin \alpha \right) \left( \tensor[^*]{F}{_b^c} \cos \alpha + \tensor{F}{_b^c} \sin \alpha \right) \big]\\
					={}& \frac{1}{8\pi} \big( \tensor{F}{_a_c} \tensor{F}{_b^c} \cos^2 \alpha - \tensor{F}{_{(a|c|}} \tensor[^*]{F}{_{b)}^c} \sin 2 \alpha + \tensor[^*]{F}{_a_c} \tensor[^*]{F}{_b^c} \sin^2 \alpha\\
					&\phantom{\frac{1}{8\pi} \big(} {}+ \tensor[^*]{F}{_a_c} \tensor[^*]{F}{_b^c} \cos^2 \alpha + \tensor{F}{_{(a|c|}} \tensor[^*]{F}{_{b)}^c} \sin 2 \alpha + \tensor{F}{_a_c} \tensor{F}{_b^c} \sin^2 \alpha \big)\\
					={}& \frac{1}{8\pi} \left( \tensor{F}{_a_c} \tensor{F}{_b^c} + \tensor[^*]{F}{_a_c} \tensor[^*]{F}{_b^c} \right)\\
					={}& \tensor[]{T}{_a_b}.
				\end{split}
			\end{equation*}
			\item 由第5 题计算过程~\eqref{eq-F*F*} 知
			\begin{equation}
					\tensor[^*]{F}{_a_b} \tensor[^*]{F}{^a^b} = - \tensor{F}{_a_b} \tensor{F}{^a^b},
			\end{equation}
			故
			\begin{align*}
					M' &= 2 \tensor[]{{F'}}{_a_b} \tensor[]{{F'}}{^a^b}\\
					&= 2 \left( \tensor{F}{_a_b} \cos \alpha - \tensor[^*]{F}{_a_b} \sin \alpha \right) \left( \tensor{F}{^a^b} \cos \alpha - \tensor[^*]{F}{^a^b} \sin \alpha \right)\\
					&= 2 \left( \tensor[]{F}{_a_b} \tensor[]{F}{^a^b} \cos^2 \alpha - 2 \tensor{F}{_a_b} \tensor[^*]{F}{^a^b} \cos \alpha \sin \alpha + \tensor[^*]{F}{_a_b} \tensor[^*]{F}{^a^b} \sin^2 \alpha \right)\\
					&= 2 \left( \tensor[]{F}{_a_b} \tensor[]{F}{^a^b} \cos^2 \alpha - 2 \tensor{F}{_a_b} \tensor[^*]{F}{^a^b} \cos \alpha \sin \alpha - \tensor{F}{_a_b} \tensor{F}{^a^b} \sin^2 \alpha \right)\\
					&= M \cos 2\alpha - N \sin 2 \alpha,\\
					N' &= 2 \tensor{{F'}}{_a_b} \tensor[^*]{{F'}}{^a^b}\\
					&= 2 \left( \tensor{F}{_a_b} \cos \alpha - \tensor[^*]{F}{_a_b} \sin \alpha \right) \left( \tensor[^*]{F}{^a^b} \cos \alpha + \tensor{F}{^a^b} \sin \alpha \right)\\
					&= 2 \left( \tensor{F}{_a_b} \tensor[^*]{F}{^a^b} \cos^2 \alpha + \tensor{F}{_a_b} \tensor{F}{^a^b} \cos \alpha \sin \alpha - \tensor[^*]{F}{_a_b} \tensor[^*]{F}{^a^b} \cos \alpha \sin \alpha - \tensor{F}{_a_b} \tensor[^*]{F}{^a^b} \sin^2 \alpha \right)\\
					&= 2 \left( \tensor{F}{_a_b} \tensor[^*]{F}{^a^b} \cos^2 \alpha + \tensor{F}{_a_b} \tensor{F}{^a^b} \cos \alpha \sin \alpha + \tensor{F}{_a_b} \tensor{F}{^a^b} \cos \alpha \sin \alpha - \tensor{F}{_a_b} \tensor[^*]{F}{^a^b} \sin^2 \alpha \right)\\
					&= M \sin 2 \alpha + N \cos 2\alpha.
			\end{align*}
			\item 由第 5 题~\eqref{eq-SS} 知
			\begin{equation*}
				\begin{split}
					K &= \tensor{\Sigma}{_a_b} \tensor{\Sigma}{^a^b}\\
					&= 2 \left( \tensor{F}{_a_b} \tensor{F}{^a^b} + \ii \tensor{F}{_a_b} \tensor[^*]{F}{^a^b} \right)\\
					&= M + \ii N,
				\end{split}
			\end{equation*}
			故
			\begin{equation*}
				\begin{split}
					K' &= M' + \ii N'\\
					&= \left( M \cos 2\alpha - N \sin 2 \alpha \right) + \ii \left( M \sin 2 \alpha + N \cos 2\alpha \right)\\
					&= M \left( \cos 2\alpha + \ii \sin 2 \alpha \right) + \ii N \left( \cos 2 \alpha + \ii \sin 2 \alpha \right)\\
					&= K \e{2\ii \alpha}.
				\end{split}
			\end{equation*}
			\item 由定义知,设观者为 $\tensor{Z}{^a}$,
			\begin{align*}
				\tensor{{E'}}{_a} &= \tensor{{F'}}{_a_b} \tensor{Z}{^b}\\
				&= \left( \tensor{F}{_a_b} \cos \alpha - \tensor[^*]{F}{_a_b} \sin \alpha \right) \tensor{Z}{^b}\\
				&= \tensor{E}{_a} \cos \alpha + \tensor{B}{_a} \sin \alpha,\\
				\tensor{{B'}}{_a} &= - \tensor[^*]{{F'}}{_a_b} \tensor{Z}{^b}\\
				&= - \left( \tensor[^*]{F}{_a_b} \cos \alpha + \tensor{F}{_a_b} \sin \alpha \right) \tensor{Z}{^a}\\
				&= \tensor{B}{_a} \cos \alpha - \tensor{E}{_a} \sin \alpha.
			\end{align*}
		\end{enumerate}
	\end{zm}

	\item $n$ 维时空称为\textbf{爱因斯坦时空},若 $\tensor{R}{_a_b} = R \tensor{g}{_a_b}/2$,其中 $\tensor{g}{_a_b}$ , $\tensor{R}{_a_b}$ 和 $R$ 分别为度规、里奇张量和标量曲率。试证电磁真空时空(其中电磁场非零)。注:由第 3 章习题 17 可知任意 2 维时空必为爱因斯坦时空。

	\begin{zm}
		设某电磁真空时空为爱因斯坦时空,则
		\begin{equation*}
			\tensor{G}{_a_b} = 0,
		\end{equation*}
		由场方程知 $\tensor{T}{_a_b} = 0$。但对任一观者
		\begin{equation*}
			\tensor{T}{_0_0} = \frac{1}{8\pi} \left( E^2 + B^2 \right) = 0 \implies \tensor{E}{_a} = \tensor{B}{_a} = 0,
		\end{equation*}
		故 $\tensor{F}{_a_b} = 0$,与电磁场非零矛盾。
	\end{zm}

	\item 考虑 Taub 的平面对称真空解 (8-6-1')\footnote{正文(8-6-1')为
	\begin{equation*}
		\dd{s}^2 = z^{-1/2} \left( -\dd{t}^2 + \dd{z}^2 \right) + z \left( \dd{x}^2 + \dd{y}^2 \right). \tag{8-6-1'}
	\end{equation*}
	}。
	\begin{enumerate}[label=(\alph*)]
		\item 写出静态观者的4速用坐标基矢的表达式;
		\item 设两静态观者的空间坐标分别为 $(x,y,z_1)$ 和 $(x,y,z_2)$,求他们的空间距离。
	\end{enumerate}

	\begin{jie}
		\begin{enumerate}[label=(\alph*)]
			\item 由于 Taub 解
			\begin{equation*}
				\dd{s}^2 = z^{-1/2} \left( -\dd{t}^2 + \dd{z}^2 \right) + z \left( \dd{x}^2 + \dd{y}^2 \right)
			\end{equation*}
			不含 $t$ 坐标,故 $\tensor{\left( \pdv{t} \right)}{^a}$ 是 Killing 矢量场,且易知与 $x$-$y$ 平面正交,而
			\begin{equation*}
				\tensor{g}{_a_b} \tensor{\left( \pdv{t} \right)}{^a} \tensor{\left( \pdv{t} \right)}{^b} = \tensor{g}{_0_0} = - z^{-1/2},
			\end{equation*}
			故
			\begin{equation*}
				\tensor{Z}{^a} = z^{1/4} \tensor{\left( \pdv{t} \right)}{^a}
			\end{equation*}
			是归一化的静态观者。
			\item 空间诱导度规为
			\begin{equation*}
				\dd{\hat{s}}^2 = z\left( \dd{x}^2 + \dd{y}^2 \right) + z^{-1/2} \dd{z}^2,
			\end{equation*}
			用第三章的代码让麦酱算得非零克氏符为
			\begin{equation*}
				\ChristoffelSymbol{1}{1}{3} = \ChristoffelSymbol{1}{3}{1} = \ChristoffelSymbol{2}{2}{3} = \ChristoffelSymbol{2}{3}{2} = \frac{1}{2z} \qc \ChristoffelSymbol{3}{1}{1} = \ChristoffelSymbol{3}{2}{2} = - \frac{\sqrt{z}}{2} \qc \ChristoffelSymbol{3}{3}{3} = - \frac{1}{4z},
			\end{equation*}
			故测地线方程为
			\begin{equation*}
				\left\{
					\begin{aligned}
						\ddot{x} &= - \frac{1}{z} \dot{x} \dot{z},\\
						\ddot{y} &= - \frac{1}{z} \dot{y} \dot{z},\\
						\ddot{z} &= \frac{\sqrt{z}}{2} \left( \dot{x}^2 + \dot{y}^2 \right) + \frac{1}{4z} \dot{z}^2,
					\end{aligned}
				\right.
			\end{equation*}
			此方程组的通解不易解,但容易猜测 $z$ 坐标线是解,将 $\dot{x} = \dot{y} = 0$ 代入,前两式成立,第三式为
			\begin{equation*}
				\ddot{z} = \frac{1}{4z} \dot{z}^2,
			\end{equation*}
			% 令 $u=\dot{z}$,则
			% \begin{equation*}
			% 	\ddot{z} = \dvt{\dot{z}} = \dv{\dot{z}}{z} \dv{z}{t} = u \dv{u}{z},
			% \end{equation*}
			% 故
			% \begin{align*}
			% 	u \dv{u}{z} &= \frac{u^2}{4z}\\
			% 	\implies \frac{\dd{u}}{u} &= \frac{\dd{z}}{4z}\\
			% 	\implies \ln{u} &= \frac{1}{4} \ln{z} + C\\
			% 	\implies u &= C_1 z^{1/4},
			% \end{align*}
			% 即
			% \begin{gather*}
			% 	\dv{z}{t} = C_1 z^{1/4}\\
			% 	\implies \frac{\dd{z}}{z^{1/4}} = C_1 \dd{t}\\
			% 	\implies z = \left( \frac{3}{4} \left( C_1 t + C_2 \right) \right)^{4/3} = (a t + b)^{4/3},
			% \end{gather*}
			% 故 $z$ 坐标线重参数化是解。
			显然必有解,故 $z$ 坐标线重参数化的确是解。
			\begin{tcolorbox}[breakable,title=另一验证方法,fonttitle=\normalfont\bfseries]
				除了观察ODEs,也可直接计算
				\begin{equation*}
					\begin{split}
						\tensor{\left( \pdv{z} \right)}{^a} \Nabla{a} \tensor{\left( \pdv{z} \right)}{^b} &= \ChristoffelSymbol{\mu}{3}{3} \tensor{\left( \pdv{x^\mu} \right)}{^b}\\
						&= -\frac{1}{4z} \tensor{\left( \pdv{z} \right)}{^b},
					\end{split}
				\end{equation*}
				故知是测地线。
			\end{tcolorbox}
			由于
			\begin{equation*}
				\tensor{h}{_a_b} \tensor{\left( \pdv{z} \right)}{^a} \tensor{\left( \pdv{z} \right)}{^b} = \tensor{h}{_3_3} = z^{-1/2},
			\end{equation*}
			故从 $\left( x,y,z_1 \right)$ 到 $(x,y,z_2)$ 的线长为
			\begin{equation*}
				\int_{z_1}^{z_2} \sqrt{\tensor{h}{_a_b} \tensor{\left( \pdv{z} \right)}{^a} \tensor{\left( \pdv{z} \right)}{^b}} \dd{z} = \int_{z_1}^{z_2} z^{-1/4} \dd{z} = \frac{4}{3} \left( z_2^{3/4} - z_1^{3/4} \right).
			\end{equation*}
		\end{enumerate}
	\end{jie}

	\item 试证式 (8-6-5)\footnote{正文 (8-6-5) 为
	\begin{equation*}
		\tensor{F}{_1_2} = C_1 \qc \tensor{F}{_3_0} = C_2 Y' Y^{-2}/2 \qc A \equiv 4\pi \left( C_1^2 + C_2^2 \right)/C \qc C_1,C_2 \text{为常数}. \tag{8-6-5}
	\end{equation*}} 的 $\tensor{F}{_a_b}$ 有平面对称性,即 $\Ld{\xi_i} \tensor{F}{_a_b} = 0 (i=1,2,3)$,其中 $\tensor{\xi}{_1^a}\equiv \tensor{\left( \pdv*{x} \right)}{^a}$, $\tensor{\xi}{_2^a} \equiv \tensor{\left( \pdv*{y} \right)}{^a}$, $\tensor{\xi}{_3^a} \equiv - y \tensor{\left( \pdv*{x} \right)}{^a} + x \tensor{\left( \pdv*{y} \right)}{^a}$ 是反映度规 (8-6-3)\footnote{正文 (8-6-3) 为
	\begin{equation*}
		\dd{s}^2 = \frac{1}{2} Y'(z) \left( -\dd{t}^2 + \dd{z}^2 \right) + Y^2(z) \left( \dd{x}^2 + \dd{y}^2 \right). \tag{8-6-3}
	\end{equation*}} 平面对称性的 Killing 场。

	\begin{zm}
		由于坐标系 $\left\{ t,x,y,z \right\}$ 是 $\tensor{\left( \pdv*{x} \right)}{^a}$、$\tensor{\left( \pdv*{y} \right)}{^a}$ 的适配坐标系,而 $\tensor{F}{_\mu_\nu}$ 不含 $x,y$ 坐标,结论显然。
	\end{zm}

	\item 推出有源麦氏方程在 NP 形式中的表达式。答案:在式 (8-8-3) 的每式右边各加一项,依次为 $-4\pi \tensor{J}{_4}$ ,$-4\pi\tensor{J}{_2}$,$-4\pi\tensor{J}{_1}$,$-4\pi\tensor{J}{_3}$($\tensor{J}{_1}$,$\tensor{J}{_2}$,$\tensor{J}{_3}$,$\tensor{J}{_4}$ 是 $\tensor{J}{_a}$ 在类光标架的分量)。

	\begin{jie}
		有源麦氏方程为
		\begin{gather}
			\tensor{\nabla}{^a} \tensor{F}{_a_b} = - 4 \pi \tensor{J}{_b},\label{eq-c8_Me1}\\
			\tensor{\nabla}{_{[a}} \tensor{F}{_{bc]}} = 0,\label{eq-c8_Me2}
		\end{gather}
		% 先改写~\eqref{eq-c8_Me1},用
		% \begin{equation*}
		% 	\tensor{\delta}{^a_b} = \tensor{m}{^a} \tensor{\bar{m}}{_b} + \tensor{\bar{m}}{^a} \tensor{m}{_b} - \tensor{l}{^a} \tensor{k}{_b} - \tensor{k}{^a} \tensor{l}{_b}
		% \end{equation*}
		% 缩并得
		由
		\begin{equation*}
			\tensor{F}{_a_b} = \tensor{F}{_\mu_\nu} \tensor{\left( \varepsilon^\mu \right)}{_a} \tensor{\left( \varepsilon^\nu \right)}{_b}
		\end{equation*}
		得
		% \begin{align*}
		% 	\tensor{F}{_a_b} ={} & 2 \tensor{F}{_2_1} \tensor{m}{_{[a}} \tensor{\bar{m}}{_{b]}} - 2 \tensor{F}{_3_1} \tensor{k}{_{[a}} \tensor{\bar{m}}{_{b]}} - 2 \tensor{F}{_4_1} \tensor{l}{_{[a}} \tensor{\bar{m}}{_{b]}}\\
		% 	& {} - 2 \tensor{F}{_2_3} \tensor{m}{_{[a}} \tensor{k}{_{b]}} - 2 \tensor{F}{_4_2} \tensor{l}{_{[a}} \tensor{m}{_{b]}} + 2 \tensor{F}{_4_3} \tensor{l}{_{[a}} \tensor{k}{_{b]}}\\
		% 	={} & - 4 \ii \Im \Phi_1 \tensor{m}{_{[a}} \tensor{\bar{m}}{_{b]}} + 2 \bar{\Phi}_2 \tensor{k}{_{[a}} \tensor{\bar{m}}{_{b]}} - 2 \Phi_0 \tensor{l}{_{[a}} \tensor{\bar{m}}{_{b]}}\\
		% 	& {} - 2 \Phi_2 \tensor{m}{_{[a}} \tensor{k}{_{b]}} - 2 \bar{\Phi}_0 \tensor{l}{_{[a}} \tensor{m}{_{b]}} + 4 \Re \Phi_1 \tensor{l}{_{[a}} \tensor{k}{_{b]}},
		% \end{align*}
		% 故
		% \begin{align*}
		% 	\tensor{\nabla}{^a} \tensor{F}{_a_b} ={} & -4 \ii \tensor{m}{_{[a}} \tensor{\bar{m}}{_{b]}} \Im \tensor{\nabla}{^a} \Phi_1 - 2 \ii \Im \Phi_1 \tensor{\bar{m}}{_{[b}} \tensor{\nabla}{^a} \tensor{m}{_{a]}} - 2 \ii \Im \Phi_1 \tensor{m}{_{[a}} \tensor{\nabla}{^a} \tensor{\bar{m}}{_{b]}}\\
		% 	&{} + 2 \tensor{k}{_{[a}} \tensor{\bar{m}}{_{b]}} \tensor{\nabla}{^a} \bar{\Phi}_2 + \bar{\Phi}_2 \tensor{\bar{m}}{_{[b}} \tensor{\nabla}{^a} \tensor{k}{_{a]}} + \bar{\Phi}_2 \tensor{k}{_{[a}} \tensor{\nabla}{^a} \tensor{\bar{m}}{_{b]}}
		% \end{align*}
		\begin{align*}
			\tensor{\left( \varepsilon_\mu \right)}{^b} \tensor{\nabla}{^a} \tensor{F}{_a_b} &= \tensor{\left( \varepsilon_\mu \right)}{^b} \tensor{\nabla}{^a} \left( \tensor{F}{_\nu_\sigma} \tensor{\left( \varepsilon^\nu \right)}{_a} \tensor{\left( \varepsilon^\sigma \right)}{_b} \right)\\
			&= \tensor{\left( \varepsilon^\nu \right)}{_a} \tensor{\nabla}{^a} \tensor{F}{_\nu_\mu} + \tensor{F}{_\nu_\mu} \tensor{\nabla}{^a} \tensor{\left( \varepsilon^\nu \right)}{_a} + \tensor{F}{_\nu_\sigma} \tensor{\left( \varepsilon_\mu \right)}{^b} \tensor{\left( \varepsilon^\nu \right)}{_a} \tensor{\nabla}{^a} \tensor{\left( e^\sigma \right)}{_b}\\
			&= \tensor{\left( \varepsilon^\nu \right)}{_a} \tensor{\nabla}{^a} \tensor{F}{_\nu_\mu} + \tensor{F}{_\nu_\mu} \tensor{\omega}{_\sigma^\nu^\sigma} + \tensor{F}{_\nu_\sigma} \tensor{\omega}{_\mu^\sigma^\nu}\\
			&= \tensor{\left( \varepsilon^\nu \right)}{_a} \tensor{\nabla}{^a} \tensor{F}{_\nu_\mu} + \tensor{g}{^\sigma^\rho} \tensor{F}{_\nu_\mu} \tensor{\omega}{_\sigma^\nu_\rho} + \tensor{g}{^\nu^\rho} \tensor{F}{_\nu_\sigma} \tensor{\omega}{_\mu^\sigma_\rho}\\
			&= \tensor{\left( \varepsilon^\nu \right)}{_a} \tensor{\nabla}{^a} \tensor{F}{_\nu_\mu} + \tensor{g}{^\sigma^\rho} \tensor{F}{_\nu_\mu} \tensor{\omega}{_\sigma^\nu_\rho} + \tensor{g}{^\sigma^\rho} \tensor{F}{_\sigma_\nu} \tensor{\omega}{_\mu^\nu_\rho}\\
			&= \tensor{\left( \varepsilon^\nu \right)}{_a} \tensor{\nabla}{^a} \tensor{F}{_\nu_\mu} + 2 \tensor{F}{_\nu_{[\mu}} \tensor{\omega}{_{\sigma]}^\nu_\rho} \tensor{g}{^\sigma^\rho},
		\end{align*}
		又因
		\begin{align*}
			&\tensor{\left( \varepsilon_\mu \right)}{^a} \tensor{\left( \varepsilon_\nu \right)}{^b} \tensor{\left( \varepsilon_\sigma \right)}{^c} \tensor{\nabla}{_{[a}} \tensor{F}{_{bc]}}\\
			={}& \tensor{\left( \varepsilon_\mu \right)}{^{[a}} \tensor{\left( \varepsilon_\nu \right)}{^b} \tensor{\left( \varepsilon_\sigma \right)}{^{c]}} \left( \tensor{\left( \varepsilon^\rho \right)}{_b} \tensor{\left( \varepsilon^\lambda \right)}{_c} \tensor{\nabla}{_a} \tensor{F}{_\rho_\lambda} + 2 \tensor{F}{_\rho_\lambda} \tensor{\left( \varepsilon^\rho \right)}{_{[b}} \tensor{\left( \varepsilon^\tau \right)}{_{c]}} \tensor{\omega}{_\tau^\lambda_a} \right)\\
			={}& \tensor{\nabla}{_{[\mu}} \tensor{F}{_{\nu \sigma]}} + 2 \tensor{\omega}{_{[\sigma}^\lambda_\mu} \tensor{F}{_{\nu]}_\lambda}\\
			={}& 0,
		\end{align*}
		由正文 (8-7-3) 知
		% \begin{align*}
		% 	\tensor{\omega}{_1^1^1} &= \bar{\beta} - \alpha, & \tensor{\omega}{_1^3^1} &= \rho, & \tensor{\omega}{_1^4^1} &= - \bar{\mu}, & \tensor{\omega}{_2^3^1} &= \bar{\sigma}, & \tensor{\omega}{_2^4^1} &= - \lambda, & \tensor{\omega}{_3^3^1} &= \alpha + \bar{\beta},\\
		% 	\tensor{\omega}{_1^1^2} &= \bar{\alpha} - \beta, & \tensor{\omega}{_1^3^2} &= \sigma, & \tensor{\omega}{_1^4^2} &= - \bar{\lambda}, & \tensor{\omega}{_2^3^2} &= \bar{\rho}, & \tensor{\omega}{_2^4^2} &= - \mu, & \tensor{\omega}{_3^3^2} &= \bar{\alpha} + \beta,\\
		% 	\tensor{\omega}{_1^1^3} &= \varepsilon - \bar{\varepsilon} , & \tensor{\omega}{_1^3^3} &= - \kappa, & \tensor{\omega}{_1^4^3} &= \bar{\pi}, & \tensor{\omega}{_2^3^3} &= - \bar{\kappa}, & \tensor{\omega}{_2^4^3} &= \pi, & \tensor{\omega}{_3^3^3} &= - \left( \varepsilon + \bar{\varepsilon} \right),\\
		% 	\tensor{\omega}{_1^1^4} &= \gamma - \bar{\gamma}, & \tensor{\omega}{_1^3^4} &= - \tau, & \tensor{\omega}{_1^4^4} &= \bar{\nu}, & \tensor{\omega}{_2^3^4} &= - \bar{\tau}, & \tensor{\omega}{_2^4^4} &= \nu, & \tensor{\omega}{_3^3^4} &= - \left( \gamma + \bar{\gamma} \right),
		% \end{align*}
		\begin{align*}
			\tensor{\omega}{_1^1_1} &= - \tensor{\omega}{_2^2_1} = \bar{\alpha} - \beta,
			& \tensor{\omega}{_1^3_1} &= \tensor{\omega}{_4^2_1} = \sigma,
			& \tensor{\omega}{_1^4_1} &= \tensor{\omega}{_3^2_1} = - \bar{\lambda},
			\\
			\tensor{\omega}{_1^1_2} &= - \tensor{\omega}{_2^2_2} = \bar{\beta} - \alpha,
			& \tensor{\omega}{_1^3_2} &= \tensor{\omega}{_4^2_2} = \rho,
			& \tensor{\omega}{_1^4_2} &= \tensor{\omega}{_3^2_2} = - \bar{\mu},
			\\
			\tensor{\omega}{_1^1_3} &= - \tensor{\omega}{_2^2_3} = \bar{\gamma} - \gamma,
			& \tensor{\omega}{_1^3_3} &= \tensor{\omega}{_4^2_3} = \tau,
			& \tensor{\omega}{_1^4_3} &= \tensor{\omega}{_3^2_3} = - \bar{\nu},
			\\
			\tensor{\omega}{_1^1_4} &= - \tensor{\omega}{_2^2_4} = \bar{\varepsilon} - \varepsilon,
			& \tensor{\omega}{_1^3_4} &= \tensor{\omega}{_4^2_4} = \kappa,
			& \tensor{\omega}{_1^4_4} &= \tensor{\omega}{_3^2_4} = - \bar{\pi},
			\\
			\tensor{\omega}{_2^3_1} &= \tensor{\omega}{_4^1_1} = \bar{\rho},
			& \tensor{\omega}{_2^4_1} &= \tensor{\omega}{_3^1_1} = - \mu,
			& \tensor{\omega}{_3^3_1} &= - \tensor{\omega}{_4^4_1} = \bar{\alpha} + \beta,
			\\
			\tensor{\omega}{_2^3_2} &= \tensor{\omega}{_4^1_2} = \bar{\sigma},
			& \tensor{\omega}{_2^4_2} &= \tensor{\omega}{_3^1_2} = - \lambda,
			& \tensor{\omega}{_3^3_2} &= - \tensor{\omega}{_4^4_2} = \alpha + \bar{\beta},
			\\
			\tensor{\omega}{_2^3_3} &= \tensor{\omega}{_4^1_3} = \bar{\tau},
			& \tensor{\omega}{_2^4_3} &= \tensor{\omega}{_3^1_3} = - \nu,
			& \tensor{\omega}{_3^3_3} &= - \tensor{\omega}{_4^4_3} = \gamma + \bar{\gamma},
			\\
			\tensor{\omega}{_2^3_4} &= \tensor{\omega}{_4^1_4} = \bar{\kappa},
			& \tensor{\omega}{_2^4_4} &= \tensor{\omega}{_3^1_4} = - \pi,
			& \tensor{\omega}{_3^3_4} &= - \tensor{\omega}{_4^4_4} = \varepsilon + \bar{\varepsilon},
		\end{align*}
		% \begin{align*}
		% 	\tensor{\omega}{_1^1^1} &= - \tensor{\omega}{_2^2^1} = \bar{\beta} - \alpha, & \tensor{\omega}{_1^3^1} &= \tensor{\omega}{_4^2^1} = \rho, & \tensor{\omega}{_1^4^1} &= \tensor{\omega}{_3^2^1} = - \bar{\mu},\\
		% 	\tensor{\omega}{_1^1^2} &= - \tensor{\omega}{_2^2^2} = \bar{\alpha} - \beta, & \tensor{\omega}{_1^3^2} &= \tensor{\omega}{_4^2^2} = \sigma, & \tensor{\omega}{_1^4^2} &= \tensor{\omega}{_3^2^2} = - \bar{\lambda},\\
		% 	\tensor{\omega}{_1^1^3} &= - \tensor{\omega}{_2^2^3} = \varepsilon - \bar{\varepsilon} , & \tensor{\omega}{_1^3^3} &= \tensor{\omega}{_4^2^3} = - \kappa, & \tensor{\omega}{_1^4^3} &= \tensor{\omega}{_3^2^3} = \bar{\pi},\\
		% 	\tensor{\omega}{_1^1^4} &= - \tensor{\omega}{_2^2^4} = \gamma - \bar{\gamma}, & \tensor{\omega}{_1^3^4} &= \tensor{\omega}{_4^2^4} = - \tau, & \tensor{\omega}{_1^4^4} &= \tensor{\omega}{_3^2^4} = \bar{\nu},\\
		% 	\tensor{\omega}{_2^3^1} &= \tensor{\omega}{_4^1^1} = \bar{\sigma}, & \tensor{\omega}{_2^4^1} &= \tensor{\omega}{_3^1^1} = - \lambda, & \tensor{\omega}{_3^3^1} &= - \tensor{\omega}{_4^4^1} = \alpha + \bar{\beta},\\
		% 	\tensor{\omega}{_2^3^2} &= \tensor{\omega}{_4^1^2} = \bar{\rho}, & \tensor{\omega}{_2^4^2} &= \tensor{\omega}{_3^1^2} = - \mu, & \tensor{\omega}{_3^3^2} &= - \tensor{\omega}{_4^4^2} = \bar{\alpha} + \beta,\\
		% 	\tensor{\omega}{_2^3^3} &= \tensor{\omega}{_4^1^3} = - \bar{\kappa}, & \tensor{\omega}{_2^4^3} &= \tensor{\omega}{_3^1^3} = \pi, & \tensor{\omega}{_3^3^3} &= - \tensor{\omega}{_4^4^3} = - \left( \varepsilon + \bar{\varepsilon} \right),\\
		% 	\tensor{\omega}{_2^3^4} &= \tensor{\omega}{_4^1^4} = - \bar{\tau}, & \tensor{\omega}{_2^4^4} &= \tensor{\omega}{_3^1^4} = \nu, & \tensor{\omega}{_3^3^4} &= - \tensor{\omega}{_4^4^4} = - \left( \gamma + \bar{\gamma} \right),
		% \end{align*}
		% 于是算得
		% \begin{align*}
		% 	\tensor{\omega}{_\mu^1^\mu} &= \bar{\beta} - \alpha + \pi - \bar{\tau},\\
		% 	\tensor{\omega}{_\mu^2^\mu} &= \beta - \bar{\alpha} + \bar{\pi} - \tau,\\
		% 	\tensor{\omega}{_\mu^3^\mu} &= \rho + \bar{\rho} - \varepsilon - \bar{\varepsilon},\\
		% 	\tensor{\omega}{_\mu^4^\mu} &= - \bar{\mu} - \mu + \gamma + \bar{\gamma},
		% \end{align*}
		故
		% \begin{align*}
		% 	\tensor{\left( \varepsilon_1 \right)}{^b} \tensor{\nabla}{^a} \tensor{F}{_a_b} ={}& \tensor{\left( \varepsilon^\nu \right)}{_a} \tensor{\nabla}{^a} \tensor{F}{_\nu_1} + \tensor{F}{_\nu_1} \tensor{\omega}{_\mu^\nu^\mu} + \tensor{F}{_\nu_\sigma} \tensor{\omega}{_1^\sigma^\nu}\\
		% 	={}& \tensor{m}{_a} \tensor{\nabla}{^a} \tensor{F}{_2_1} - \tensor{k}{_a} \tensor{\nabla}{^a} \tensor{F}{_3_1} - \tensor{l}{_a} \tensor{\nabla}{^a} \tensor{F}{_4_1}\\
		% 	&{} + \tensor{F}{_2_1} \left( \beta - \bar{\alpha} + \bar{\pi} - \tau \right) + \tensor{F}{_3_1} \left( \rho + \bar{\rho} - \varepsilon - \bar{\varepsilon} \right) + \tensor{F}{_4_1} \left( - \bar{\mu} - \mu + \gamma + \bar{\gamma} \right)\\
		% 	&{} + \tensor{F}{_1_3} \rho - \tensor{F}{_1_4} \bar{\mu} + \tensor{F}{_2_1} \left( \bar{\alpha} - \beta \right) + \tensor{F}{_2_3} \sigma - \tensor{F}{_2_4} \bar{\lambda}\\
		% 	&{} + \tensor{F}{_3_1} \left( \varepsilon - \bar{\varepsilon} \right) + \tensor{F}{_3_4} \bar{\pi} + \tensor{F}{_4_1} \left( \gamma - \bar{\gamma} \right) - \tensor{F}{_4_3} \tau\\
		% 	={} & \delta \left( \Phi_1 - \bar{\Phi}_1 \right) + D \bar{\Phi}_2 - \Delta \Phi_0\\
		% 	&{} + \left( \Phi_1 - \bar{\Phi}_1 \right) \left( \bar{\pi} - \tau \right) - \bar{\Phi}_2 \left( \bar{\rho} - 2 \bar{\varepsilon} \right) + \Phi_0 \left( - \mu + 2 \gamma \right)\\
		% 	&{} + \Phi_2 \sigma + \bar{\Phi}_0 \bar{\lambda} - \left( \Phi_1 + \bar{\Phi}_1 \right) \left( \bar{\pi} + \tau \right)\\
		% 	={} & \delta \left( \Phi_1 - \bar{\Phi}_1 \right) + D \bar{\Phi}_2 - \Delta \Phi_0\\
		% 	&{} -2 \Phi_1 \tau -2 \bar{\Phi}_1 \bar{\pi} - \bar{\Phi}_2 \left( \bar{\rho} - 2 \bar{\varepsilon} \right) + \Phi_0 \left( - \mu + 2 \gamma \right) + \Phi_2 \sigma + \bar{\Phi}_0 \bar{\lambda},\\
		% 	\tensor{\left( \varepsilon_2 \right)}{^b} \tensor{\nabla}{^a} \tensor{F}{_a_b} ={}& \tensor{\left( \varepsilon^\nu \right)}{_a} \tensor{\nabla}{^a} \tensor{F}{_\nu_2} + \tensor{F}{_\nu_2} \tensor{\omega}{_\mu^\nu^\mu} + \tensor{F}{_\nu_\sigma} \tensor{\omega}{_2^\sigma^\nu}\\
		% 	={}& \tensor{\bar{m}}{_a} \tensor{\nabla}{^a} \tensor{F}{_1_2} - \tensor{k}{_a} \tensor{\nabla}{^a} \tensor{F}{_3_2} - \tensor{l}{_a} \tensor{\nabla}{^a} \tensor{F}{_4_2}\\
		% 	&{}+ \tensor{F}{_1_2} \left( \bar{\beta} - \alpha + \pi - \bar{\tau} \right) + \tensor{F}{_3_2} \left( \rho + \bar{\rho} - \varepsilon - \bar{\varepsilon} \right) + \tensor{F}{_4_2} \left( - \bar{\mu} - \mu + \gamma + \bar{\gamma} \right)\\
		% 	&{} + \tensor{F}{_1_2} \left( \alpha - \bar{\beta} \right) + \tensor{F}{_1_3} \bar{\sigma} - \tensor{F}{_1_4} \lambda + \tensor{F}{_2_3} \bar{\rho} - \tensor{F}{_2_4} \mu\\
		% 	&{} + \tensor{F}{_3_2} \left( \bar{\varepsilon} - \varepsilon \right) + \tensor{F}{_3_4} \pi + \tensor{F}{_4_2} \left( \bar{\gamma} - \gamma \right) - \tensor{F}{_4_3} \bar{\tau}\\
		% 	={} & \bar{\delta} \left( \bar{\Phi}_1 - \Phi_1 \right) + D \Phi_2 - \Delta \bar{\Phi}_0\\
		% 	&{} - \left( \Phi_1 - \bar{\Phi}_1 \right) \left( \pi - \bar{\tau} \right) - \Phi_2 \left( \rho - 2 \varepsilon \right) + \bar{\Phi}_0 \left( - \mu - \bar{\mu} + 2 \bar{\gamma} \right)\\
		% 	&{} + \bar{\Phi}_2 \bar{\sigma} - \left( \Phi_1 + \bar{\Phi}_1 \right) \bar{\tau}\\
		% 	={} & \bar{\delta} \left( \bar{\Phi}_1 - \Phi_1 \right) + D \Phi_2 - \Delta \bar{\Phi}_0\\
		% 	&{} + \bar{\Phi}_0 \lambda + \bar{\Phi}_0 \left( - \bar{\mu} + 2 \bar{\gamma} \right) - 2 \Phi_1 \pi -2 \bar{\Phi}_1 \bar{\tau} - \Phi_2 \left( \rho - 2 \varepsilon \right) + \bar{\Phi}_2 \bar{\sigma},\\
		% 	\tensor{\left( \varepsilon_3 \right)}{^b} \tensor{\nabla}{^a} \tensor{F}{_a_b} ={}& \tensor{\left( \varepsilon^\nu \right)}{_a} \tensor{\nabla}{^a} \tensor{F}{_\nu_3} + \tensor{F}{_\nu_3} \tensor{\omega}{_\mu^\nu^\mu} + \tensor{F}{_\nu_\sigma} \tensor{\omega}{_3^\sigma^\nu}
		% \end{align*}
		\begin{align*}
			\tensor{\left( \varepsilon_1 \right)}{^b} \tensor{\nabla}{^a} \tensor{F}{_a_b} ={} & \tensor{g}{^\nu^\sigma} \tensor{\nabla}{_\sigma} \tensor{F}{_\nu_1} + \tensor{g}{^\sigma^\rho} \tensor{F}{_\nu_1} \tensor{\omega}{_\sigma^\nu_\rho} + \tensor{g}{^\sigma^\rho} \tensor{F}{_\sigma_\nu} \tensor{\omega}{_1^\nu_\rho}\\
			={} & \tensor{\nabla}{_1} \tensor{F}{_2_1} - \Nabla{4} \tensor{F}{_3_1} - \Nabla{3} \tensor{F}{_4_1} + \tensor{F}{_2_1} \left( \tensor{\omega}{_2^2_1} - \tensor{\omega}{_3^2_4} - \tensor{\omega}{_4^2_3} \right)\\
			&{} + \tensor{F}{_3_1} \left( \tensor{\omega}{_1^3_2} + \tensor{\omega}{_2^3_1} - \tensor{\omega}{_3^3_4} \right) + \tensor{F}{_4_1} \left( \tensor{\omega}{_1^4_2} + \tensor{\omega}{_2^4_1} - \tensor{\omega}{_4^4_3} \right)\\
			&{} + \tensor{F}{_1_3} \tensor{\omega}{_1^3_2} + \tensor{F}{_1_4} \tensor{\omega}{_1^4_2} + \tensor{F}{_2_1} \tensor{\omega}{_1^1_1} + \tensor{F}{_2_3} \tensor{\omega}{_1^3_1} + \tensor{F}{_2_4} \tensor{\omega}{_1^4_1}\\
			&{} - \tensor{F}{_3_1} \tensor{\omega}{_1^1_4} - \tensor{F}{_3_4} \tensor{\omega}{_1^4_4} - \tensor{F}{_4_1} \tensor{\omega}{_1^1_4} - \tensor{F}{_4_3} \tensor{\omega}{_1^3_4}\\
			={}&
		\end{align*}
		我先放弃了……
	\end{jie}

	\item 试证式 (8-8-7) 和 (8-8-10)。

	\begin{zm}
		由正文 (7-2-6)
		\begin{equation*}
			\tensor{T}{_a_b} = \frac{1}{4\pi} \left( \tensor{F}{_a_c} \tensor{F}{_b^c} - \frac{1}{4} \tensor{g}{_a_d} \tensor{F}{_c_d} \tensor{F}{^c^d} \right) \tag{7-2-6}
		\end{equation*}
		知
		\begin{equation*}
			\tensor{T}{_a_b} = \frac{1}{4\pi} \left( \tensor{g}{^c^d} \tensor{F}{_a_c} \tensor{F}{_b_d} - \frac{1}{4} \tensor{g}{_a_b} \tensor{F}{_c_d} \tensor{F}{^c^d} \right),
		\end{equation*}
		而
		\begin{align*}
			\tensor{F}{_a_b} \tensor{F}{^a^b} &= \tensor{g}{^\mu^\sigma} \tensor{g}{^\nu^\rho} \tensor{F}{_\mu_\nu} \tensor{F}{_\sigma_\rho}\\
			&= 2 \tensor{g}{^\nu^\rho} \tensor{F}{_1_\nu} \tensor{F}{_2_\rho} - 2 \tensor{g}{^\nu^\rho} \tensor{F}{_3_\nu} \tensor{F}{_4_\rho}\\
			&= 2 \tensor{F}{_1_2} \tensor{F}{_2_1} - 2 \tensor{F}{_1_3} \tensor{F}{_2_4} - 2 \tensor{F}{_1_4} \tensor{F}{_2_3} - 2 \tensor{F}{_3_1} \tensor{F}{_4_2} - 2 \tensor{F}{_3_2} \tensor{F}{_4_1} + 2 \tensor{F}{_3_4} \tensor{F}{_4_3}\\
			&= 2 \left( -\tensor{F}{_1_2}^2 - \tensor{F}{_3_4}^2 + 2 \tensor{F}{_1_3} \tensor{F}{_4_2} + 2 \tensor{F}{_4_1} \tensor{F}{_2_3} \right)\\
			&= -2 \left( \Phi_1 - \bar{\Phi}_1 \right)^2 - 2 \left( \Phi_1 + \bar{\Phi}_1 \right)^2 + 4 \bar{\Phi}_2 \bar{\Phi}_0 + 4 \Phi_0 \Phi_2\\
			&= 4 \left( - \Phi_1^2 - \bar{\Phi}_1^2 + \Phi_0 \Phi_2 + \bar{\Phi}_0 \bar{\Phi}_2 \right),
		\end{align*}
		故
		\begin{align*}
			4 \pi \tensor{T}{_1_1} &= - 2 \tensor{F}{_1_3} \tensor{F}{_1_4}\\
			&= 2 \Phi_0 \bar{\Phi}_2,\\
			4 \pi \tensor{T}{_1_2} &= \tensor{F}{_1_2} \tensor{F}{_2_1} - \tensor{F}{_1_3} \tensor{F}{_2_4} - \tensor{F}{_1_4} \tensor{F}{_2_3} - \frac{1}{4} \tensor{F}{_c_d} \tensor{F}{^c^d}\\
			&= - \left( \Phi_1 - \bar{\Phi}_1 \right)^2 + \bar{\Phi}_2 \bar{\Phi}_0 + \Phi_0 \Phi_2 + \Phi_1^2 + \bar{\Phi}_1^2 - \Phi_0 \Phi_2 - \bar{\Phi}_0 \bar{\Phi}_2\\
			&= 2 \Phi_1 \bar{\Phi}_1,\\
			4 \pi \tensor{T}{_1_3} &= \tensor{F}{_1_2} \tensor{F}{_3_1} - \tensor{F}{_1_3} \tensor{F}{_3_4}\\
			&= \tensor{F}{_1_3} \left( \tensor{F}{_4_3} + \tensor{F}{_2_1} \right)\\
			&= 2 \Phi_1 \bar{\Phi}_2,\\
			4 \pi \tensor{T}{_1_4} &= \tensor{F}{_1_2} \tensor{F}{_4_1} - \tensor{F}{_1_4} \tensor{F}{_4_3}\\
			&= \tensor{F}{_4_1} \left( \tensor{F}{_4_3} - \tensor{F}{_2_1} \right)\\
			&= 2 \Phi_0 \bar{\Phi}_1,\\
			4 \pi \tensor{T}{_2_2} &= - 2 \tensor{F}{_2_3} \tensor{F}{_2_4}\\
			&= 2 \bar{\Phi}_0 \Phi_2,\\
			4 \pi \tensor{T}{_2_3} &= \tensor{F}{_2_1} \tensor{F}{_3_2} - \tensor{F}{_2_3} \tensor{F}{_3_4}\\
			&= \tensor{F}{_2_3} \left( \tensor{F}{_4_3} - \tensor{F}{_2_1} \right)\\
			&= 2 \bar{\Phi}_1 \Phi_2,\\
			4 \pi \tensor{T}{_2_4} &= \tensor{F}{_2_1} \tensor{F}{_4_2} - \tensor{F}{_2_4} \tensor{F}{_4_3}\\
			&= \tensor{F}{_4_2} \left( \tensor{F}{_4_3} + \tensor{F}{_2_1} \right)\\
			&= 2 \bar{\Phi}_0 \Phi_1,\\
			4 \pi \tensor{T}{_3_3} &= 2 \tensor{F}{_3_1} \tensor{F}{_3_2}\\
			&= 2 \Phi_2 \bar{\Phi}_2,\\
			4 \pi \tensor{T}{_3_4} &= \tensor{F}{_3_1} \tensor{F}{_4_2} + \tensor{F}{_3_2} \tensor{F}{_4_1} - \tensor{F}{_3_4} \tensor{F}{_4_3} + \frac{1}{4} \tensor{F}{_c_d} \tensor{F}{^c^d}\\
			&= - \bar{\Phi}_2 \bar{\Phi}_0 - \Phi_2 \Phi_0 + \left( \Phi_1 + \bar{\Phi}_1 \right)^2 - \Phi_1^2 - \bar{\Phi}_1^2 + \Phi_0 \Phi_2 + \bar{\Phi}_0 \bar{\Phi}_2\\
			&= 2 \Phi_1 \bar{\Phi}_1,\\
			4 \pi \tensor{T}{_4_4} &= 2 \tensor{F}{_4_1} \tensor{F}{_4_2}\\
			&= 2 \Phi_0 \bar{\Phi}_0,
		\end{align*}
		(8-8-7) 证毕。

		按~\eqref{eq-SS},
		\begin{align*}
			\tensor{\Sigma}{_a_b} \tensor{\Sigma}{^a^b} &= 2 \left( \tensor{F}{_a_b} \tensor{F}{^a^b} + \ii \tensor{F}{_a_b} \tensor[^*]{F}{^a^b} \right),
		\end{align*}
		而
		\begin{align*}
			\tensor{F}{_a_b} \tensor[^*]{F}{^a^b} &= \frac{1}{2} \tensor{\varepsilon}{_\mu_\nu_\sigma_\rho} \tensor{F}{^\mu^\nu} \tensor{F}{^\sigma^\rho}\\
			&= 4 \ii \tensor{F}{_1_2} \tensor{F}{_3_4} - 4 \ii \tensor{F}{_1_3} \tensor{F}{_2_4} + 4 \ii \tensor{F}{_1_4} \tensor{F}{_2_3}\\
			&= 4 \ii \Phi_1^2 - 4 \ii \bar{\Phi}_1^2 - 4 \ii \Phi_0 \Phi_2 + 4 \ii \bar{\Phi}_0 \bar{\Phi}_2,
		\end{align*}
		故
		\begin{align*}
			\tensor{\Sigma}{_a_b} \tensor{\Sigma}{^a^b} &= -8 \Phi_1^2 - 8 \bar{\Phi}_1^2 + 8 \Phi_0 \Phi_2 + 8 \bar{\Phi}_0 \bar{\Phi}_2 - 8 \Phi_1^2 + 8 \bar{\Phi}_1^2 + 8 \Phi_0 \Phi_2 - 8 \bar{\Phi}_0 \bar{\Phi}_2\\
			&= 16 \left( \Phi_0 \Phi_2 - \Phi_1^2 \right),
		\end{align*}
		(8-8-10) 证毕。
	\end{zm}
\end{xiti}
