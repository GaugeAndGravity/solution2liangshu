% !TeX root = ../document.tex

\chapter{拓扑空间简介}

\begin{xiti}
	\item 试证$\displaystyle A-B=A\cap (X-B)$,$\displaystyle \forall A,B\subset X$。
	
	\begin{zm}
			$  x\in A-B \iff x\in A \wedge x\notin B\iff x\in A\cap (X-B) $。 
	\end{zm}
	
	\item 试证$X-(B-A)=(X-B)\cup A$,$ \forall A,B\subset X$。
	
	\begin{zm}
		$ x\in X-(B-A) \iff x\notin B-A \iff x\notin B \vee x\in A \iff x\in (X-B)\cup A $。
	\end{zm}
	
	\item 用“对”或“错”在下表中填空:
	\begin{table}[htb]
		\begin{tabularx}{\textwidth}{YYY}
			\toprule
			$f\colon \mathbb{R}\rightarrow \mathbb{R} $&是一一的&是到上的\\
			\midrule
			$f(x)=x^3 $& & \\
			$f(x)=x^2$& &\\
			$f(x)=\e{x}$&&\\
			$f(x)=\cos x$&&\\
			$f(x)=5,\forall x\in \mathbb{R}$&&\\
			\bottomrule
		\end{tabularx}
	\end{table}
	
	\begin{jie}
		如下表:
		\begin{table}[htb]
			\begin{tabularx}{\textwidth}{YYY}
				\toprule
				$f\colon \mathbb{R}\rightarrow \mathbb{R} $&是一一的&是到上的\\
				\midrule
				$f(x)=x^3$&\textit{对} &\textit{对}\\
				$f(x)=x^2 $&\textit{错} &\textit{错}\\
				$f(x)=\e{x}$&\textit{对}&\textit{错}\\
				$f(x)=\cos x$&\textit{错}&\textit{错}\\
				$f(x)=5,\forall x\in \mathbb{R}$&\textit{错}&\textit{错}\\
				\bottomrule
			\end{tabularx}
		\end{table}
	\end{jie}
	
	\item 判断下列说法的是非并简述理由:
	\begin{enumerate}
		\item[(a)] 正切函数是由$\mathbb{R}$到$\mathbb{R}$的映射;
		\item[(b)] 对数函数是由$\mathbb{R}$到$\mathbb{R}$的映射;
		\item[(c)] $\left(a,b\right]\subset \mathbb{R}$用$\TT_u$衡量是开集;
		\item[(d)] $\left[a,b\right]\subset \mathbb{R}$用$\TT_u$衡量是闭集。
	\end{enumerate}
	
	\begin{jie}
		\begin{enumerate}
			\item[(a)] 错,定义域不是$\mathbb{R}$;
			\item[(b)] 错,定义域不是$\mathbb{R}$;
			\item[(c)] 错,任意包含于$\left(a,b\right]$的开区间都不会含有$b$,故$\left(a,b\right]$不能写为开区间之并;
			\item[(d)] 对,其补集$(-\infty,a)\cup (b,\infty)$是开集。
		\end{enumerate}
	\end{jie}

	
	\item 举一反例证明命题“$\left(\mathbb{R},\TT_u\right)$的无限个开子集之交为开“不真。
	
	\begin{zm}
		记$\displaystyle O_n=\left(-\frac{1}{n},\frac{1}{n}\right)$,则$\displaystyle\bigcap_{n=1}^{\infty} O_n=\{0\}$为闭集。
	\end{zm}
	
	\item 试证\S 1.2例5中定义的诱导拓扑满足定义1的3个条件。
	
	\begin{zm}
		拓扑空间$(X,\TT)$的子集$A$上的诱导拓扑按照定义为
		$$ \mathscr{S}:=\left\{ V\subset A\mid \exists O\in \TT, \qq{s.t.}V=A\cap O \right\}, $$
		\begin{enumerate}
			\item[(a)] $A,\varnothing \in \mathscr{S}$:取$O=X$即知$A\in\mathscr{S}$,取$O=\varnothing$即知$A\in \mathscr{S}$;
			\item[(b)] 有限交:设$V_i=A\cap O_i\in \mathscr{S}$,其中$O_i\in\TT $,$i=1,2,\cdots,n$。则
			\begin{equation*}
			\bigcap_{i=1}^n V_i =A\cap \left(\bigcap_{i=1}^n O_i \right)\in \mathscr{S};
			\end{equation*}
			\item[(c)] 无限并:设$V_\alpha=A\cap O_\alpha \in \mathscr{S}$,其中$O_\alpha\in\TT $,$\alpha\in \text{某个指标集} I$。则
			\begin{equation*}
			\bigcup_{\alpha\in I} V_\alpha =A\cap \left(\bigcup_{\alpha\in I} O_\alpha \right)\in \mathscr{S}.
			\end{equation*}
		\end{enumerate}
	\end{zm}

	\item 举例说明$(\mathbb{R}^3,\TT_u)$中存在不开不闭的子集。
	
	\begin{jie}
		令$A=\left(0,1 \right]^3 $,任何包含于$A$的开球$B_{r}(x_0,y_0,z_0) $的$z$坐标的范围为开区间$(z_0-r,z_0+r)\in (0,1]$,故$(x,y,1)$不能属于此开球,于是$A$不能由一族开球之并得到,故$A$不是开集。其补集中$(x,y,0)$不能属于开球,故补集不是开集,故$A$不是闭集。
	\end{jie}
	
	\item \hypertarget{1.8}{} 常值映射$f\colon \left(X,\TT\right)\rightarrow\left(Y,\mathscr{S}\right) $是否连续?为什么?
	
	\begin{jie}
		连续。证明如下:设$f[X]=\{y\}\subset Y$,$\forall O\in \mathscr{S} $,若$y\in O$,则$f^{-1}[O]=X\in \TT$;若$y\notin O$,则$f^{-1}[O]=\varnothing \in\TT $。故$f$连续。
	\end{jie}

	\item 设$\TT$为集$X$上的离散拓扑,$\mathscr{S}$为集$Y$上的凝聚拓扑,
	\begin{enumerate}
		\item[(a)] 找出从$\left(X,\TT\right)$到$\left(Y,\mathscr{S}\right)$的全部连续映射;
		\item[(b)] 找出从$\left(Y,\mathscr{S}\right)$到$\left(X,\mathscr{T}\right)$的全部连续映射。
	\end{enumerate}
	
	\begin{jie}
		\begin{enumerate}
			\item[(a)] 设$f\colon X\rightarrow Y$,则由于$\mathscr{S}=\{Y,\varnothing\}$,$f$连续当且仅当$f^{-1}[Y]=X\in \TT \wedge f^{-1}[\varnothing]=\varnothing\in \TT$,可是这是必然满足的,于是所有映射$f\colon \left(X,\TT\right)\rightarrow\left(Y,\mathscr{S}\right) $均连续。
			\item[(b)] 设$g\colon Y\rightarrow X$,则由于$\mathscr{T}=2^X$,$g$连续当且仅当$\forall O\subset X $,$g^{-1}[O]=X \vee g^{-1}[O]=\varnothing $。假设存在$x,y\in g[Y]$,$x\neq y$,则取$O={x} $,有$g^{-1}[O]=g^[-1](x)\notin \mathscr{S} $,故$g$不是连续的。于是连续映射$g$的像只能有一个,即为常值映射。又~\hyperlink{1.8}{8}~中已证明常值映射为连续,故$g\colon \left(Y,\mathscr{S}\right)\rightarrow\left(X,\TT\right) $连续当且仅当其为常值映射。
		\end{enumerate}
	\end{jie}
	
	\item 试证明定义3a与3b的等价性。
	
	\begin{zm}
		\begin{enumerate}
			\item[(1)] 3a推导3b。设$f\colon \left(X,\TT \right)\rightarrow\left(Y,\mathscr{S}\right) $连续,按照定义3a即满足$\forall O\in\mathscr{S} $,$f^{-1}[O]\in\TT $。则$\forall x\in X$,任取$G^\prime\in\mathscr{S} $使得$f(x)\in G^\prime $,则只需取$G=f^{-1}[G^\prime] $,即有$G\in\TT $并且$f[G]=G^\prime \subset G^\prime $,于是按照定义3b,$f$也连续。
			\item[(2)] 3b推导3a。设$f\colon \left(X,\TT \right)\rightarrow\left(Y,\mathscr{S}\right) $连续,按照定义3b即满足$\forall x\in X $,$\forall G^\prime\in \mathscr{S} $且$f(x)\in G^\prime $,$\exists G\in \TT $使得$f[G]\subset G^\prime $。于是任取$O\in \mathscr{S} $,令$x$跑遍$f^{-1}[O]$,对每一个$x$存在$G_x\in \TT$使得$f[G_x]\subset O $,考虑$\displaystyle G=\bigcup_{x\in f^{-1}[O]} G_x $,显然$G\in\TT $。由于$x\in f^{-1}[O] $,$x\in G_x $因而$x\in G $,于是$f^{-1}[O]\subset G $;而$\forall x\in G $,不妨设$x\in G_{x_0}$,则由于$f[G_{x_0}]\subset O $,知$x\in f^{-1}[O] $,故又有$G\subset f^{-1}[O] $,于是$G$正是$f^{-1}[O] $,也就是$f^{-1}[O]=G\in \TT $,按照定义3a,$f$也是连续的。
			
		\end{enumerate}
	\end{zm}
	
	\item 试证任一开区间$(a,b)\subset \mathbb{R} $与$\mathbb{R} $同胚。
	
	\begin{zm}
		只需找到一个同胚映射。函数$f\colon (a,b)\rightarrow \mathbb{R} $定义为$\displaystyle f(x)=\tan(\pi\frac{x-a}{b-a}-\frac{\pi}{2}) $即满足要求。
	\end{zm}
	
	\item 设$X_1 $和$X_2$是$\mathbb{R}$的子集,$X_1\equiv(1,2)\cup(2,3) $,$X_2\equiv(1,2)\cup[2,3)$。以$\TT_1$和$\TT_2$分别代表由$\mathbb{R}$的通常拓扑在$X_1$和$X_2$上的诱导拓扑。拓扑空间$(X_1,\TT_1)$和$(X_2,\TT_2)$是否连通?
	
	\begin{jie}
		\begin{enumerate}
			\item[(1)] $(X_1,\TT_1) $不连通。考虑$O=(1,2)\subset X_1$,$O=X_1\cap (1,2)\in\TT_1 $,故$O$为开集;而$X-O=(2,3)$同样为开集,于是$O$即开又闭,故$(X_1,\TT_1)$不连通。
			\item[(2)] $(X_2,\TT_2)$连通。假设$\exists O\neq X_2 ,O\neq\varnothing$,$O\in \TT$且$X-O\in\TT_2$,任取$a\in O$,$b\in X-O$,不妨设$a<b$,于是$[a,b]\subset X_2 $,记$A=[a,b]\cap O$,$B=[a,b]\cap (X-O) $,$c=\sup A$,我们来证明$O$和$X-O$都是开集将导致$c\notin A$并且$c\notin (X-O)$,从而矛盾。
			\begin{enumerate}
				\item \hypertarget{1.12.2.a}{}若$c\in B$,由于$X-O$是开集,且由于$X_2=(1,3)\in\TT_u \implies \TT_2=\TT_u\cap 2^{X_2}$,$X-O$可以写作一系列开区间之并,于是$B=(X-O)\cap[a,b]$是一系列形如$[a,y),(x,y)$或$(x,b]$的区间之并,现在$c\neq a$,故包含$c$的区间属后两种,则一定存在$d\in B$,使$(d,c]\subset B$,
				\begin{enumerate}
					\item 若$c=b$,则$(d,b]\subset B$;
					\item 若$a<c<b$,则$(d,b]=(d,c]\cup(c,b]\subset B $,
				\end{enumerate}
			    于是$d$是$A$的上界,然而却小于上确界$c$,矛盾。
			    \item 若$c\in A$,同\hyperlink{1.12.2.a}{(a)}有$O$是开集将导致$\exists e\in A$,使得$[c,e)\subset A$,与$c$是$A$的上确界矛盾。
			\end{enumerate}
		    至此$c\in A$与$c\in B$均导致矛盾,然而$c\notin A \wedge c\notin B$又与$A$和$B$的定义矛盾,故$O$与$X-O$均为非空开集是不可能的。故${X_2,\TT_2}$连通。
		\end{enumerate}
	\end{jie}
	
	\item 任意集合$X$配以离散拓扑$\TT$所得的拓扑空间是否连通?
	
	\begin{jie}
		不连通。$\forall O\in X $,$O\in \TT \wedge X-O\in\TT \implies X\text{不连通}$。
	\end{jie}
	
	\item 设$A\subset B$,试证
	\begin{enumerate}
		\item[(a)] $\bar{A}\subset\bar{B}$;提示:$A\subset B$表明$\bar{B}$是含$A$的闭集。
		\item[(b)] $\mathrm{i}(A)\subset \mathrm{i}(B)$。
	\end{enumerate}
	
	\begin{zm}
		\begin{enumerate}
			\item[(a)] $A\subset B\subset\bar{B} $,根据闭包定义有$\bar{A}\subset\bar{B} $;
			\item[(b)] $\ii{A}\subset A\subset B $,根据内部定义有$\ii{A}\subset \ii{B} $。
		\end{enumerate}
	\end{zm}
	
	\item 试证$x\in \bar{A}\iff x $的任一邻域与$A$之交非空。对$\implies$证明的提示:设$O\in \TT$且$O\cap A=\varnothing$,先证$A\subset X-O$,再证(利用闭包定义)$\bar{A}\subset X-O$。
	
	\begin{zm}
		\begin{enumerate}
			\item[(1)] $\implies$:不妨设$O$是$x$的开邻域。假设$O\cap A=\varnothing$,于是$\forall a\in A$,$a\neq A $,于是$a\in X-O$,$A\subset X-O$,而$X-O$为闭集,于是$\bar{A}\subset X-O$,故知$x\notin\bar{A} $,矛盾;
			\item[(2)] $\impliedby$:设$\forall O\in \TT$使得$x\in O$,都有$O\cap A\neq \varnothing$。假设$x\notin \bar{A}$,根据定义,$\exists B$为闭集,$A\subset B$且$x\notin B$。于是$x\in X-B \in\TT$,于是$X-B$是$x$的一个与$A$无交的开邻域,矛盾。
		\end{enumerate}
	\end{zm}
	
	\item 试证$\mathbb{R}$不是紧致的。
	
	\begin{zm}
		记$O_i=(i-1,i+1)$,显然$\{O_i\}_{i\in\mathbb{Z}}$是$\mathbb{R}$的开覆盖。现挑出其中任意$n$个$O_{i_k}\qc k=1,2,\cdots,n$,则$\displaystyle\max_{k=1,2,\cdots,n} i_k+1$即为$\displaystyle\bigcup_{k=1,2,\cdots,n}O_{i_k} $的一个上界,故有限个元素不能覆盖$\mathbb{R}$,于是$\mathbb{R}$不是紧致的。
	\end{zm}
	
	
	
	
	
	
	
	
	
\end{xiti}