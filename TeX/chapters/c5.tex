% !TeX root = ../document.tex

\chapter{微分形式及其积分}

\begin{xiti}
	\item 在定理5-1-3中补证$\{ \tensor{\left(e^1\right)}{_a} \wedge \tensor{\left(e^2\right)}{_b}, \tensor{\left(e^2\right)}{_a} \wedge \tensor{\left(e^3\right)}{_b} , \tensor{\left(e^3\right)}{_a} \wedge \tensor{\left(e^1\right)}{_b} \}$线性独立。
	
	\begin{zm}
		设$\alpha \tensor{\left(e^1 \right)}{_a} \wedge \tensor{\left(e^2\right)}{_b} + \beta \tensor{\left(e^2\right)}{_a} \wedge \tensor{\left(e^3\right)}{_b} + \gamma \tensor{\left(e^3\right)}{_a} \wedge \tensor{\left(e^1\right)}{_b}=0 $,将$\wedge $展开,有
		\begin{align*}
		&\alpha \tensor{\left(e^1 \right)}{_a} \wedge \tensor{\left(e^2\right)}{_b} + \beta \tensor{\left(e^2\right)}{_a} \wedge \tensor{\left(e^3\right)}{_b} + \gamma \tensor{\left(e^3\right)}{_a} \wedge \tensor{\left(e^1\right)}{_b} \\
		=&\; \alpha \left( \tensor{\left(e^1\right)}{_a} \tensor{\left(e^2\right)}{_b} - \tensor{\left(e^2\right)}{_a} \tensor{\left(e^1\right)}{_b} \right) + \beta \left( \tensor{\left(e^2\right)}{_a} \tensor{\left(e^3\right)}{_b} - \tensor{\left(e^3\right)}{_a} \tensor{\left(e^2\right)}{_b} \right) \\
		&+ \gamma \left( \tensor{\left(e^3\right)}{_a} \tensor{\left(e^1\right)}{_b} - \tensor{\left(e^1\right)}{_a} \tensor{\left(e^3\right)}{_b} \right)\\
		=&\; 0,
		\end{align*}
		而$\left\{ \tensor{\left(e^i\right)}{_a} \tensor{\left(e^j\right)}{_b} \right\} $是$\F(0,2) $的一组基,故必有$\alpha=\beta=\gamma=0 $。
	\end{zm}
	
	\item 设$V$为矢量空间,$\{ \tensor{\left(e^1\right)}{_a} , \tensor{\left(e^2\right)}{_a} , \tensor{\left(e^3\right)}{_a}, \tensor{\left(e^4\right)}{_a} \}$是$V^*$的基底,写出$\tensor{\omega}{_a}\in \varLambda(1) $,$ \tensor{\omega}{_a_b_c} \in \varLambda(3) $ 和 $ \tensor{\omega}{_a_b_c_d} \in \varLambda(4) $在此基底的展开式,说明展开系数(如$\tensor{\omega}{_1_2}$)的定义。
	
	\begin{jie}
		\begin{enumerate}
			\item $ \tensor{\omega}{_a} = \tensor{\omega}{_1} \tensor{\left(e^1\right)}{_a} + \tensor{\omega}{_2} \tensor{\left(e^2\right)}{_a} + \tensor{\omega}{_3} \tensor{\left(e^3\right)}{_a} + \tensor{\omega}{_4} \tensor{\left(e^4\right)}{_a} $,其中$\tensor{\omega}{_\mu}= \tensor{\omega}{_a} \tensor{\left(e_\mu\right)}{^a} $。
			\item $\tensor{\omega}{_a_b_c} = \tensor{\omega}{_{123}} \tensor{\left(e^1\right)}{_a} \wedge \tensor{\left(e^2\right)}{_b} \wedge \tensor{\left(e^3\right)}{_c} + \tensor{\omega}{_{124}} \tensor{\left(e^1\right)}{_a} \wedge \tensor{\left(e^2\right)}{_b} \wedge \tensor{\left(e^4\right)}{_c} + \tensor{\omega}{_{134}} \tensor{\left(e^1\right)}{_a} \wedge \tensor{\left(e^3\right)}{_b} \wedge \tensor{\left(e^4\right)}{_c} + \tensor{\omega}{_{234}} \tensor{\left(e^2\right)}{_a} \wedge \tensor{\left(e^3\right)}{_b} \wedge \tensor{\left(e^4\right)}{_c} $,其中$\tensor{\omega}{_{\mu\nu\sigma}} = \tensor{\omega}{_{abc}} \tensor{\left(e_\mu\right)}{^a} \tensor{\left(e_\nu\right)}{^b} \tensor{\left(e_\sigma\right)}{_c} $。
			\item $\tensor{\omega}{_{abcd}} = \tensor{\omega}{_{1234}} \tensor{\left(e^1\right)}{_a} \wedge \tensor{\left(e^2\right)}{_b} \wedge \tensor{\left(e^3\right)}{_c} \wedge \tensor{\left(e^4\right)}{_d} $,其中$\tensor{\omega}{_{1234}}= \tensor{\omega}{_{abcd}} \tensor{\left(e_1\right)}{^a} \tensor{\left(e_2\right)}{^b} \tensor{\left(e_3\right)}{^c} \tensor{\left(e_4\right)}{^d} $。
		\end{enumerate}
	\end{jie}
	
	\item 用数学归纳法证明$ \tensor{\left( \omega^1 \right)}{_{a_1}} \wedge \cdots \wedge \tensor{\left( \omega^l \right)}{_{a_l}} = l! \tensor{\left( \omega^1 \right)}{_{[a_1}} \cdots \tensor{\left( \omega^l \right)}{_{a_l]}} $,其中$ \tensor{\left( \omega^1 \right)}{_{a_1}}, \cdots , \tensor{\left( \omega^l \right)}{_{a_l}} $是任意对偶矢量。
	
	\begin{zm}
		\begin{enumerate}
			\item $l=1$时 trivial; $l=2 $时,按照定义,\[ \tensor{\left(\omega^1\right)}{_{a_1}} \wedge \tensor{\left(\omega^2\right)}{_{a_2}} = 2 \tensor{\left(\omega^1\right)}{_{[a_1}} \tensor{\left(\omega^2\right)}{_{a_2 ]}}; \]
			\item 设$l=k$时,$\tensor{\left( \omega^1 \right)}{_{a_1}} \wedge \cdots \wedge \tensor{\left( \omega^k \right)}{_{a_k}} = k! \tensor{\left( \omega^1 \right)}{_{[a_1}} \cdots \tensor{\left( \omega^k \right)}{_{a_k]}} $,则
			\begin{align*}
			\left(\tensor{\left( \omega^1 \right)}{_{a_1}} \wedge \cdots \wedge \tensor{\left( \omega^k \right)}{_{a_k}}\right) \wedge \tensor{\left(\omega^{k+1}\right)}{_{a_{k+1}}} &= \frac{\left(k+1\right)!}{k!} \left(l! \tensor{\left( \omega^1 \right)}{_{[[a_1}} \cdots \tensor{\left( \omega^k \right)}{_{a_k]}}\right) \tensor{\left(\omega^{k+1}\right)}{_{a_{k+1}]}}\\
			&= \left(k+1\right)! \tensor{\left( \omega^1 \right)}{_{[a_1}} \cdots \tensor{\left( \omega^k \right)}{_{a_k}} \tensor{\left(\omega^{k+1}\right)}{_{a_{k+1}]}}.
			\end{align*}
			综上所述,$\forall l \in \mathbb{N}^+ $,$\tensor{\left( \omega^1 \right)}{_{a_1}} \wedge \cdots \wedge \tensor{\left( \omega^l \right)}{_{a_l}} = l! \tensor{\left( \omega^1 \right)}{_{[a_1}} \cdots \tensor{\left( \omega^l \right)}{_{a_l]}}$。
		\end{enumerate}
	\end{zm}
	
	\item 试证定理5-1-4。
	
	\begin{zm}
		定理5-1-4为
		\begin{yl}{Thm}
			设$\displaystyle \tensor{\omega}{_{a_1 \cdots a_l }} = \sum_{C} \tensor{\omega}{_{\mu_1 \cdots \mu_l}} \tensor{\left(\dd{x^{\mu_1}}\right)}{_{a_1}} \wedge \cdots \wedge  \tensor{\left(\dd{x^{\mu_l}}\right)}{_{a_l}} $,则\[ \tensor{\left(\dd{\omega}\right)}{_{b a_1 \cdots a_l}} = \sum_{C} \tensor{\left(\dd{\tensor{\omega}{_{\mu_1 \cdots \mu_l}}}\right)}{_b} \wedge \tensor{\left(\dd{x^{\mu_1}}\right)}{_{a_1}} \wedge \cdots \wedge \tensor{\left(\dd{x^{\mu_l}}\right)}{_{a_l}}. \]
		\end{yl}
	    \begin{yl}{Prf}
	    	按照定义,将导数算符选为$\Partial{a} $,有
	    	\begin{align*}
	    	\tensor{\left(\dd{\omega}\right)}{_{b a_1 \cdots a_l}} &= \left(l+1\right) \Partial{[b} \tensor{\omega}{_{a_1 \cdots a_l]}}\\
	    	&= \left(l+1\right) \sum_{C} \Partial{[b} \left(\tensor{\omega}{_{\mu_1 \cdots \mu_l}} l! \tensor{\left(\dd{x^{\mu_1}}\right)}{_{[a_1}} \cdots   \tensor{\left(\dd{x^{\mu_l}}\right)}{_{a_l]]}}\right) \\
	    	&= \left(l+1\right)! \sum_{C}\left(\Partial{[b} \tensor{\omega}{_{\mu_1 \cdots \mu_l}}\right) \tensor{\left(\dd{x^{\mu_1}}\right)}{_{a_1}} \cdots \tensor{\left(\dd{x^{\mu_l}}\right)}{_{a_l]}}\\
	    	&= \left(l+1\right)! \sum_{C} \tensor{\left( \dd{\tensor{\omega}{_{\mu_1 \cdots \mu_l}}} \right)}{_{[b}}  \tensor{\left(\dd{x^{\mu_1}}\right)}{_{a_1}} \cdots \tensor{\left(\dd{x^{\mu_l}}\right)}{_{a_l]}}\\
	    	&= \sum_{C} \tensor{\left(\dd{\tensor{\omega}{_{\mu_1 \cdots \mu_l}}}\right)}{_b} \wedge \tensor{\left(\dd{x^{\mu_1}}\right)}{_{a_1}} \wedge \cdots \wedge \tensor{\left(\dd{x^{\mu_l}}\right)}{_{a_l}}.
	    	\end{align*}
	    \end{yl}
	\end{zm}
	
	\item 设$\boldsymbol{\omega} $是1形式场,$u,v$是矢量场,试证$\dd{\boldsymbol{\omega}} \left(u,v\right) = u\left(\boldsymbol{\omega}(v)\right) - v\left(\boldsymbol{\omega}(u)\right) - \boldsymbol{\omega}\left([u,v]\right) $。等式左边代表$\dd{\boldsymbol{\omega}}$对$u,v$的作用结果,即$ \tensor{\left( \dd{\boldsymbol{\omega}} \right)}{_a_b} \tensor{u}{^a} \tensor{v}{^b} $。
	
	\begin{zm}
		\begin{align*}
		\dd{\bm{\omega}} (u,v) &= \tensor{u}{^a} \tensor{v}{^b} \left( \Nabla{a} \tensor{\omega}{_b} - \Nabla{b} \tensor{\omega}{_a} \right)\\
		&= \tensor{u}{^a} \Nabla{a} \left( \tensor{v}{^b} \tensor{\omega}{_b} \right) - \tensor{u}{^a} \tensor{\omega}{_b} \Nabla{a} \tensor{v}{^b} - \tensor{v}{^b} \Nabla{a} \left( \tensor{u}{^a} \tensor{\omega}{_a} \right) + \tensor{v}{^b} \tensor{\omega}{_a} \Nabla{b} \tensor{u}{^a}\\
		&= u\left( \bm{\omega} (v) \right) - v \left( \bm{\omega} (u) \right) - \bm{\omega} \left( [u,v] \right).
		\end{align*}
	\end{zm}
	
	\item 设$\tensor{v}{^b} $和$\tensor{\omega}{_{a_1 \cdots a_l}} $分别是流形$M$上的矢量场和$l$形式场,试证
	\begin{enumerate}
		\item[(a)] \hypertarget{5.6.a}{}$\Ld{v} \tensor{\omega}{_{a_1 \cdots a_l}} = \dd\indices{_{a_1}} \left( \tensor{v}{^b} \tensor{\omega}{_{ba_2 \cdots a_l}} \right) + \tensor{\left( \dd{\omega} \right)}{_{ba_1 \cdots a_l}} \tensor{v}{^b}. $
		
		注:令$\tensor{\mu}{_{a_2 \cdots a_l}} \equiv \tensor{v}{^b} \tensor{\omega}{_{ba_2 \cdots a_l}} $,则$\dd\indices{_{a_1}} \tensor{\mu}{_{a_2 \cdots a_l}} $是指$\tensor{\left(\dd{\mu}\right)}{_{a_1 a_2 \cdots a_l }} $。
		\item[(b)] $\Ld{v} \dd{\boldsymbol{\omega}} = \dd{\Ld{v}\boldsymbol{\omega}} $(这本身就是一个很有用的命题)。
	\end{enumerate}
    
    提示:
    \begin{enumerate}
    	\item[(1)] 证(a)时可先证$l=2$时的特例,找到感觉后不难推广至一般情况。
    	\item[(2)] 利用(a)的结果将使(b)的证明变得十分简单。
    \end{enumerate}

    \begin{zm}
    	\begin{enumerate}[listparindent=2em]
    		\item[(a)] 对于$l=2 $的情况,由式(4-2-8),左边等于
    		\begin{displaymath}
    		\Ld{v} \tensor{\omega}{_{a_1 a_2}} = \tensor{v}{^b} \Nabla{b} \tensor{\omega}{_{a_1 a_2}} + \tensor{\omega}{_{a_1 b}} \Nabla{a_2} \tensor{v}{^b} + \tensor{\omega}{_{b a_2}} \Nabla{a_1} \tensor{v}{^b}
    		\end{displaymath}
    		而右边第一项展开为
    		\begin{align*}
    		\dd\indices{_{a_1}} \left( \tensor{v}{^b} \tensor{\omega}{_{b a_2}} \right) &= 2 \Nabla{[a_1} \left( \tensor{v}{^b} \tensor{\omega}{_{|b|a_2]}} \right)\\
    		&= \Nabla{a_1} \left( \tensor{v}{^b} \tensor{\omega}{_{b a_2}} \right) - \Nabla{a_2} \left( \tensor{v}{^b} \tensor{\omega}{_{b a_1}} \right)\\
    		&= {\color{red} \tensor{v}{^b} \Nabla{a_1} \tensor{\omega}{_{b a_2}} } + \tensor{\omega}{_{b a_2}} \Nabla{a_1} \tensor{v}{^b} - {\color{blue} \tensor{v}{^b} \Nabla{a_2} \tensor{\omega}{_{b a_1}}} - \tensor{\omega}{_{b a_1}} \Nabla{a_2} \tensor{v}{^b},
    		\end{align*}
    		右边第二项为
    		\begin{align*}
    		\tensor{\left(\dd{\omega}\right)}{_{b a_1 a_2}} \tensor{v}{^b} &= 3 \tensor{v}{^b} \Nabla{[b} \tensor{\omega}{_{a_1 a_2]}}\\
    		&= \frac{1}{2} \tensor{v}{^b} \left( \Nabla{b} \tensor{\omega}{_{a_1 a_2}} + \Nabla{a_1} \tensor{\omega}{_{a_2 b}} + \Nabla{a_2} \tensor{\omega}{_{b a_1}} - \Nabla{b} \tensor{\omega}{_{a_2 a_1}} - \Nabla{a_1} \tensor{\omega}{_{b a_2}} - \Nabla{a_2} \tensor{\omega}{_{a_1 b}} \right)\\
    		&= \tensor{v}{^b} \left( \Nabla{b} \tensor{\omega}{_{a_1 a_2}} + {\color{red} \Nabla{a_1} \tensor{\omega}{_{a_2 b}}} + {\color{blue} \Nabla{a_2} \tensor{\omega}{_{b a_1}} } \right)
    		\end{align*}
    		可以看到,红色项和蓝色项分别消去,余下的项与左边相等。
    		
    		对于一般情况,左边为
    		\begin{displaymath}
    		\Ld{v} \tensor{\omega}{_{a_1 \cdots a_l}} = \tensor{v}{^b} \Nabla{b} \tensor{\omega}{_{a_1 \cdots a_l}} + \sum_{i} \tensor{\omega}{_{a_1 \cdots b \cdots a_l}} \Nabla{a_i} \tensor{v}{^b}
    		\end{displaymath}
    		
    		右边第一项为
    		\begin{align*}
    		\dd\indices{_{a_1}} \left( \tensor{v}{^b} \tensor{\omega}{_{b a_2 \cdots a_l }} \right) &= l \Nabla{[a_1} \left( \tensor{v}{^b} \tensor{\omega}{_{|b|a_2 \cdots a_l]}} \right)\\
    		&= \frac{1}{(l-1)!} \sum_{\pi} \delta_\pi \Nabla{a_{\pi_1}} \left( \tensor{v}{^b} \tensor{\omega}{_{b a_{\pi_2} \cdots a_{\pi_l} }} \right)\\ \displaybreak[1]
    		&= \frac{1}{(l-1)!} \sum_{i} \sum_{\sigma} (-1)^{i-1} \delta_\sigma \Nabla{a_{i}} \left( \tensor{v}{^b} \tensor{\omega}{_{b a_{\sigma_1} \cdots a_{\sigma_l} }} \right)\\ \displaybreak[1]
    		&= \sum_{i} (-1)^{i-1} \Nabla{a_i} \left( \tensor{v}{^b} \tensor{\omega}{_{b[a_1\cdots a_{i-1} a_{i+1} \cdots a_l]}} \right)\\ \displaybreak[1]
    		&= \sum_{i} (-1)^{i-1} \Nabla{a_i} \left( \tensor{v}{^b} \tensor{\omega}{_{[b a_1\cdots a_{i-1} a_{i+1} \cdots a_l]}} \right)\\
    		&= \sum_{i} \Nabla{a_i} \left( \tensor{v}{^b} \tensor{\omega}{_{[a_1\cdots a_{i-1} b a_{i+1} \cdots a_l]}} \right)\\
    		&= \sum_{i} \Nabla{a_i} \left( \tensor{v}{^b} \tensor{\omega}{_{a_1\cdots b \cdots a_l}} \right)\\
    		&= { \color{blue} \tensor{v}{^b} \sum_{i} \Nabla{a_i} \tensor{\omega}{_{a_1\cdots b \cdots a_l}} }  + \tensor{\omega}{_{a_1\cdots b \cdots a_l}} \sum_{i} \Nabla{a_i} \tensor{v}{^b}
    		\end{align*}
    		其中$\pi$是$1,2,\cdots ,l$的排列,$\sigma $是$1,\cdots i-1,i+1,\cdots ,l $的排列。而右边第二项为
    		\begin{align*}
    		\displaybreak[1] \tensor{\left(\dd{\omega}\right)}{_{b a_1 \cdots a_l }} \tensor{v}{^b} &= (l+1) \tensor{v}{^b} \Nabla{[b} \tensor{\omega}{_{a_1 \cdots a_l ]}}\\ \displaybreak[1]
    		&= \frac{1}{l!} \tensor{v}{^b} \sum_{\pi} \delta_\pi \Nabla{\pi_1} \tensor{\omega}{_{\pi_2 \cdots \pi_{l+1}}}\\ \displaybreak[1]
    		&= \frac{1}{l!} \tensor{v}{^b} \sum_{\sigma} \delta_\sigma \Nabla{b} \tensor{\omega}{_{a_{\sigma_1} \cdots a_{\sigma_l}}} + \frac{1}{l!} \tensor{v}{^b}  \sum_{i} \sum_{\rho} -\delta_\rho \Nabla{a_i} \tensor{\omega}{_{a_{\rho_1} \cdots b \cdots a_{\rho_{l-1}}}} \\ \displaybreak[1]
    		&= \tensor{v}{^b} \Nabla{b} \tensor{\omega}{_{[a_1 \cdots a_l]}} - \tensor{v}{^b} \sum_i \Nabla{a_i} \tensor{\omega}{_{[a_1 \cdots b \cdots a_l]}}\\
    		&= \tensor{v}{^b} \Nabla{b} \tensor{\omega}{_{a_1 \cdots a_l}} - { \color{blue} \tensor{v}{^b} \sum_i \Nabla{a_i} \tensor{\omega}{_{a_1 \cdots b \cdots a_l}} }
    		\end{align*}
    		其中$\pi$是$b,a_1,\cdots a_l $的任意排序,$\sigma$是$1,2,\cdots ,l $的排序,$\rho $是$1,\cdots ,i-1,i+1,\cdots ,l $的排序。可以看到,蓝色的项相消,余下的和左边相等,证毕。
    		\item[(b)] 由(a),
    		\begin{align*}
    		\Ld{v} \tensor{\left(\dd{\omega}\right)}{_{a_1 \cdots a_{l+1}}} &= \dd\indices{_{a_1}} \left( \tensor{v}{^b} \tensor{\left(\dd{\omega}\right)}{_{b a_2\cdots a_{l+1}}} \right) + \tensor{\left(\dd(\dd{\omega})\right)}{_{b a_1 \cdots a_{l+1} }} \tensor{v}{^b}\\
    		&= \dd\indices{_{a_1}} \left( \tensor{v}{^b} \tensor{\left(\dd{\omega}\right)}{_{b a_2\cdots a_{l+1}}} \right),\\
    		\tensor{\dd(\Ld{v}\omega)}{_{a_1\cdots a_{l+1}}} &= \dd\indices{_{a_1}} \left( \dd\indices{_{a_2}} \tensor{v}{^b} \tensor{\omega}{_{b a_3 \cdots a_{l+1} }} + \tensor{\left(\dd{\omega}\right)}{_{b a_2 \cdots a_{l+1}}} \tensor{v}{^b} \right)\\
    		&= \dd\indices{_{a_1}} \left( \tensor{\left(\dd{\omega}\right)}{_{b a_2 \cdots a_{l+1}}} \tensor{v}{^b} \right),
    		\end{align*}
    		故
    		\begin{displaymath}
    		\Ld{v} \left( \dd{\bm{\omega}} \right) = \dd(\Ld{v} \bm{\omega}).
    		\end{displaymath}
    	\end{enumerate}
    \end{zm}
	
	\item 设$O$是$n$维流形$M$上的坐标系$\{x^\mu \} $的坐标域(且$O$同胚于$\mathbb{R}^n $),$\tensor{\omega}{_a} $是$O$上的1形式场,试证
	\begin{displaymath}
	\pdv*{\tensor{\omega}{_\mu}}{x^\nu}= \pdv*{\tensor{\omega}{_\nu}}{x^\mu} (\mu,\nu=1,\cdots n) \text{当且仅当存在} f\colon O \rightarrow \mathbb{R} \text{使} \Nabla{a} f= \tensor{\omega}{_a}.
	\end{displaymath}
	提示:仿照 \S 5.1 推论 5-1-6 的证明。
	
	\begin{zm}
		设$\tensor{\omega}{_a} = \tensor{\omega}{_\mu} \tensor{\left(\dd{x^\mu}\right)}{_a} $,则
		\begin{align*}
		\tensor{\left(\dd{\omega}\right)}{_{ab}} &= \tensor{\left( \dd{\tensor{\omega}{_\mu}} \right)}{_a} \wedge \tensor{\left(\dd{x^\mu}\right)}{_{b}}\\
		&= \pdv{\tensor{\omega}{_\mu}}{x^\nu} \tensor{\left(\dd{x^\nu}\right)}{_a} \wedge \tensor{\left(\dd{x^\mu}\right)}{_b}\\
		&= \left( \pdv{\tensor{\omega}{_\mu}}{x^\nu} - \pdv{\tensor{\omega}{_\nu}}{x^\mu} \right) \tensor{\left(\dd{x^\nu}\right)}{_a} \tensor{\left(\dd{x^\mu}\right)}{_b}.
		\end{align*}
		\begin{enumerate}
			\item 若$\exists f\qq{s.t.} \tensor{\omega}{_a} = \Nabla{a} f=\tensor{\left(\dd{f}\right)}{_a} $,则$\dd{\bm{\omega}} = \dd(\dd{f})=0 $,于是知\[\pdv{\tensor{\omega}{_\mu}}{x^\nu} - \pdv{\tensor{\omega}{_\nu}}{x^\mu}=0. \]
			\item 若$\displaystyle \pdv{\tensor{\omega}{_\mu}}{x^\nu} - \pdv{\tensor{\omega}{_\nu}}{x^\mu}=0 $,则$\dd{\bm{\omega}}=0 $,$\bm{\omega}$是闭的,而$O$同胚于$\mathbb{R}^n$,由于上同调群是拓扑不变量,故$H^1 (O)={0} $,$O$上的闭形式必恰当,故$\exists f\colon O\rightarrow \mathbb{R}^n \qq{s.t.} \bm{\omega} = \dd{f}. $
		\end{enumerate}
	\end{zm}
	
	\item 设$ \{ x,y,z \} $和$\{r,\theta,\phi \} $分别为3维欧氏空间的笛卡尔坐标系和球坐标系,写出$ \dd{r} \wedge \dd{\theta} \wedge \dd{\phi} $用 $\dd{x} \wedge \dd{y} \wedge \dd{z} $的表达式。
	
	\begin{jie}
		球坐标与笛卡尔系的变换关系为:
		\begin{displaymath}
		\left\{
		\begin{aligned}
		x&=r\sin \theta \cos \phi,\\
		y&=r\sin \theta \sin \phi,\\
		z&=r\cos \theta.
		\end{aligned}
		\right.
		\end{displaymath}
		则
		\begin{align*}
		\dd{x} &= \pdv{x}{r} \dd{r} + \pdv{x}{\theta} \dd{\theta} + \pdv{x}{\phi} \dd{\phi}\\
		&= \sin\theta \cos\phi \dd{r} + r \cos\theta \cos\phi \dd{\theta} - r \sin\theta \sin \phi \dd{\phi}\\
		\dd{y} &= \pdv{y}{r} \dd{r} + \pdv{y}{\theta} \dd{\theta} + \pdv{y}{\phi} \dd{\phi}\\
		&= \sin \theta \sin\phi \dd{r} + r \cos\theta \sin \phi \dd{\theta} + r \sin\theta \cos\phi \dd{\phi}\\
		\dd{z} &= \pdv{z}{r} \dd{r} + \pdv{z}{\theta} \dd{\theta} + \pdv{z}{\phi} \dd{\phi}\\
		&= \cos\theta \dd{r} - r\sin\theta \dd{\theta}.
		\end{align*}
		故
		\begin{align*}
		\dd{x} \wedge \dd{y} \wedge \dd{z} =&\tmu \left( \pdv{x}{r} \pdv{y}{\theta} \pdv{z}{\phi} + \pdv{y}{r} \pdv{z}{\theta} \pdv{x}{\phi} + \pdv{z}{r} \pdv{x}{\theta} \pdv{y}{\phi} -\pdv{x}{r} \pdv{z}{\theta} \pdv{y}{\phi} \right. \\
		&\left. -\pdv{z}{r} \pdv{y}{\theta} \pdv{x}{\phi} - \pdv{y}{r} \pdv{x}{\theta} \pdv{z}{\phi} \right) \dd{r} \wedge \dd{\theta} \wedge \dd{\phi}\\
		=& \tmu \left( 0 + r^2 \sin^3 \theta \sin^2 \phi + r^2 \cos^2 \theta \sin \theta \cos^2 \phi + r^2 \sin^3 \theta \cos^2 \phi  \right. \\
		&\left. + r^2 \cos^2 \theta \sin \theta \sin^2 \phi + 0 \right) \dd{r} \wedge \dd{\theta} \wedge \dd{\phi} \\
		=&\; r^2 \sin \theta \dd{r} \wedge \dd{\theta} \wedge \dd{\phi},
		\end{align*}
		故
		\begin{align*}
		\dd{r} \wedge \dd{\theta} \wedge \dd{\phi} &= \frac{1}{r^2 \sin\theta} \dd{x} \wedge \dd{y} \wedge \dd{z}\\
		&= \frac{1}{\sqrt{\left(x^2+y^2+z^2\right) \left(x^2+y^2\right)}} \dd{x} \wedge \dd{y} \wedge \dd{z}.
		\end{align*}
	\end{jie}
	
	\item 连通流形$M$配以洛伦兹号差的度规场$\tensor{g}{_a_b} $叫\textbf{时空}(spacetime)。设$\tensor{F}{_a_b} $是任意4维时空的2形式场(第6章将看到电磁场张量$\tensor{F}{_a_b} $就是一个2形式场),试证
	\[ \frac{1}{2} \left( \tensor{F}{_a_c}\tensor{F}{_b^c} + \tensor[^*]{F}{_a_c} \tensor[^*]{F}{_b^c} \right) = \tensor{F}{_a_c} \tensor{F}{_b^c} - \frac{1}{4} \tensor{g}{_a_b} \tensor{F}{_c_d} \tensor{F}{^c^d}, \]
	其中$\tensor[^*]{F}{_a_c} \equiv \tensor{\left(\tensor[^*]{F}{}\right)}{_a_c} $,$\tensor[^*]{F}{_b^c}= \tensor{g}{^a^c} \tensor[^*]{F}{_b_a} $(此式对研究电磁场很有帮助)。
	
	\begin{zm}
		按照定义,
		\begin{displaymath}
		\tensor[^*]{F}{_a_b} = \frac{1}{2} \tensor{F}{^c^d} \tensor{\varepsilon}{_c_d_a_b}
		\end{displaymath}
		故
		\begin{align*}
		\tensor[^*]{F}{_a_c} \tensor[^*]{F}{_b^c} =& \; \tensor{g}{^c^d} \tensor[^*]{F}{_a_c} \tensor[^*]{F}{_b_d} \\
		=& \mspace{5mu} \frac{1}{4} \tensor{g}{^c^d} \tensor{F}{^e^f} \tensor{\varepsilon}{_e_f_a_c} \tensor{F}{^g^h} \tensor{\varepsilon}{_g_h_b_d}\\
		=& \mspace{5mu} \frac{1}{4} \tensor{F}{^e^f} \tensor{F}{^g^h} \tensor{\varepsilon}{_e_f_a_c}  \tensor{\varepsilon}{_g_h_b^c}\\
		=& \mspace{5mu} \frac{1}{4} \tensor{g}{_b_d} \tensor{F}{^e^f} \tensor{F}{_g_h} \tensor{\varepsilon}{^c^g^h^d} \tensor{\varepsilon}{_c_e_f_a}\\
		=& \mspace{5mu} \frac{1}{4} \tensor{g}{_b_d} \tensor{F}{^e^f} \tensor{F}{_g_h} (-6) \tensor{\delta}{^{[g}_e} \tensor{\delta}{^h_f} \tensor{\delta}{^{d]}_a}\\
		=& \mspace{1mu} - \mspace{-4mu} \frac{1}{4} \tensor{g}{_b_d} \tensor{F}{^e^f} \tensor{F}{_g_h} \left( \tensor{\delta}{^g_e} \tensor{\delta}{^h_f} \tensor{\delta}{^d_a} + \tensor{\delta}{^d_e} \tensor{\delta}{^g_f} \tensor{\delta}{^h_a} + \tensor{\delta}{^h_e} \tensor{\delta}{^d_f} \tensor{\delta}{^g_a}\right.\\
		& \left.- \tensor{\delta}{^g_e} \tensor{\delta}{^d_f} \tensor{\delta}{^h_a} - \tensor{\delta}{^h_e} \tensor{\delta}{^g_f} \tensor{\delta}{^d_a} - \tensor{\delta}{^d_e} \tensor{\delta}{^h_f} \tensor{\delta}{^g_a} \right)\\
		=& \mspace{5mu} \frac{1}{4} \left( - \tensor{g}{_a_b} \tensor{F}{^e^f} \tensor{F}{_e_f} - \tensor{F}{_b^f} \tensor{F}{_f_a} - \tensor{F}{^e_b} \tensor{F}{_a_e} \right. \\
		& + \tensor{F}{^e_b} \tensor{F}{_e_a} + \tensor{g}{_b_a} \tensor{F}{^e^f} \tensor{F}{_f_e} + \tensor{F}{_b^f} \tensor{F}{_a_f} \Big)\\
		=& \mspace{5mu} \frac{1}{4} \left( -2\tensor{g}{_a_b} \tensor{F}{^c^d} \tensor{F}{_c_d} + 4 \tensor{F}{_a_c} \tensor{F}{_b^c}  \right),
		\end{align*}
		于是
		\begin{align*}
		\frac{1}{2} \left( \tensor{F}{_a_c}\tensor{F}{_b^c} + \tensor[^*]{F}{_a_c} \tensor[^*]{F}{_b^c} \right)&= \frac{1}{2} \left( \tensor{F}{_a_c} \tensor{F}{_b^c} - \frac{1}{2} \tensor{g}{_a_b} \tensor{F}{^c^d} \tensor{F}{_c_d} + \tensor{F}{_a_c} \tensor{F}{_b^c} \right)\\
		&= \tensor{F}{_a_c} \tensor{F}{^a^c} - \frac{1}{4} \tensor{g}{_a_b} \tensor{F}{_c_d} \tensor{F}{^c^d}.
		\end{align*}
	\end{zm}
	
	\item 试证$\tensor{\hat{\varepsilon}}{_{a_1\cdots a_{n-1}}} \equiv \pm \tensor{n}{^b} \tensor{\varepsilon}{_{b a_1 \cdots a_{n-1}}} $是$\partial N $上与诱导度规场$\tensor{h}{_a_b} $相适配的体元。
	
	\begin{zm}
		需要证明的是
		\begin{displaymath}
		\tensor{\hat{\varepsilon}}{^{a_1 \cdots a_{n-1} }} \tensor{\hat{\varepsilon}}{_{ a_1 \cdots a_{n-1} }} = \left(-1\right)^{\hat{s}} \left(n-1\right)!
		\end{displaymath}
		根据定义展开:
		\begin{align*}
		\tensor{\hat{\varepsilon}}{^{a_1 \cdots a_{n-1} }} \tensor{\hat{\varepsilon}}{_{ a_1 \cdots a_{n-1} }} &= \tensor{h}{^{a_1 b_1}} \cdots \tensor{h}{^{a_{n-1} b_{n-1} }} \tensor{\hat{\varepsilon}}{_{b_1 \cdots b_{n-1} }} \tensor{\hat{\varepsilon}}{_{ a_1 \cdots a_{n-1} }}\\
		&= \tensor{h}{^{a_1 b_1}} \cdots \tensor{h}{^{a_{n-1} b_{n-1} }} \tensor{n}{^c} \tensor{\varepsilon}{_{c b_1 \cdots b_{n-1} }} \tensor{n}{^d} \tensor{\varepsilon}{_{ d a_1 \cdots a_{n-1} }}
		\end{align*}
		这里要明确一下 $\tensor{h}{^a^b}$ 的含义。本来$\tensor{h}{_a_b} \in \F_{\partial N}(0,2)$ 作为诱导度规,是 $\partial N$ 上的张量场,在每点的值是一个 $W_q $ 到 $W_q^* $ 的映射,而 $\tensor{h}{^a^b}$ 在每一点的值自然是其逆映射。然而,在上述计算中,第二行中把 $\bm{\hat{\varepsilon}}$ 展开实际上是作为 $N$ 上的 $n-1$ 形式看待,它在 $p$ 点的值是 $V_p$ 上的张量,而两个不同矢量空间上的张量显然没有缩并这种操作,所以这里的 $\tensor{h}{_a_b}$ 是选读4-4-3中的 $\tensor{\bar{h}}{_a_b} = \tensor{g}{_a_b} \mp \tensor{n}{_a} \tensor{n}{_b} \in \F_N{0,2} $,它作用在 $W_p$ 中的元素的结果与 $\tensor{h}{_a_b}$ 相同,而作用在补空间 $V_p-W_p$ 中的元素结果为零,所以当我们把 $W_p$ 视为 $V_p$ 的子空间时,就用 $\tensor{\bar{h}}{_a_b}$ 来代替 $\tensor{h}{_a_b}$ 来计算。而 $\tensor{h}{^a^b}$ 就解释为 $\tensor{g}{^a^c} \tensor{g}{^b^d} \tensor{\bar{h}}{_c_d} = \tensor{g}{^a^b} \mp \tensor{n}{^a} \tensor{n}{^b}$ ,事实上,这样理解的 $\tensor{h}{^a^b} \colon \F_N (1,0) \rightarrow \F_N(0,1) $ 和 $\tensor{\bar{h}}{_a_b} \colon \F_N (1,0) \rightarrow \F_N (0,1) $ 的复合为
		\begin{align*}
		\tensor{\bar{\delta}}{^a_b} &= \left( \tensor{g}{^a^c} \mp \tensor{n}{^a} \tensor{n}{^c} \right) \left( \tensor{g}{_c_b} \mp \tensor{n}{_c} \tensor{n}{_b} \right)\\
		&= \tensor{\delta}{^a_b} \mp \tensor{n}{^a} \tensor{n}{_b} \mp \tensor{n}{^a} \tensor{n}{_b} \pm \tensor{n}{^a} \tensor{n}{_b}\\
		&= \tensor{\delta}{^a_b} \mp \tensor{n}{^a} \tensor{n}{_b},
		\end{align*}
		可以验证上面这个张量 $\tensor{\bar{\delta}}{^a_b} \colon \F_N(1,0) \rightarrow \F_N(0,1) $ 在 $\F_{\partial N}(1,0) $ 上的限制(restriction)就是 $\F_{\partial N}(1,0) $ 上的恒等映射,而在 $\F_N(1,0) - \F_{\partial N}(1,0) $ 上的限制为零,故用 $\tensor{g}{^a^c} \tensor{g}{^b^d} \tensor{\bar{h}}{_b_d}$ 代替 $\tensor{h}{^a^b}$ 是没有问题的。
		
		于是
		\begin{align*}
		\tensor{\hat{\varepsilon}}{^{a_1 \cdots a_{n-1} }} \tensor{\hat{\varepsilon}}{_{ a_1 \cdots a_{n-1} }} &= \tensor{h}{^{a_1 b_1}} \cdots \tensor{h}{^{a_{n-1} b_{n-1} }} \tensor{n}{^c} \tensor{\varepsilon}{_{c b_1 \cdots b_{n-1} }} \tensor{n}{^d} \tensor{\varepsilon}{_{ d a_1 \cdots a_{n-1} }}\\
		&= \tensor{g}{^{a_1 c_1}} \tensor{g}{^{b_1 d_1}} \cdots \tensor{g}{^{a_{n-1} c_{n-1}}} \tensor{g}{^{b_{n-1} d_{n-1}}} \left( \tensor{g}{_{c_1 d_1}} \mp \tensor{n}{_{c_1}} \tensor{n}{_{d_{1}}} \right) \cdots\\
		&\;{\color{white} =}\; \left( \tensor{g}{_{c_{n-1} d_{n-1}}} \mp \tensor{n}{_{c_{n-1}}} \tensor{n}{_{d_{n-1}}} \right) \tensor{n}{^c} \tensor{\varepsilon}{_{c b_1 \cdots b_{n-1}}} \tensor{n}{^d} \tensor{\varepsilon}{_{d a_1 \cdots a_{n-1}}},
		\end{align*}
		将中间 $\left(\tensor{g}{_{c_1 d_1}} \mp \tensor{n}{_{c_1}} \tensor{n}{_{d_1}}\right) \cdots \left(\tensor{g}{_{c_{n-1} d_{n-1}}} \mp \tensor{n}{_{c_{n-1}}} \tensor{n}{_{d_{n-1}}}\right)$ 展开,可以证明含 $\tensor{n}{_{c_i}} \tensor{n}{_{d_i}}$ 的项全为零,因为
		\begin{align*}
		\cdots \tensor{g}{^{a_i c_i}} \tensor{g}{^{b_i d_i}} \cdots \tensor{n}{_{c_i}} \tensor{n}{_{d_i}} \cdots \tensor{n}{^c} \tensor{\varepsilon}{_{c \cdots b_i \cdots}} \tensor{n}{^d} \tensor{\varepsilon}{_{d \cdots a_i \cdots}} &= \cdots \tensor{n}{^{a_i}} \tensor{n}{^d} \tensor{\varepsilon}{_{d\cdots a_i \cdots}} \tensor{n}{^{b_i}} \tensor{n}{^c} \tensor{\varepsilon}{_{c \cdots b_i \cdots}}\\
		&= \cdots \tensor{n}{^{(a_i}} \tensor{n}{^{d)}} \tensor{\varepsilon}{_{d\cdots a_i \cdots}} \tensor{n}{^{(b_i}} \tensor{n}{^{c)}} \tensor{\varepsilon}{_{c \cdots b_i \cdots}}\\
		&= 0,
		\end{align*}
		则
		\begin{align*}
		&\;\tensor{\hat{\varepsilon}}{^{a_1 \cdots a_{n-1} }} \tensor{\hat{\varepsilon}}{_{ a_1 \cdots a_{n-1} }}\\
		=&\; \tensor{g}{^{a_1 c_1}} \tensor{g}{^{b_1 d_1}} \cdots \tensor{g}{^{a_{n-1} c_{n-1}}} \tensor{g}{^{b_{n-1} d_{n-1}}} \tensor{g}{_{c_1 d_1}} \cdots \tensor{g}{_{c_{n-1} d_{n-1}}} \tensor{n}{^c} \tensor{\varepsilon}{_{c b_1 \cdots b_{n-1}}} \tensor{n}{^d} \tensor{\varepsilon}{_{d a_1 \cdots a_{n-1}}}\\
		=&\; \tensor{g}{^{a_1 b_1}} \cdots \tensor{g}{^{a_{n-1} b_{n-1}}} \tensor{n}{^c} \tensor{\varepsilon}{_{c b_1 \cdots b_{n-1}}} \tensor{n}{^d} \tensor{\varepsilon}{_{d a_1 \cdots a_{n-1}}}\\
		=&\; \tensor{n}{_c} \tensor{n}{^d} \tensor{\varepsilon}{^{c a_1 \cdots a_{n-1}}} \tensor{\varepsilon}{_{d a_1 \cdots a_{n-1}}}\\
		=&\; \tensor{n}{_c} \tensor{n}{^d} \left(-1\right)^s \left(n-1\right)! \tensor{\delta}{^c_d}\\
		=& \; (-1)^s (n-1)! \tensor{n}{_c} \tensor{n}{^c},
		\end{align*}
		在 $V_p$ 中取正交归一基底 $\left\{\tensor{{e_\mu}}{^a}\right\}$ ,使得 $\tensor{{e_0}}{^a}=\tensor{n}{^a}$ ,易知
		\begin{displaymath}
		\hat{s}=\begin{cases}
		s, &\qif* \tensor{n}{^a}\tensor{n}{_a}=1;\\
		s-1, &\qif* \tensor{n}{^a}\tensor{n}{_a}=-1.
		\end{cases}
		\end{displaymath}
		于是 $(-1)^s \tensor{n}{^c} \tensor{n}{_c} = (-1)^{\hat{s}} $,证毕。
	\end{zm}
	
	\item 试证定理5-6-1和5-6-2。
	
	\begin{zm}
		\begin{enumerate}
			\item 定理5-6-1 如下
			\begin{yl}{Thm}
				$\tensor[^*^*]{\bm{\omega}}{} = (-1)^{s+l(n-l)} \bm{\omega}. $
			\end{yl}
		    \begin{yl}{Prf}
		    	\begin{align*}
		    	\tensor[^*^*]{\omega}{_{a_1 \cdots a_l}} &= \frac{1}{(n-l)!} \tensor[^*]{\omega}{_{b_1 \cdots b_{n-l}}} \tensor{\varepsilon}{^{b_1 \cdots b_{n-l}}_{a_1 \cdots a_l}} \\
		    	&= \frac{1}{(n-l)!} \frac{1}{l!} \tensor{\omega}{_{c_1 \cdots c_l}} \tensor{\varepsilon}{^{c_1 \cdots c_l}_{b_1 \cdots b_{n-l}}} \tensor{\varepsilon}{^{b_1 \cdots b_{n-l}}_{a_1 \cdots a_l}}\\
		    	&= \frac{1}{(n-l)!} \frac{1}{l!} (-1)^{l(n-l)} \tensor{\varepsilon}{^{b_1 \cdots b_{n-l} c_1 \cdots c_l}} \tensor{\varepsilon}{_{b_1 \cdots b_{n-l}a_1 \cdots a_l}} \tensor{\omega}{_{c_1 \cdots c_l}}\\
		    	&= (-1)^{s+l(n-l)} \tensor{\delta}{^{[c_1}_{a_1}} \cdots \tensor{\delta}{^{c_l]}_{a_l}} \tensor{\omega}{_{[c_1\cdots c_l]}}\\
		    	&= (-1)^{s+l(n-l)} \tensor{\omega}{_{a_1\cdots a_l}}.
		    	\end{align*}
		    \end{yl}
	        \item 定理5-6-2如下
	        \begin{yl}{Thm}
	        	设 $f$ 和 $\myvec{A}$ 是3维欧氏空间的函数和矢量场,则
	        	\[ \ggrad f=\dd{f}\qc \ccurl \myvec{A} = \tensor[^*]{\dd{\bm{A}}}{} \qc \ddiv \myvec{A} = \tensor[^*]{\dd(\tensor[^*]{\bm{A}}{})}{}. \]
	        \end{yl}
            \begin{yl}{Prf}
            	\begin{align*}
            	\tensor{\left(\dd{f}\right)}{^a}&= \pdv{f}{x^i} \tensor{\left(\dd{x^i}\right)}{^a}\\
            	\tensor{\left(\tensor[^*]{\dd{A}}{}\right)}{^k}&= \frac{1}{2} \tensor{\varepsilon}{^i^j^k} \tensor{\left(\dd{A}\right)}{_i_j}\\
            	&= \tensor{\varepsilon}{^i^j^k}  \tensor{A}{_{[j,i]}}\\
            	&= \tensor{\varepsilon}{^i^j^k} \tensor{A}{_{j,i}}\\
            	\tensor[^*]{\dd(\tensor[^*]{\bm{A}}{})}{}&= \frac{1}{6} \tensor{\varepsilon}{^i^j^k} \tensor{\left(\dd(\tensor[^*]{A}{})\right)}{_i_j_k}\\
            	&= \frac{1}{2} \tensor{\varepsilon}{^i^j^k} \tensor{\left(\tensor[^*]{A}{}\right)}{_{[jk,i]}}\\
            	&= \frac{1}{2} \tensor{\varepsilon}{^i^j^k} \Partial{i} \left( \tensor{A}{^l} \tensor{\varepsilon}{_l_j_k} \right)\\
            	&= \tensor{\delta}{^i_l} \tensor{A}{^l_{,i}}\\
            	&= \tensor{A}{^i_{,i}}
            	\end{align*}
            \end{yl}
		\end{enumerate}
	\end{zm}
	
	\item 设$x,y,z $是3维欧氏空间的笛卡尔坐标,试证
	\begin{enumerate}
		\item[(a)] $\tensor[^*]{\dd{x}}{} = \dd{y} \wedge \dd{z} $;
		\item[(b)] $\tensor[^*]{\left( \dd{x} \wedge \dd{y} \wedge \dd{z} \right)}{}= 1 $。
	\end{enumerate}

    \begin{zm}
    	\begin{enumerate}
    		\item[(a)] 
    		\begin{align*}
    		\displaybreak[1] \tensor{\left(\tensor[^*]{\dd{x}}{}\right)}{_a_b}&= \tensor{\left(\dd{x}\right)}{^c} \tensor{\left(\dd{x}\right)}{_c} \wedge \tensor{\left(\dd{y}\right)}{_a} \wedge \tensor{\left(\dd{z}\right)}{_b}\\ \displaybreak[1]
    		&= \tensor{\left(\dd{x}\right)}{^c} \left( \tensor{\left(\dd{x}\right)}{_c} \tensor{\left(\dd{y}\right)}{_a} \tensor{\left(\dd{z}\right)}{_b} - \tensor{\left(\dd{x}\right)}{_c} \tensor{\left(\dd{y}\right)}{_b} \tensor{\left(\dd{z}\right)}{_a} + \cdots \right)\\ \displaybreak[1]
    		&= \tensor{\left(\dd{y}\right)}{_a} \tensor{\left(\dd{z}\right)}{_b} -  \tensor{\left(\dd{y}\right)}{_b} \tensor{\left(\dd{z}\right)}{_a}\\
    		&= \tensor{\left(\dd{y}\wedge\dd{z}\right)}{_a_b}.
    		\end{align*}
    		\item[(b)] 
    		\begin{align*}
    		\tensor[^*]{\left(\dd{x}\wedge\dd{y} \wedge \dd{z} \right)}{} &= \frac{1}{6} \tensor{\left(\dd{x}\wedge\dd{y}\wedge\dd{z} \right)}{^a^b^c} \tensor{\left(\dd{x}\wedge\dd{y}\wedge\dd{z} \right)}{_a_b_c}\\
    		&= 1.
    		\end{align*}
    	\end{enumerate}
    \end{zm}
	
	\item 设$\{r,\theta, \phi \} $是3维欧氏空间的球坐标系,试证$\tensor[^*]{\dd{r}}{} = \left(r^2 \sin\theta \right) \dd{\theta} \wedge \dd{\phi} $。
	
	\begin{zm}
		由第2章~\hyperlink{2.19a}{19(a)}~知
		\begin{displaymath}
		\abs{g} = \tensor{g}{_r_r} \tensor{g}{_{\theta\theta}} \tensor{g}{_\phi_\phi} = r^4 \sin^2 \theta,
		\end{displaymath}
		故体元在球坐标系下为
		\begin{displaymath}
		\tensor{\varepsilon}{_a_b_c} = \sqrt{\abs{g}} \tensor{\left(\dd{r}\right)}{_a} \wedge \tensor{\left(\dd{\theta}\right)}{_b} \wedge \tensor{\left(\dd{\phi}\right)}{_c} = r^2 \sin\theta \tensor{\left(\dd{r}\right)}{_a} \wedge \tensor{\left(\dd{\theta}\right)}{_b} \wedge \tensor{\left(\dd{\phi}\right)}{_c}
		\end{displaymath}
		则
		\begin{align*}
		\tensor{\left(\tensor[^*]{\dd{r}}{}\right)}{_a_b} &= \tensor{\left(\dd{r}\right)}{^c} \tensor{\varepsilon}{_c_a_b}\\
		&= r^2 \sin\theta \tensor{\left(\dd{r}\right)}{^c} \left( \tensor{\left(\dd{r}\right)}{_c} \tensor{\left(\dd{\theta}\right)}{_a} \tensor{\left(\dd{\phi}\right)}{_b} - \tensor{\left(\dd{r}\right)}{_c} \tensor{\left(\dd{\theta}\right)}{_b} \tensor{\left(\dd{\phi}\right)}{_a} + \cdots \right)\\
		&= r^2 \sin\theta \tensor{g}{_r_r} \tensor{\left(\dd{\theta}\wedge\dd{\phi}\right)}{_a_b}\\
		&= r^2 \sin\theta \tensor{\left(\dd{\theta}\wedge\dd{\phi}\right)}{_a_b}.
		\end{align*}
	\end{zm}
	
	\item 设$\myvec{A},\myvec{B} $为$\mathbb{R}^3 $上的矢量场,$\grad $为$\mathbb{R}^3$上与欧氏度规相适配的导数算符,试证\[ \curl(\myvec{A}\cp\myvec{B}) = \left( \myvec{B} \vdot \grad \right) \myvec{A} + \left( \div{\myvec{B}} \right) \myvec{A} - \left( \myvec{A} \vdot \grad \right) \myvec{B} - \left( \div{\myvec{A}} \right) \myvec{B}. \]
	
	\begin{zm}
		\begin{align*}
		\tensor{\varepsilon}{^a^b^c} \Partial{a} \left( \tensor{\varepsilon}{_d_e_b} \tensor{A}{^d} \tensor{B}{^e} \right) &= -\tensor{\varepsilon}{^b^a^c} \tensor{\varepsilon}{_b_d_e} \Partial{a} \left(  \tensor{A}{^d} \tensor{B}{^e} \right)\\
		&= 2 \tensor{\delta}{^{[a}_d} \tensor{\delta}{^{c]}_e} \left( \tensor{A}{^d} \Partial{a} \tensor{B}{^e} + \tensor{B}{^e} \Partial{a} \tensor{A}{^d} \right)\\
		&= \tensor{A}{^a} \Partial{a} \tensor{B}{^c} - \tensor{A}{^c} \Partial{a} \tensor{B}{^a} + \tensor{B}{^c} \Partial{a} \tensor{A}{^a} - \tensor{B}{^a} \Partial{a} \tensor{A}{^c}.
		\end{align*}
	\end{zm}
	
	\item 用微分形式证明3维欧氏空间场论中并不易证的下列熟知命题:
	\begin{enumerate}
		\item[(a)] 无旋矢量场必可表为梯度;
		\item[(b)] 无散矢量场必可表为旋度(见 \S 5.6 末)。
	\end{enumerate}

    \begin{zm}
    	\begin{enumerate}
    		\item[(a)] 设 $\curl{\myvec{A}}=0$ ,则 $\tensor[^*]{\dd{\bm{A}}}{}=0 $,于是 $\bm{A}$ 为闭形式,而 $\mathbb{R}^3$ 单连通,故闭1形式必恰当,于是存在 $f$ 使得 $\bm{A} = \dd{f}$ ,即 $\myvec{A}= \grad{f}$。
    		\item[(b)] 设 $\div{\myvec{B}}=0$,则 $\tensor[^*]{\dd(\tensor[^*]{\bm{B}}{})}{}=0$,于是 $\tensor[^*]{\bm{B}}{}$ 是闭2形式,在 $\mathbb{R}^3$ 中必恰当,于是存在1形式 $\bm{C}$ 使得 $\tensor[^*]{\bm{B}}{} = \dd{\bm{C}}$,即 $\bm{B} = \tensor[^*]{\dd{\bm{C}}}{}$,$\myvec{B}=\curl{\myvec{C}}$。
    	\end{enumerate}
    \end{zm}
    
    \item 设$\Nabla{a} $是广义黎曼空间$\left(M,\tensor{g}{_a_b} \right) $上的适配导数算符(即$\Nabla{a} \tensor{g}{_b_c}=0 $),$\bm{\varepsilon} $是适配体元(即$\Nabla{a} \tensor{\varepsilon}{_{b_1 \cdots b_n}}=0 $),$\tensor{v}{^a} $是$M$上的矢量场,$\tensor{v}{_a} \equiv \tensor{g}{_a_b} \tensor{v}{^b} $是$\tensor{v}{^a} $相应的1形式场,$\tensor[^*]{\bm{v}}{} $是$\tensor{v}{_a} $的对偶形式场,试证$\left( \Nabla{a} \tensor{v}{^a} \right) \bm{\varepsilon} = \dd{\tensor[^*]{\bm{v}}{}} $。注:这个结论可做如下推广:设$\tensor{F}{_{a_1 \cdots a_k}} $是$k$形式场($k\leqslant n $),简记作$\bm{F} $,把$k-1$形式场$\tensor{\nabla}{^{a_k}} \tensor{F}{_{a_1 \cdots a_k}} $记作$\ddiv \bm{F} $,则$\tensor[^*]{\left(\ddiv\bm{F} \right)}{} = \dd{\tensor[^*]{\bm{F}}{}} $。电磁场的麦氏方程[式(12-6-2)]就是一例。
    
    \begin{zm}
    	由第 \hyperlink{5.6.a}{6} 题(a)知,
    	\begin{align*}
    	\tensor{\left(\dd{\tensor[^*]{v}{}}\right)}{_{a_1 \cdots a_n}} &= \dd\indices{_{a_1}} \left( \tensor{v}{^b} \tensor{\varepsilon}{_{b a_2 \cdots a_n}} \right)\\
    	&= \Ld{v} \tensor{\varepsilon}{_{a_1 \cdots a_n}}\\
    	&= \cancel{\tensor{v}{^b} \Nabla{b} \tensor{\varepsilon}{_{a_1 \cdots a_n}}} + \sum_{i} \tensor{\varepsilon}{_{a_1 \cdots b \cdots a_n}} \Nabla{a_i} \tensor{v}{^b}
    	\end{align*}
    	设 $n$ 形式 $\dd{\tensor[^*]{\bm{v}}{}}=h\bm{\varepsilon}$ ,则 $\tensor{\left(\dd{\tensor[^*]{v}{}}\right)}{_{a_1 \cdots a_n}} \tensor{\varepsilon}{^{a_1 \cdots a_n}} = (-1)^s n! h$,另一方面,
    	\begin{align*}
    	\tensor{\varepsilon}{^{a_1 \cdots a_n}} \sum_{i} \tensor{\varepsilon}{_{a_1 \cdots b \cdots a_n}} \Nabla{a_i} \tensor{v}{^b} &= \sum_{i} \tensor{\varepsilon}{^{a_1 \cdots a_n}} \tensor{\varepsilon}{_{a_1 \cdots b \cdots a_n}} \Nabla{a_i} \tensor{v}{^b}\\
    	&= \sum_{i} (-1)^s (n-1)! \tensor{\delta}{^{a_i}_b} \Nabla{a_i} \tensor{v}{^b}\\
    	&= (-1)^s n! \Nabla{b} \tensor{v}{^b},
    	\end{align*}
    	于是 $h=\Nabla{b} \tensor{v}{^b}$ ,即 $\dd{\tensor[^*]{\bm{v}}{}} = \left(\Nabla{b} \tensor{v}{^b}\right) \bm{\varepsilon}$。
    \end{zm}
    
    \item 试证由式(5-7-2)定义的$\ChristoffelSymbol{\sigma}{\mu}{\tau} $正是 \S 3.1定义的克氏符$\ChristoffelSymbol{c}{a}{b} $在式(5-7-2)涉及的坐标基底的分量。
    
    \begin{zm}
    	这个……和 \hyperlink{3.4}{第三章第4题} 重了吧……
    \end{zm}
    
    \item 用正交归一标架分别求第3章习题$\hyperlink{3.14}{14} \sim \hyperlink{3.16}{16}$所给度规的曲率张量的全部标架分量,并验证所得结果与用坐标基底法求得的曲率张量相同。为与$\tensor{R}{_a_b_c^d} $的坐标分量$\tensor{R}{_\mu_\nu_\sigma^\rho} $区别,在求得$\tensor{R}{_a_b_c^d}$的全部标架分量后宜改用符号$\tensor{R}{_{(\mu)}_{(\nu)}_{(\sigma)}^{(\rho)}} $代表标架分量。
	
	\begin{jie}
		\begin{enumerate}
			\item \hyperlink{3.14}{第三章第14题}:
			\begin{yl}{Prob}
				求度规$\dd{s}^2 = \Omega^2(t,x) \left(-\dd{t}^2 + \dd{x}^2\right) $的黎曼张量在$\{t,x\}$系的全部分量(在结果中以$\dot{\Omega} $和$ \Omega^\prime$ 分别代表函数 $\Omega$对$t$和$x$的偏导数)。
			\end{yl}
			\begin{yl}{Solv}
				用式(5-7-20)计算。
				\begin{enumerate}[leftmargin=2em]
					\item 选取标架:由于是洛伦兹度规,设有正交归一基底 $\{ \tensor{\left(e_\mu\right)}{^a} \} $,
					\begin{align*}
					\displaybreak[1] \tensor{g}{_a_b} &= - \Omega^2(t,x) \tensor{\left(\dd{t}\right)}{_a} \tensor{\left(\dd{t}\right)}{_b} + \Omega^2(t,x) \tensor{\left(\dd{x}\right)}{_a} \tensor{\left(\dd{x}\right)}{_b}\\ \displaybreak[1]
					&= \tensor{\eta}{_\mu_\nu} \tensor{\left(e^\mu\right)}{_a} \tensor{\left(e^\nu\right)}{_b}\\
					&= - \tensor{\left(e^0\right)}{_a} \tensor{\left(e^0\right)}{_b} + \tensor{\left(e^1\right)}{_a} \tensor{\left(e^1\right)}{_b},
					\end{align*}
					最简单的选择是
					\begin{displaymath}
					\tensor{\left(e^0\right)}{_a} = \Omega(t,x) \tensor{\left(\dd{t}\right)}{_a} \qc \tensor{\left(e^1\right)}{_a} = \Omega(t,x) \tensor{\left(\dd{x}\right)}{_a},
					\end{displaymath}
					于是
					\begin{displaymath}
					\tensor{\left(e_0\right)}{^a} = \frac{1}{\Omega} \tensor{\left(\pdv{t}\right)}{^a} \qc \tensor{\left(e_1\right)}{^a} = \frac{1}{\Omega} \tensor{\left(\pdv{x}\right)}{^a}.
					\end{displaymath}
					并有
					\begin{displaymath}
					\tensor{\left(e_0\right)}{_a} = \tensor{\eta}{_0_\nu} \tensor{\left(e^\nu\right)}{_a} = - \Omega \tensor{\left(\dd{t}\right)}{_a} \qc \tensor{\left(e_1\right)}{_a} = \tensor{\eta}{_1_\nu} \tensor{\left(e^\nu\right)}{_a} = \Omega \tensor{\left(\dd{x}\right)}{_a}
					\end{displaymath}
					\item 计算联络1形式:由(5-7-19):
					\begin{displaymath}
					\tensor{\Lambda}{_\mu_\nu_\rho} \equiv \left[ \tensor{\left(e_\nu\right)}{_{\lambda,\tau}} - \tensor{\left(e_\nu\right)}{_{\tau,\lambda}} \right] \tensor{\left(e_\mu\right)}{^\lambda} \tensor{\left(e_\rho\right)}{^\tau},
					\end{displaymath}
					易知必有 $\lambda=\mu$,$\tau=\rho$,由于 $\mu\rho$ 反称,只需取 $\mu=0,\rho=1$ 的项计算:
					\begin{align*}
					\tensor{\Lambda}{_0_0_1} &= \left[ \tensor{\left(e_0\right)}{_{0,1}} - \tensor{\left(e_0\right)}{_{1,0}} \right] \tensor{\left(e_0\right)}{^0} \tensor{\left(e_1\right)}{^1}\\
					&= \frac{1}{\Omega^2} \left( - \Omega^\prime - 0 \right)\\
					&= - \frac{\Omega^\prime}{\Omega^2}\\
					\tensor{\Lambda}{_0_1_1} &= \left[ \tensor{\left(e_1\right)}{_{0,1}} - \tensor{\left(e_1\right)}{_{1,0}} \right] \tensor{\left(e_0\right)}{^0} \tensor{\left(e_1\right)}{^1}\\
					&= \frac{1}{\Omega^2} \left( 0- \dot{\Omega} \right) \\
					&= -\frac{\dot{\Omega}}{\Omega^2}		
					\end{align*}
					故所有非零项:
					\begin{displaymath}
					\tensor{\Lambda}{_0_0_1} = - \tensor{\Lambda}{_1_0_0} = - \frac{\Omega^\prime}{\Omega^2} \qc \tensor{\Lambda}{_0_1_1} = - \tensor{\Lambda}{_1_1_0} = - \frac{\dot{\Omega}}{\Omega^2}
					\end{displaymath}
					于是由式(5-7-20):
					\begin{displaymath}
					\tensor{\omega}{_\mu_\nu_\rho} = \frac{1}{2} \left( \tensor{\Lambda}{_\mu_\nu_\rho} + \tensor{\Lambda}{_\rho_\mu_\nu} - \tensor{\Lambda}{_\nu_\rho_\mu} \right)
					\end{displaymath}
					由 $\mu\nu $ 反称,只需取 $\mu=0,\nu=1$ :
					\begin{align*}
					\displaybreak[1] \tensor{\omega}{_0_1_0} &= \frac{1}{2} \left( \tensor{\Lambda}{_0_1_0} + \tensor{\Lambda}{_0_0_1} - \tensor{\Lambda}{_1_0_0} \right)\\ \displaybreak[1]
					&= - \frac{\Omega^\prime}{\Omega^2} \\ \displaybreak[1]
					\tensor{\omega}{_0_1_1} &= \frac{1}{2} \left( \tensor{\Lambda}{_0_1_1} + \tensor{\Lambda}{_1_0_1} - \tensor{\Lambda}{_1_1_0} \right)\\
					&= - \frac{\dot{\Omega}}{\Omega^2}
					\end{align*}
					于是所有非零项为
					\begin{displaymath}
					\tensor{\omega}{_0_1_0} = - \tensor{\omega}{_1_0_0} = - \frac{\Omega^\prime}{\Omega^2} \qc \tensor{\omega}{_0_1_1} = - \tensor{\omega}{_1_0_1} = - \frac{\dot{\Omega}}{\Omega^2}
					\end{displaymath}
					故
					\begin{align*}
					\tensor{\bm{\omega}}{_0_0} &= 0, & \tensor{\bm{\omega}}{_0_1} &= - \frac{\Omega^\prime}{\Omega^2} \dd{t} - \frac{\dot{\Omega}}{\Omega^2} \dd{x},\\
					\tensor{\bm{\omega}}{_1_0} &= \frac{\Omega^\prime}{\Omega^2} \dd{t} + \frac{\dot{\Omega}}{\Omega^2} \dd{x}, & \tensor{\bm{\omega}}{_1_1} &= 0.
					\end{align*}
					进而
					\begin{align*}
					\tensor{\bm{\omega}}{_0^0} &= 0, & \tensor{\bm{\omega}}{_0^1} &= - \frac{\Omega^\prime}{\Omega^2} \dd{t} - \frac{\dot{\Omega}}{\Omega^2} \dd{x},\\
					\tensor{\bm{\omega}}{_1^0} &= - \frac{\Omega^\prime}{\Omega^2} \dd{t} - \frac{\dot{\Omega}}{\Omega^2} \dd{x}, & \tensor{\bm{\omega}}{_1^1} &= 0.
					\end{align*}
					\item 计算曲率2形式。用嘉当第二结构方程:
					\begin{displaymath}
					\tensor{\bm{R}}{_\mu^\nu} = \dd{\tensor{\bm{\omega}}{_\mu^\nu}} + \tensor{\bm{\omega}}{_\mu^\lambda} \wedge \tensor{\bm{\omega}}{_\lambda^\nu}
					\end{displaymath}
					联络1形式的外微分计算如下
					\begin{align*}
					\dd{\tensor{\bm{\omega}}{_0^0}} =&\; 0\\
					\dd{\tensor{\bm{\omega}}{_0^1}} =&\tmu \left( \frac{2\Omega^\prime \dot{\Omega} - \Omega \dot{\Omega}^\prime}{\Omega^3} \dd{t} + \frac{2{\Omega^\prime}^2 - \Omega \Omega''}{\Omega^3} \dd{x} \right) \wedge \dd{t} \\
					&+ \left( \frac{2 \dot{\Omega}^2- \Omega \ddot{\Omega}}{\Omega^3} \dd{t} + \frac{2\dot{\Omega}\Omega^\prime - \Omega \dot{\Omega}^\prime}{\Omega^3} \dd{x} \right) \wedge \dd{x}\\
					=& \mspace{5mu} \frac{2\left(\dot{\Omega}^2 - {\Omega^\prime}^2\right) + \Omega \left(\Omega'' - \ddot{\Omega}\right)}{\Omega^3} \dd{t} \wedge \dd{x}\\
					=& \mspace{5mu} \frac{2\left(\dot{\Omega}^2 - {\Omega^\prime}^2\right) + \Omega \left(\Omega'' - \ddot{\Omega}\right)}{\Omega^5} \bm{e}^0 \wedge \bm{e}^1\\
					\dd{\tensor{\bm{\omega}}{_1^0}} =& \tmu \dd{\tensor{\bm{\omega}}{_0^1}}\\ \displaybreak[1]
					=& \mspace{5mu} \frac{2\left(\dot{\Omega}^2 - {\Omega^\prime}^2\right) + \Omega \left(\Omega'' - \ddot{\Omega}\right)}{\Omega^5} \bm{e}^0 \wedge \bm{e}^1\\
					\dd{\tensor{\bm{\omega}}{_1^1}} =& \; 0.
					\end{align*}
					于是
					\begin{align*}
					\tensor{\bm{R}}{_0^0} &= \dd{\tensor{\bm{\omega}}{_0^0}} + \tensor{\bm{\omega}}{_0^1} \wedge \tensor{\bm{\omega}}{_1^0}\\
					&= 0\\
					\tensor{\bm{R}}{_0^1} &= \dd{\tensor{\bm{\omega}}{_0^1}} + 0\\
					&= \frac{2\left(\dot{\Omega}^2 - {\Omega^\prime}^2\right) + \Omega \left(\Omega'' - \ddot{\Omega}\right)}{\Omega^5} \bm{e}^0 \wedge \bm{e}^1\\
					\tensor{\bm{R}}{_1^0} &= \dd{\tensor{\bm{\omega}}{_1^0}} + 0\\
					&= \frac{2\left(\dot{\Omega}^2 - {\Omega^\prime}^2\right) + \Omega \left(\Omega'' - \ddot{\Omega}\right)}{\Omega^5} \bm{e}^0 \wedge \bm{e}^1\\
					\tensor{\bm{R}}{_1^1} &= \dd{\tensor{\bm{\omega}}{_1^1}} + \tensor{\bm{\omega}}{_1^0} \wedge \tensor{\bm{\omega}}{_0^1}\\
					&= 0.
					\end{align*}
					于是黎曼张量的非零标架分量为
					\begin{align*}
					&\tensor{R}{_{(0)(1)(0)}^{(1)}} = - \tensor{R}{_{(1)(0)(0)}^{(1)}} = \tensor{R}{_{(0)(1)(1)}^{(0)}} = - \tensor{R}{_{(1)(0)(1)}^{(0)}}\\
					=& \mspace{5mu} \frac{2\left(\dot{\Omega}^2 - {\Omega^\prime}^2\right) + \Omega \left(\Omega'' - \ddot{\Omega}\right)}{\Omega^5}
					\end{align*}
					\item[{\heiti 验证}] 由
					\begin{gather*}
					\tensor{\left(e^0\right)}{_a} = \Omega(t,x) \tensor{\left(\dd{t}\right)}{_a} \qc \tensor{\left(e^1\right)}{_a} = \Omega(t,x) \tensor{\left(\dd{x}\right)}{_a},\\
					\tensor{\left(e_0\right)}{^a} = \frac{1}{\Omega} \tensor{\left(\pdv{t}\right)}{^a} \qc \tensor{\left(e_1\right)}{^a} = \frac{1}{\Omega} \tensor{\left(\pdv{x}\right)}{^a}
					\end{gather*}
					知
					\begin{align*}
					\tensor{R}{_a_b_c^d} &= \tensor{R}{_{(\mu)}_{(\nu)}_{(\sigma)}^{(\rho)}} \tensor{\left(e^\mu\right)}{_a} \tensor{\left(e^\nu\right)}{_b} \tensor{\left(e^\sigma\right)}{_c} \tensor{\left(e_\rho\right)}{^d}\\
					&= \Omega^3 \tensor{R}{_{(\mu)}_{(\nu)}_{(\sigma)}^{(\rho)}} \tensor{\left(\dd{x^\mu}\right)}{_a} \tensor{\left(\dd{x^\nu}\right)}{_b} \tensor{\left(\dd{x^\sigma}\right)}{_c} \tensor{\left(\pdv{x^\rho}\right)}{^d}
					\end{align*}
					故应有
					\begin{displaymath}
					\tensor{R}{_\mu_\nu_\sigma^\rho} = \Omega^3 \tensor{R}{_{(\mu)}_{(\nu)}_{(\sigma)}^{(\rho)}}
					\end{displaymath}
					与第三章求得的
					\[ \tensor{R}{_t_x_x^t}= -\tensor{R}{_x_t_x^t}=\tensor{R}{_t_x_t^x}= -\tensor{R}{_x_t_t^x}= \frac{\Omega \left( \Omega^{\prime\prime}- \ddot{\Omega} \right)+ \dot{\Omega}^2 - {\Omega^\prime}^2 }{\Omega^2} \]
					对比,知两种方法是一致的。
				\end{enumerate}
			\end{yl}
		    \item \hyperlink{3.15}{第三章第15题}:
		    \begin{yl}{Prob}
		    	求度规$\dd{s}^2=z^{-1/2} \left(-\dd{t}^2+\dd{z}^2\right)+ z\left(\dd{x}^2+\dd{y}^2\right)$的黎曼张量在$\{t,x,y,z\}$系的全部分量。
		    \end{yl}
	        \begin{yl}{Solv}
	        	\begin{enumerate}[leftmargin=2em]
	        		\item 选取标架。由于度规是洛伦兹的,设有正交归一基底 $\{ \tensor{\left(e_\mu\right)}{^a} \} $,
	        		\begin{align*}
	        		\tensor{g}{_a_b} &= \frac{1}{\sqrt{z}} \left( - \tensor{\left(\dd{t}\right)}{_a} \tensor{\left(\dd{t}\right)}{_b} + \tensor{\left(\dd{z}\right)}{_a} \tensor{\left(\dd{z}\right)}{_b} \right) + z \left( \tensor{\left(\dd{x}\right)}{_a} \tensor{\left(\dd{x}\right)}{_b} + \tensor{\left(\dd{y}\right)}{_a} \tensor{\left(\dd{y}\right)}{_b} \right) \\
	        		&= \tensor{\eta}{_\mu_\nu} \tensor{\left(e^\mu\right)}{_a} \tensor{\left(e^\nu\right)}{_b}\\
	        		&= - \tensor{\left(e^0\right)}{_a} \tensor{\left(e^0\right)}{_b} + \tensor{\left(e^1\right)}{_a} \tensor{\left(e^1\right)}{_b} + \tensor{\left(e^2\right)}{_a} \tensor{\left(e^2\right)}{_b} + \tensor{\left(e^3\right)}{_a} \tensor{\left(e^3\right)}{_b}
	        		\end{align*}
	        		最简单的选择是
	        		\begin{align*}
	        		\tensor{\left(e^0\right)}{_a} &= \frac{1}{\sqrt[4]{z}} \tensor{\left(\dd{t}\right)}{_a} \\
	        		\tensor{\left(e^1\right)}{_a} &= \sqrt{z} \tensor{\left(\dd{x}\right)}{_a} \\
	        		\tensor{\left(e^2\right)}{_a} &= \sqrt{z} \tensor{\left(\dd{y}\right)}{_a} \\
	        		\tensor{\left(e^3\right)}{_a} &= \frac{1}{\sqrt[4]{z}} \tensor{\left(\dd{z}\right)}{_a} 
	        		\end{align*}
	        		于是标架基矢为
	        		\begin{align*}
	        		\tensor{\left(e_0\right)}{^a} &= \sqrt[4]{z} \tensor{\left(\pdv{t}\right)}{^a} \\
	        		\tensor{\left(e_1\right)}{^a} &= \frac{1}{\sqrt{z}} \tensor{\left(\pdv{x}\right)}{^a}\\
	        		\tensor{\left(e_2\right)}{^a} &= \frac{1}{\sqrt{z}} \tensor{\left(\pdv{y}\right)}{^a} \\
	        		\tensor{\left(e_3\right)}{^a} &= \sqrt[4]{z} \tensor{\left(\pdv{z}\right)}{^a}
	        		\end{align*}
	        	    并有
	        	    \begin{align*}
	        	    \tensor{\left(e_0\right)}{_a} &= \tensor{\eta}{_0_\nu} \tensor{\left(e^\nu\right)}{_a}\\ \displaybreak[1]
	        	    &= - \frac{1}{\sqrt[4]{z}} \tensor{\left(\dd{t}\right)}{_a} \\
	        	    \tensor{\left(e_1\right)}{_a} &= \tensor{\eta}{_1_\nu} \tensor{\left(e^\nu\right)}{_a}\\ \displaybreak[1]
	        	    &= \sqrt{z} \tensor{\left(\dd{x}\right)}{_a}\\
	        	    \tensor{\left(e_2\right)}{_a} &= \tensor{\eta}{_2_\nu} \tensor{\left(e^\nu\right)}{_a}\\ \displaybreak[1]
	        	    &= \sqrt{z} \tensor{\left(\dd{y}\right)}{_a}\\
	        	    \tensor{\left(e_3\right)}{_a} &= \tensor{\eta}{_3_\nu} \tensor{\left(e^\nu\right)}{_a}\\ \displaybreak[1]
	        	    &= \frac{1}{\sqrt[4]{z}} \tensor{\left(\dd{z}\right)}{_a}
	        	    \end{align*}
	        	    \item 计算联络1形式。\begin{displaymath}
	        	    \tensor{\Lambda}{_\mu_\nu_\rho} \equiv \left[ \tensor{\left(e_\nu\right)}{_{\lambda,\tau}} - \tensor{\left(e_\nu\right)}{_{\tau,\lambda}} \right] \tensor{\left(e_\mu\right)}{^\lambda} \tensor{\left(e_\rho\right)}{^\tau},
	        	    \end{displaymath}
	        	    易知必有 $\lambda=\mu$,$\tau=\rho$。求导仅对 $z$ 求不为零,故 $\mu,\rho$ 至少有一个取 $3$ ,而其余两个相同。由于 $\mu\rho$ 反称,只需取 $\mu<\rho$ 的项计算:
	        	    \begin{align*}
	        	    \tensor{\Lambda}{_0_0_3} &= \sqrt{z} \left( \frac{1}{4} z^{-\frac{5}{4}} - 0 \right)\\
	        	    &= \frac{1}{4} z^{-\frac{3}{4}}\\
	        	    \tensor{\Lambda}{_1_1_3} &= \frac{1}{\sqrt[4]{z}} \left( \frac{1}{2\sqrt{z}} - 0 \right)\\
	        	    &= \frac{1}{2} z^{-\frac{3}{4}}\\
	        	    \tensor{\Lambda}{_2_2_3} &= \frac{1}{\sqrt[4]{z}} \left( \frac{1}{2\sqrt{z}} \right)\\
	        	    &= \frac{1}{2} z^{-\frac{3}{4}}
	        	    \end{align*}
	        	    故所有非零项为
	        	    \begin{displaymath}
	        	    \tensor{\Lambda}{_0_0_3} = - \tensor{\Lambda}{_3_0_0} = \frac{1}{4} z^{-\frac{3}{4}} \qc
	        	    \tensor{\Lambda}{_1_1_3} = - \tensor{\Lambda}{_3_1_1} = \tensor{\Lambda}{_2_2_3} = - \tensor{\Lambda}{_3_2_2} = \frac{1}{2} z^{-\frac{3}{4}}
	        	    \end{displaymath}
	        	    于是由式(5-7-20):
	        	    \begin{displaymath}
	        	    \tensor{\omega}{_\mu_\nu_\rho} = \frac{1}{2} \left( \tensor{\Lambda}{_\mu_\nu_\rho} + \tensor{\Lambda}{_\rho_\mu_\nu} - \tensor{\Lambda}{_\nu_\rho_\mu} \right)
	        	    \end{displaymath}
	        	    同样 $\mu,\nu,\rho$ 中有一个取3,另两个相同,由于 $\mu\nu$ 反称,故 $\nu$ 取3而$\mu\rho$ 相同:
	        	    \begin{align*}
	        	    \tensor{\omega}{_\mu_3_\mu} &= \frac{1}{2} \left( \tensor{\Lambda}{_\mu_3_\mu} + \tensor{\Lambda}{_\mu_\mu_3} - \tensor{\Lambda}{_3_\mu_\mu} \right)\\
	        	    &= \tensor{\Lambda}{_\mu_\mu_3}
	        	    \end{align*}
	        	    故
	        	    \begin{displaymath}
	        	    \tensor{\omega}{_0_3_0} = - \tensor{\omega}{_3_0_0} = \frac{1}{4} z^{-\frac{3}{4}} \qc \tensor{\omega}{_1_3_1} = - \tensor{\omega}{_3_1_1} = \tensor{\omega}{_2_3_2} = - \tensor{\omega}{_3_2_2} = \frac{1}{2} z^{-\frac{3}{4}}
	        	    \end{displaymath}
	        	    于是
	        	    \begin{align*}
	        	    \tensor{\bm{\omega}}{_0_3} &= - \tensor{\bm{\omega}}{_3_0} = \frac{1}{4} z^{-\frac{3}{4}} \bm{e}^0 = \frac{1}{4} z^{-1} \dd{t} \\
	        	    \tensor{\bm{\omega}}{_1_3} &= - \tensor{\bm{\omega}}{_3_1} = \frac{1}{2} z^{-\frac{3}{4}} \bm{e}^1 = \frac{1}{2} z^{-\frac{1}{4}} \dd{x} \\
	        	    \tensor{\bm{\omega}}{_2_3} &= - \tensor{\bm{\omega}}{_3_2} = \frac{1}{2} z^{-\frac{3}{4}} \bm{e}^2 = \frac{1}{2} z^{-\frac{1}{4}} \dd{y}
	        	    \end{align*}
	        	    用 $\tensor{\eta}{^\mu^\nu}$ 升编号指标:
	        	    \begin{align*}
	        	    \tensor{\bm{\omega}}{_0^3} =  \tensor{\bm{\omega}}{_3^0} &= \frac{1}{4} z^{-\frac{3}{4}} \bm{e}^0 = \frac{1}{4} z^{-1} \dd{t} \\
	        	    \tensor{\bm{\omega}}{_1^3} = - \tensor{\bm{\omega}}{_3^1} &= \frac{1}{2} z^{-\frac{3}{4}} \bm{e}^1 = \frac{1}{2} z^{-\frac{1}{4}} \dd{x}\\
	        	    \tensor{\bm{\omega}}{_2^3} = - \tensor{\bm{\omega}}{_3^2} &= \frac{1}{2} z^{-\frac{3}{4}} \bm{e}^2 = \frac{1}{2} z^{-\frac{1}{4}} \dd{y}
	        	    \end{align*}
	        	    \item 计算曲率2形式。首先计算联络1形式的外微分:
	        	    \begin{align*}
	        	    \dd{\tensor{\bm{\omega}}{_0^3}} = \dd{\tensor{\bm{\omega}}{_3^0}} &= \frac{1}{4} z^{-2} \dd{t} \wedge \dd{z} = \frac{1}{4} z^{-\frac{3}{2}} \bm{e}^0 \wedge \bm{e}^3 \\
	        	    \dd{\tensor{\bm{\omega}}{_1^3}} = - \dd{\tensor{\bm{\omega}}{_3^1}} &= \frac{1}{8} z^{-\frac{5}{4}} \dd{x} \wedge \dd{z} = \frac{1}{8} z^{-\frac{3}{2}} \bm{e}^1 \wedge \bm{e}^3 \\
	        	    \dd{\tensor{\bm{\omega}}{_2^3}} = - \dd{\tensor{\bm{\omega}}{_3^2}} &= \frac{1}{8} z^{-\frac{5}{4}} \dd{y} \wedge \dd{z} = \frac{1}{8} z^{-\frac{3}{2}} \bm{e}^2 \wedge \bm{e}^3
	        	    \end{align*}
	        	    于是由嘉当第二结构方程
	        	    \begin{displaymath}
	        	    \tensor{\bm{R}}{_\mu^\nu} = \dd{\tensor{\bm{\omega}}{_\mu^\nu}} + \tensor{\bm{\omega}}{_\mu^\lambda} \wedge \tensor{\bm{\omega}}{_\lambda^\nu}
	        	    \end{displaymath}
	        	    $\mu\nu$中没有3时,第二项中 $\lambda$ 为3;有一个为3时,第二项中$\lambda$取3或不取3都为零:
	        	    \begin{align*}
	        	    \tensor{\bm{R}}{_0^1} &= \tensor{\bm{\omega}}{_0^3} \wedge \tensor{\bm{\omega}}{_3^1}\\
	        	    &=- \frac{1}{8} z^{-\frac{3}{2}} \bm{e}^0 \wedge \bm{e}^1 \\
	        	    \tensor{\bm{R}}{_0^2} &= \tensor{\bm{\omega}}{_0^3} \wedge \tensor{\bm{\omega}}{_3^2}\\
	        	    &=- \frac{1}{8} z^{-\frac{3}{2}} \bm{e}^0 \wedge \bm{e}^2 \\
	        	    \tensor{\bm{R}}{_0^3} &= \dd{\tensor{\bm{\omega}}{_0^3}}\\
	        	    &= \frac{1}{4} z^{-\frac{3}{2}} \bm{e}^0 \wedge \bm{e}^3\\
	        	    \tensor{\bm{R}}{_1^0} &= \tensor{\bm{\omega}}{_1^3} \wedge \tensor{\bm{\omega}}{_3^0}\\
	        	    &=- \frac{1}{8} z^{-\frac{3}{2}} \bm{e}^0 \wedge \bm{e}^1 \\
	        	    \tensor{\bm{R}}{_1^2} &= \tensor{\bm{\omega}}{_1^3} \wedge \tensor{\bm{\omega}}{_3^2}\\
	        	    &=- \frac{1}{4} z^{-\frac{3}{2}} \bm{e}^1 \wedge \bm{e}^2 \\
	        	    \tensor{\bm{R}}{_1^3} &= \dd{\tensor{\bm{\omega}}{_1^3}}\\ \displaybreak[1]
	        	    &= \frac{1}{8} z^{-\frac{3}{2}} \bm{e}^1 \wedge \bm{e}^3\\
	        	    \tensor{\bm{R}}{_2^0} &= \tensor{\bm{\omega}}{_2^3} \wedge \tensor{\bm{\omega}}{_3^0}\\ \displaybreak[1]
	        	    &=- \frac{1}{8} z^{-\frac{3}{2}} \bm{e}^0 \wedge \bm{e}^2 \\
	        	    \tensor{\bm{R}}{_2^1} &= \tensor{\bm{\omega}}{_2^3} \wedge \tensor{\bm{\omega}}{_3^1}\\ \displaybreak[1]
	        	    &= \frac{1}{8} z^{-\frac{3}{2}} \bm{e}^1 \wedge \bm{e}^2 \\
	        	    \tensor{\bm{R}}{_2^3} &= \dd{\tensor{\bm{\omega}}{_2^3}}\\ \displaybreak[1]
	        	    &= \frac{1}{8} z^{-\frac{3}{2}} \bm{e}^2 \wedge \bm{e}^3\\
	        	    \tensor{\bm{R}}{_3^0} &= \dd{\tensor{\bm{\omega}}{_3^0}}\\ \displaybreak[1]
	        	    &= \frac{1}{4} z^{-\frac{3}{2}} \bm{e}^0 \wedge \bm{e}^3\\
	        	    \tensor{\bm{R}}{_3^1} &= \dd{\tensor{\bm{\omega}}{_3^1}}\\ \displaybreak[1]
	        	    &= - \frac{1}{8} z^{-\frac{3}{2}} \bm{e}^1 \wedge \bm{e}^3\\
	        	    \tensor{\bm{R}}{_3^2} &= \dd{\tensor{\bm{\omega}}{_3^2}}\\ \displaybreak[1]
	        	    &= - \frac{1}{8} z^{-\frac{3}{2}} \bm{e}^2 \wedge \bm{e}^3\\
	        	    \tensor{\bm{R}}{_3^3} &= \tensor{\bm{\omega}}{_3^\lambda} \wedge \tensor{\bm{\omega}}{_\lambda^3} \\
	        	    &= 0
	        	    \end{align*}
	        	    于是所有非零的标架分量为
	        	    \begin{align*}
	        	    \tensor{R}{_{(0)(1)(0)}^{(1)}} &= - \tensor{R}{_{(1)(0)(0)}^{(1)}} = - \frac{1}{8} z^{-\frac{3}{2}} \\
	        	    \tensor{R}{_{(0)(2)(0)}^{(2)}} &= - \tensor{R}{_{(2)(0)(0)}^{(2)}} = - \frac{1}{8} z^{-\frac{3}{2}} \\
	        	    \tensor{R}{_{(0)(3)(0)}^{(3)}} &= - \tensor{R}{_{(3)(0)(0)}^{(3)}} = \frac{1}{4} z^{-\frac{3}{2}} \\
	        	    \tensor{R}{_{(1)(0)(1)}^{(0)}} &= - \tensor{R}{_{(0)(1)(1)}^{(0)}} = \frac{1}{8} z^{-\frac{3}{2}} \\
	        	    \tensor{R}{_{(1)(2)(1)}^{(2)}} &= - \tensor{R}{_{(2)(1)(1)}^{(2)}} = - \frac{1}{4} z^{-\frac{3}{2}} \\
	        	    \tensor{R}{_{(1)(3)(1)}^{(3)}} &= - \tensor{R}{_{(3)(1)(1)}^{(3)}} = \frac{1}{8} z^{-\frac{3}{2}} \\ \displaybreak[1]
	        	    \tensor{R}{_{(2)(0)(2)}^{(0)}} &= - \tensor{R}{_{(0)(2)(2)}^{(0)}} = \frac{1}{8} z^{-\frac{3}{2}} \\ \displaybreak[1]
	        	    \tensor{R}{_{(2)(1)(2)}^{(1)}} &= - \tensor{R}{_{(1)(2)(2)}^{(1)}} = - \frac{1}{8} z^{-\frac{3}{2}} \\ \displaybreak[1]
	        	    \tensor{R}{_{(2)(3)(2)}^{(3)}} &= - \tensor{R}{_{(3)(2)(2)}^{(3)}} = \frac{1}{8} z^{-\frac{3}{2}} \\ \displaybreak[1]
	        	    \tensor{R}{_{(3)(0)(3)}^{(0)}} &= - \tensor{R}{_{(0)(3)(3)}^{(0)}} = - \frac{1}{4} z^{-\frac{3}{2}} \\ \displaybreak[1]
	        	    \tensor{R}{_{(3)(1)(1)}^{(3)}} &= - \tensor{R}{_{(1)(3)(1)}^{(3)}} = \frac{1}{8} z^{-\frac{3}{2}} \\
	        	    \tensor{R}{_{(3)(2)(3)}^{(2)}} &= - \tensor{R}{_{(2)(3)(3)}^{(2)}} = \frac{1}{8} z^{-\frac{3}{2}}
	        	    \end{align*}
	        	    \item[{\heiti 验证}] 根据
	        	    \begin{align*}
	        	    \tensor{\left(e^0\right)}{_a} &= \frac{1}{\sqrt[4]{z}} \tensor{\left(\dd{t}\right)}{_a} \\
	        	    \tensor{\left(e^1\right)}{_a} &= \sqrt{z} \tensor{\left(\dd{x}\right)}{_a} \\
	        	    \tensor{\left(e^2\right)}{_a} &= \sqrt{z} \tensor{\left(\dd{y}\right)}{_a} \\
	        	    \tensor{\left(e^3\right)}{_a} &= \frac{1}{\sqrt[4]{z}} \tensor{\left(\dd{z}\right)}{_a} 
	        	    \end{align*}
	        	    和
	        	    \begin{align*}
	        	    \tensor{\left(e_0\right)}{^a} &= \sqrt[4]{z} \tensor{\left(\pdv{t}\right)}{^a} \\
	        	    \tensor{\left(e_1\right)}{^a} &= \frac{1}{\sqrt{z}} \tensor{\left(\pdv{x}\right)}{^a}\\
	        	    \tensor{\left(e_2\right)}{^a} &= \frac{1}{\sqrt{z}} \tensor{\left(\pdv{y}\right)}{^a} \\
	        	    \tensor{\left(e_3\right)}{^a} &= \sqrt[4]{z} \tensor{\left(\pdv{z}\right)}{^a}
	        	    \end{align*}
	        	    ……懒得验证了\wl{12}
	        	\end{enumerate}
	        \end{yl}
            \item \hyperlink{3.16}{第三章第16题}:
            \begin{yl}{Prob}
            	设$\alpha(z)$,$\beta(z)$,$\gamma(z)$为任意函数,$h=t+ \alpha(z)x+\beta(z)y+\gamma(z) $,求度规\[ \dd{s}^2= -\dd{t}^2 + \dd{x}^2 +\dd{y}^2 +h^2\dd{z}^2 \] 的黎曼张量在$\{t,x,y,z\}$系的全部分量。
            \end{yl}
            \begin{yl}{Solv}
            	\begin{enumerate}[leftmargin=2em]
            		\item 选取标架。令
            		\begin{align*}
            		\displaybreak[1] \tensor{\left(e^0\right)}{_a} &= \tensor{\left(\dd{t}\right)}{_a}\\ \displaybreak[1]
            		\tensor{\left(e^1\right)}{_a} &= \tensor{\left(\dd{x}\right)}{_a}\\ \displaybreak[1]
            		\tensor{\left(e^2\right)}{_a} &= \tensor{\left(\dd{y}\right)}{_a}\\
            		\tensor{\left(e^3\right)}{_a} &= h \tensor{\left(\dd{z}\right)}{_a}
            		\end{align*}
            		则标架基矢为
            		\begin{align*}
            		\tensor{\left(e_0\right)}{^a} &= \tensor{\left(\pdv{t}\right)}{^a}\\
            		\tensor{\left(e_1\right)}{^a} &= \tensor{\left(\pdv{x}\right)}{^a} \\
            		\tensor{\left(e_2\right)}{^a} &= \tensor{\left(\pdv{y}\right)}{^a} \\
            		\tensor{\left(e_3\right)}{^a} &= \frac{1}{h} \tensor{\left(\pdv{z}\right)}{^a}
            		\end{align*}
            		并有
            		\begin{align*}
            		\tensor{\left(e_0\right)}{_a} &= \tensor{\eta}{_0_\nu} \tensor{\left(e^\nu\right)}{_a}\\ \displaybreak[1]
            		&= - \tensor{\left(\dd{t}\right)}{_a} \\
            		\tensor{\left(e_1\right)}{_a} &= \tensor{\eta}{_1_\nu} \tensor{\left(e^\nu\right)}{_a}\\ \displaybreak[1]
            		&= \tensor{\left(\dd{x}\right)}{_a}\\
            		\tensor{\left(e_2\right)}{_a} &= \tensor{\eta}{_2_\nu} \tensor{\left(e^\nu\right)}{_a}\\ \displaybreak[1]
            		&= \tensor{\left(\dd{y}\right)}{_a}\\
            		\tensor{\left(e_3\right)}{_a} &= \tensor{\eta}{_3_\nu} \tensor{\left(e^\nu\right)}{_a}\\ \displaybreak[1]
            		&= h \tensor{\left(\dd{z}\right)}{_a}
            		\end{align*}
            		\item 计算联络1形式。
            		\begin{displaymath}
            		\tensor{\Lambda}{_\mu_\nu_\rho} \equiv \left[ \tensor{\left(e_\nu\right)}{_{\lambda,\tau}} - \tensor{\left(e_\nu\right)}{_{\tau,\lambda}} \right] \tensor{\left(e_\mu\right)}{^\lambda} \tensor{\left(e_\rho\right)}{^\tau},
            		\end{displaymath}
            		$\lambda\tau$ 必须取 $\mu\rho$ ;为使导数项不为零,$\nu$ 必须为3,$\lambda\tau$ ,因而$\mu\rho$中有一个为3。
            		\begin{align*}
            		\tensor{\Lambda}{_0_3_3} &= - \frac{1}{h} \pdv{h}{t}\\
            		&= -\frac{1}{h}\\
            		\tensor{\Lambda}{_1_3_3} &= - \frac{1}{h} \pdv{h}{x}\\
            		&= -\frac{\alpha}{h}\\
            		\tensor{\Lambda}{_2_3_3} &= - \frac{1}{h} \pdv{h}{y} \\
            		&= - \frac{\beta}{h}
            		\end{align*}
            		所以所有非零项为
            		\begin{align*}
            		\tensor{\Lambda}{_0_3_3} = - \tensor{\Lambda}{_3_3_0} &= - \frac{1}{h}\\
            		\tensor{\Lambda}{_1_3_3} = - \tensor{\Lambda}{_3_3_1} &= - \frac{\alpha}{h}\\
            		\tensor{\Lambda}{_2_3_3} = - \tensor{\Lambda}{_3_3_2} &= - \frac{\beta}{h}
            		\end{align*}
            		由
            		\begin{displaymath}
            		\tensor{\omega}{_\mu_\nu_\rho} = \frac{1}{2} \left( \tensor{\Lambda}{_\mu_\nu_\rho} + \tensor{\Lambda}{_\rho_\mu_\nu} - \tensor{\Lambda}{_\nu_\rho_\mu} \right)
            		\end{displaymath}
            		知
            		\begin{displaymath}
            		\tensor{\omega}{_\mu_3_3} = \tensor{\Lambda}{_\mu_3_3}
            		\end{displaymath}
            		于是
            		\begin{align*}
            		\tensor{\bm{\omega}}{_0_3} = - \tensor{\bm{\omega}}{_3_0} &= - \frac{1}{h} \bm{e}^3 = - \dd{z}\\
            		\tensor{\bm{\omega}}{_1_3} = - \tensor{\bm{\omega}}{_3_1} &= - \frac{\alpha}{h} \bm{e}^3 = - \alpha \dd{z}\\
            		\tensor{\bm{\omega}}{_2_3} = - \tensor{\bm{\omega}}{_3_2} &= - \frac{\beta}{h} \bm{e}^3 = - \beta \dd{z}
            		\end{align*}
            		升指标得
            		\begin{align*}
            		\tensor{\bm{\omega}}{_0^3} = \tensor{\bm{\omega}}{_3^0} &= - \frac{1}{h} \bm{e}^3 = - \dd{z}\\
            		\tensor{\bm{\omega}}{_1^3} = - \tensor{\bm{\omega}}{_3^1} &= - \frac{\alpha}{h} \bm{e}^3 = - \alpha \dd{z}\\
            		\tensor{\bm{\omega}}{_2^3} = - \tensor{\bm{\omega}}{_3^2} &= - \frac{\beta}{h} \bm{e}^3 = - \beta \dd{z}
            		\end{align*}
            		\item 求曲率2形式。先求联络1形式的外微分:
            		\begin{align*}
            		\dd{\tensor{\bm{\omega}}{_0^3}} &= \dd{\tensor{\bm{\omega}}{_3^0}} = 0\\
            		\dd{\tensor{\bm{\omega}}{_1^3}} &= - \dd{\tensor{\bm{\omega}}{_3^1}} = 0\\
            		\dd{\tensor{\bm{\omega}}{_2^3}} &= \dd{\tensor{\bm{\omega}}{_3^2}} = 0
            		\end{align*}
            		而所有的联络1形式正比于 $\dd{z}$ ,它们的楔积全为零,故所有的曲率2形式为零。
            		\item[{\heiti 验证}] 曲率为零的结果与第三章相同。
            	\end{enumerate}
            \end{yl}
		\end{enumerate}
	\end{jie}
\end{xiti}