% !TeX root = ../document.tex

\chapter{施瓦西时空}

\begin{xiti}
	\item 考虑 Taub 的平面对称时空,其线元为式 (8-6-1')\footnote{正文(8-6-1')为
	\begin{equation*}
		\dd{s}^2 = z^{-1/2} \left( - \dd{t}^2 + \dd{z}^2 \right) + z \left( \dd{x}^2 + \dd{y}^2 \right). \tag{8-6-1'}
	\end{equation*}},试借助 Killing 矢量场写出类时测地线 $\gamma(\tau)$ 的参数表达式 $t(\tau)$, $x(\tau)$, $y(\tau)$, $z(\tau)$ 所满足的解耦方程 (参考 \S 9.1)。

	\begin{jie}
		对类时测地线,有
		\begin{equation}
			\begin{split}
				-1 &= \tensor{g}{_a_b} \tensor{\left( \pdv{\tau} \right)}{^a} \tensor{\left( \pdv{\tau} \right)}{^b}\\
				&= - z^{-1/2} \left( \dv{t}{\tau} \right)^2 - z^{-1/2} \left( \dv{z}{\tau} \right)^2 + z \left( \dv{x}{\tau} \right)^2 + z \left( \dv{y}{\tau} \right)^2,\label{eq-9--1}
			\end{split}
		\end{equation}
		由于有 $\tensor{{\xi_0}}{^a} = \tensor{\left( \pdv*{t} \right)}{^a}$、$\tensor{{\xi_1}}{^a} = \tensor{\left( \pdv*{x} \right)}{^a}$、$\tensor{{\xi_2}}{^a} = \tensor{\left( \pdv*{y} \right)}{^a}$、$\tensor{{\xi_3}}{^a} = - y \tensor{\left( \pdv*{x} \right)}{^a} + x \tensor{\left( \pdv*{y} \right)}{^a}$ 四个 Killing 矢量场,故有以下守恒量
		\begin{align*}
			E :={}& - \tensor{g}{_a_b} \tensor{\left( \pdv{\tau} \right)}{^a} \tensor{\left( \pdv{t} \right)}{^b}\\
			={} & z^{-1/2} \dv{t}{\tau},\\
			P_1 :={} & \tensor{g}{_a_b} \tensor{\left( \pdv{\tau} \right)}{^a} \tensor{\left( \pdv{x} \right)}{^b}\\
			={} & z \dv{x}{\tau},\\
			P_2 :={} & \tensor{g}{_a_b} \tensor{\left( \pdv{\tau} \right)}{^a} \tensor{\left( \pdv{y} \right)}{^b}\\
			={} & z \dv{y}{\tau},\\
			L := {} & \tensor{g}{_a_b} \tensor{\left( \pdv{\tau} \right)}{^a} \tensor{{\xi_3}}{^a}\\
			={}& z \left( -y \dv{x}{\tau} + z \dv{y}{\tau} \right),
		\end{align*}
		则
		\begin{equation}
			\begin{split}
				\dv{t}{\tau} &= \sqrt{z} E,\\
				\dv{x}{\tau} &= \frac{1}{z} P_1,\\
				\dv{y}{\tau} &= \frac{1}{z} P_2,
			\end{split}\label{eq-9-eq_txy}
		\end{equation}
		代入~\eqref{eq-9--1} 知
		\begin{equation*}
			-1 = - \sqrt{z} E^2 + \frac{1}{z} \left( P_1^2 + P_2^2 \right) - \frac{1}{\sqrt{z}} \left( \dv{z}{\tau} \right)^2,
		\end{equation*}
		由此式可解得 $z(\tau)$,再代入~\eqref{eq-9-eq_txy} 即可解得 $t(\tau)$、$x(\tau)$、$y(\tau)$。
		\begin{tcolorbox}[breakable,title=补充,fonttitle=\normalfont\bfseries]
			可以仿照命题 9-1-1 ,通过 $\mathrm{SE}(2)$ 对称性使得 $\left. y \right|_{\tau=0} = \left. \dv*{y}{\tau} \right|_{\tau=0} = 0$,则整条测地线上有 $y=0$。不过这里 $x$、$y$ 自动是解耦的,因此无需费这番口舌。
		\end{tcolorbox}
	\end{jie}

	\item 用牛顿引力论借图 9-8 直接推出式 (9-3-18)。

	\begin{figure}[!htbp]
		\centering
		\begin{tikzpicture}[scale=1.3]
			\draw[semithick] (0,0) circle (2);
			\draw[semithick] (0,0) circle (1.3);
			\draw[semithick] (0,0) circle (1.5);
			\filldraw (0,0) circle (1pt);
			\draw[semithick] (0:1.3) -- (0:1.5);
			\draw[semithick] (12:1.3) -- (12:1.5);
			\draw[thick,-{Latex[length=2pt 6,width'=0pt 0.4]}] (6:1) -- (6:1.3);
			\draw[thick,-{Latex[length=1pt 6,width'=0pt 0.4]}] (6:1.7) -- (6:1.5);
			\node[below left] (O) at (0,0) {$O$};
			\node[above] (p) at (6:1) {$p$};
			\draw (8:1.65) -- (1.8,1.6);
			\node[above] (dp) at (1.9,1.6) {$p+\dd{p}<p$};
			\draw (4:1.43) -- (2.2,-0.4);
			\node[below right] (dV) at (2.2,-0.4) {$\dd{V}$};
		\end{tikzpicture}
		\caption{正文图 9-8}\label{pic-9-9-8}
	\end{figure}

	\begin{jie}
		如图~\ref{pic-9-9-8},取厚度 $\dd{r}$ 的球壳上截面积为 $\dd{S}$ 的体积元 $\dd{V} = \dd{r} \dd{S}$,则由受力平衡知
		\begin{gather*}
			p \dd{S} = (p+\dd{p}) \dd{S} + \frac{1}{r^2} m(r) \rho \dd{V},\\
			\implies \dd{p} \dd{S} = - \frac{1}{r^2} m(r) \rho \dd{V},\\
			\implies \dv{p}{r} = - \frac{1}{r^2} m(r) \rho.
		\end{gather*}
	\end{jie}
\end{xiti}
