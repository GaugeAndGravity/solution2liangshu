% !TeX root = ../document.tex

\chapter{施瓦西时空}

\begin{xiti}
	\item 考虑 Taub 的平面对称时空,其线元为式 (8-6-1')\footnote{正文(8-6-1')为
	\begin{equation*}
		\dd{s}^2 = z^{-1/2} \left( - \dd{t}^2 + \dd{z}^2 \right) + z \left( \dd{x}^2 + \dd{y}^2 \right). \tag{8-6-1'}
	\end{equation*}},试借助 Killing 矢量场写出类时测地线 $\gamma(\tau)$ 的参数表达式 $t(\tau)$, $x(\tau)$, $y(\tau)$, $z(\tau)$ 所满足的解耦方程 (参考 \S 9.1)。

	\begin{jie}
		对类时测地线,有
		\begin{equation}
			\begin{split}
				-1 &= \tensor{g}{_a_b} \tensor{\left( \pdv{\tau} \right)}{^a} \tensor{\left( \pdv{\tau} \right)}{^b}\\
				&= - z^{-1/2} \left( \dv{t}{\tau} \right)^2 - z^{-1/2} \left( \dv{z}{\tau} \right)^2 + z \left( \dv{x}{\tau} \right)^2 + z \left( \dv{y}{\tau} \right)^2,\label{eq-9--1}
			\end{split}
		\end{equation}
		由于有 $\tensor{{\xi_0}}{^a} = \tensor{\left( \pdv*{t} \right)}{^a}$、$\tensor{{\xi_1}}{^a} = \tensor{\left( \pdv*{x} \right)}{^a}$、$\tensor{{\xi_2}}{^a} = \tensor{\left( \pdv*{y} \right)}{^a}$、$\tensor{{\xi_3}}{^a} = - y \tensor{\left( \pdv*{x} \right)}{^a} + x \tensor{\left( \pdv*{y} \right)}{^a}$ 四个 Killing 矢量场,故有以下守恒量
		\begin{align*}
			E :={}& - \tensor{g}{_a_b} \tensor{\left( \pdv{\tau} \right)}{^a} \tensor{\left( \pdv{t} \right)}{^b}\\
			={} & z^{-1/2} \dv{t}{\tau},\\
			P_1 :={} & \tensor{g}{_a_b} \tensor{\left( \pdv{\tau} \right)}{^a} \tensor{\left( \pdv{x} \right)}{^b}\\
			={} & z \dv{x}{\tau},\\
			P_2 :={} & \tensor{g}{_a_b} \tensor{\left( \pdv{\tau} \right)}{^a} \tensor{\left( \pdv{y} \right)}{^b}\\
			={} & z \dv{y}{\tau},\\
			L := {} & \tensor{g}{_a_b} \tensor{\left( \pdv{\tau} \right)}{^a} \tensor{{\xi_3}}{^a}\\
			={}& z \left( -y \dv{x}{\tau} + z \dv{y}{\tau} \right),
		\end{align*}
		则
		\begin{equation}
			\begin{split}
				\dv{t}{\tau} &= \sqrt{z} E,\\
				\dv{x}{\tau} &= \frac{1}{z} P_1,\\
				\dv{y}{\tau} &= \frac{1}{z} P_2,
			\end{split}\label{eq-9-eq_txy}
		\end{equation}
		代入~\eqref{eq-9--1} 知
		\begin{equation*}
			-1 = - \sqrt{z} E^2 + \frac{1}{z} \left( P_1^2 + P_2^2 \right) - \frac{1}{\sqrt{z}} \left( \dv{z}{\tau} \right)^2,
		\end{equation*}
		由此式可解得 $z(\tau)$,再代入~\eqref{eq-9-eq_txy} 即可解得 $t(\tau)$、$x(\tau)$、$y(\tau)$。
		\begin{tcolorbox}[breakable,title=补充,fonttitle=\normalfont\bfseries]
			可以仿照命题 9-1-1 ,通过 $\mathrm{SE}(2)$ 对称性使得 $\left. y \right|_{\tau=0} = \left. \dv*{y}{\tau} \right|_{\tau=0} = 0$,则整条测地线上有 $y=0$。不过这里 $x$、$y$ 自动是解耦的,因此无需费这番口舌。
		\end{tcolorbox}
	\end{jie}

	\item 用牛顿引力论借图 9-8 直接推出式 (9-3-18)。

	\begin{figure}[!htbp]
		\centering
		\begin{tikzpicture}[scale=1.3]
			\draw[semithick] (0,0) circle (2);
			\draw[semithick] (0,0) circle (1.3);
			\draw[semithick] (0,0) circle (1.5);
			\filldraw (0,0) circle (1pt);
			\draw[semithick] (0:1.3) -- (0:1.5);
			\draw[semithick] (12:1.3) -- (12:1.5);
			\draw[thick,-{Latex[length=2pt 6,width'=0pt 0.4]}] (6:1) -- (6:1.3);
			\draw[thick,-{Latex[length=1pt 6,width'=0pt 0.4]}] (6:1.7) -- (6:1.5);
			\node[below left] (O) at (0,0) {$O$};
			\node[above] (p) at (6:1) {$p$};
			\draw (8:1.65) -- (1.8,1.6);
			\node[above] (dp) at (1.9,1.6) {$p+\dd{p}<p$};
			\draw (4:1.43) -- (2.2,-0.4);
			\node[below right] (dV) at (2.2,-0.4) {$\dd{V}$};
		\end{tikzpicture}
		\caption{正文图 9-8}\label{pic-9-9-8}
	\end{figure}

	\begin{jie}
		如图~\ref{pic-9-9-8},取厚度 $\dd{r}$ 的球壳上截面积为 $\dd{S}$ 的体积元 $\dd{V} = \dd{r} \dd{S}$,则由受力平衡知
		\begin{gather*}
			p \dd{S} = (p+\dd{p}) \dd{S} + \frac{1}{r^2} m(r) \rho \dd{V},\\
			\implies \dd{p} \dd{S} = - \frac{1}{r^2} m(r) \rho \dd{V},\\
			\implies \dv{p}{r} = - \frac{1}{r^2} m(r) \rho.
		\end{gather*}
	\end{jie}

	\item 试证 OV 流体静力学平衡方程可改写为
	\begin{equation*}
		\left[ 1- \frac{2m(r)}{r} \right]^{1/2} \dv{p}{r} = - \left( \rho + p \right) g, \tag{9-4-60}
	\end{equation*}
	其中 $g$ 代表流体质点的 4 加速 $\tensor{U}{^b} \Nabla{b} \tensor{U}{^a}$ 的大小。

	\begin{yl}{注}
		在牛顿近似下 $\left[ 1 - 2 m(r) / r \right]^{1/2} \cong 1$, $p \cong 0$, 式 (9-4-60) 成为 $\dv*{p}{r} \cong - \rho g$。而 $g \cong m(r)/r^2$,故得式 (9-3-18),即 $\dv*{p}{r} \cong - \rho m(r)/r^2$。
	\end{yl}

	\begin{zm}
		由于 $\tensor{U}{^a} = \e{-A} \tensor{\left( \pdv*{t} \right)}{^a}$,有
		\begin{align*}
			\tensor{U}{^b} \tensor{\nabla}{_b} \tensor{U}{^a} &= \tensor{U}{^b} \tensor{\left( \pdv{t} \right)}{^a} \Nabla{b} \e{-A} + \e{-2A} \tensor{\left( \pdv{t} \right)}{^b} \Nabla{b} \tensor{\left( \pdv{t} \right)}{^a}\\
			&= 0 + \e{-2A} \ChristoffelSymbol{\mu}{0}{0} \tensor{\left( \pdv{x^\mu} \right)}{^a}\\
			&= \dv{A}{r} \e{-2B} \tensor{\left( \pdv{r} \right)}{^a}\\
			&= \left( 1- \frac{2m(r)}{r} \right) \dv{A}{r} \tensor{\left( \pdv{r} \right)}{^a}\\
			&= \frac{r - 2m(r)}{r} \frac{m(r) + 4 \pi p r^3}{r \left( r - 2 m(r) \right)} \tensor{\left( \pdv{r} \right)}{^a}\\
			&= \frac{m(r) + 4 \pi p r^3}{r^2} \tensor{\left( \pdv{r} \right)}{^a},
		\end{align*}
		故
		\begin{align*}
			g &= \sqrt{\tensor{g}{_1_1} \left( \frac{m(r) + 4 \pi p r^3}{r^2} \right)^2}\\
			&= \left( 1 - \frac{2 m(r)}{r} \right)^{-1/2} \frac{m(r) + 4 \pi p r^3}{r^2},
		\end{align*}
		于是
		\begin{align*}
			\left( 1 - \frac{2 m(r)}{r} \right)^{1/2} \dv{p}{r} &= - \left( 1 - \frac{2 m(r)}{r} \right)^{1/2} \left( \rho + p \right) \frac{m(r) + 4 \pi p r^3}{r^2 \left( 1 - \frac{2 m(r)}{r} \right)}\\
			&= - \left( 1 - \frac{2 m(r)}{r} \right)^{-1/2} \left( \rho + p \right) \frac{m(r) + 4 \pi p r^3}{r^2}\\
			&= - \left( \rho + p \right) g.
		\end{align*}
	\end{zm}

	\item 试证当 $R \gg M$ 时式 (9-3-26) 近似回到牛顿引力论的式 (9-3-23)。

	\begin{zm}
		当 $M/R \rightarrow 0$ 时,
		\begin{align*}
			p_0 &= \rho \frac{1- \left( 1 - 2 M/R \right)^{1/2}}{3\left( 1- 2 M/R \right)^{1/2} - 1}\\
			&\sim \rho \frac{1 - \left( 1 - M/R \right)}{2}\\
			&= \frac{\rho}{2 R} M\\
			&= \frac{\rho}{2 R} \frac{4}{3} \pi R^3 \rho\\
			&= \frac{2}{3} \pi \rho^2 R^2.
		\end{align*}
	\end{zm}

	\item 求闵氏时空中 Rindler 坐标 $t,x$ 与洛伦兹坐标 $T,X$ 的关系。

	\begin{jie}
		\begin{align*}
			T &= \frac{U+V}{2}\\
			&= \frac{\e{v} - \e{-u}}{2}\\
			&= \frac{x \e{t} - x \e{-t}}{2}\\
			&= x \sinh{t},\\
			X &= \frac{V - U}{2}\\
			&= \frac{\e{v} + \e{-u}}{2}\\
			&= \frac{x \e{t} + x\e{-t}}{2}\\
			&= x \cosh{t},
		\end{align*}
		反解得
		\begin{align*}
			t &= \tanh^{-1} \frac{T}{X},\\
			&= \frac{1}{2} \ln \frac{X + T}{X - T},\\
			x &= \sqrt{X^2 - T^2}.
		\end{align*}
	\end{jie}

	\item Rindler 时空的类时 Killing 矢量场 $\tensor{\left( \pdv*{t} \right)}{^a}$ 是闵氏时空的哪个 Killing 矢量场?

	\begin{jie}
		由上题得
		\begin{align*}
			\tensor{\left( \pdv{t} \right)}{^a} &= \pdv{T}{t} \tensor{\left( \pdv{T} \right)}{^a} + \pdv{X}{t} \tensor{\left( \pdv{X} \right)}{^a}\\
			&= x \cosh{t} \tensor{\left( \pdv{T} \right)}{^a} + x \sinh{t} \tensor{\left( \pdv{X} \right)}{^a}\\
			&= X \tensor{\left( \pdv{T} \right)}{^a} + T \tensor{\left( \pdv{X} \right)}{^a},
		\end{align*}
		这是 boost 矢量场。事实上,这个矢量场在正文 (4-3-3) 出现过了。
	\end{jie}

	\item 求施瓦西时空中静态观者的 4 加速的长度 $A = \left( \tensor{A}{^a} \tensor{A}{_a} \right)^{1/2}$ 。提示:可借用\hyperlink{prob-8.3}{第 8 章习题 3} 的结论,即 $\tensor{A}{_a} = \Nabla{a} \ln{\chi}$。

	\begin{jie}
		对施瓦西时空,$\tensor{\xi}{^a} = \tensor{\left( \pdv*{t} \right)}{^a}$,
		\begin{equation*}
			\chi = \sqrt{- \tensor{\xi}{_a} \tensor{\xi}{^a}} = \sqrt{1 - \frac{2M}{r}},
		\end{equation*}
		知
		\begin{align*}
			\tensor{A}{_a} &= \Nabla{a} \ln{\chi}\\
			&= \frac{1}{2} \Nabla{a} \ln(1 - \frac{2M}{r})\\
			&= \frac{1}{2} \left( 1 - \frac{2M}{r} \right)^{-1} \Nabla{a} \left( 1 - \frac{2M}{r} \right)\\
			&= \left( 1 - \frac{2M}{r} \right)^{-1} \frac{M}{r^2} \tensor{\left( \dd{r} \right)}{_a},
		\end{align*}
		故
		\begin{align*}
			A &= \sqrt{\tensor{g}{^a^b} \tensor{A}{_a} \tensor{A}{_b}}\\
			&= \left( 1 - \frac{M}{r^2} \right)^{-1} \frac{2M}{r} \sqrt{\tensor{g}{^1^1}}\\
			&= \left( 1 - \frac{M}{r^2} \right)^{-1/2} \frac{2M}{r}.
		\end{align*}
		吐槽:这不是 (8-3-23) 算过的么……
	\end{jie}
\end{xiti}
