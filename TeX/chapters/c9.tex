% !TeX root = ../document.tex

\chapter{施瓦西时空}

\begin{xiti}
	\item 考虑 Taub 的平面对称时空,其线元为式 (8-6-1')\footnote{正文(8-6-1')为
	\begin{equation*}
		\dd{s}^2 = z^{-1/2} \left( - \dd{t}^2 + \dd{z}^2 \right) + z \left( \dd{x}^2 + \dd{y}^2 \right). \tag{8-6-1'}
	\end{equation*}},试借助 Killing 矢量场写出类时测地线 $\gamma(\tau)$ 的参数表达式 $t(\tau)$, $x(\tau)$, $y(\tau)$, $z(\tau)$ 所满足的解耦方程 (参考 \S 9.1)。

	\begin{jie}
		对类时测地线,有
		\begin{equation}
			\begin{split}
				-1 &= \tensor{g}{_a_b} \tensor{\left( \pdv{\tau} \right)}{^a} \tensor{\left( \pdv{\tau} \right)}{^b}\\
				&= - z^{-1/2} \left( \dv{t}{\tau} \right)^2 - z^{-1/2} \left( \dv{z}{\tau} \right)^2 + z \left( \dv{x}{\tau} \right)^2 + z \left( \dv{y}{\tau} \right)^2,\label{eq-9--1}
			\end{split}
		\end{equation}
		由于有 $\tensor{{\xi_0}}{^a} = \tensor{\left( \pdv*{t} \right)}{^a}$、$\tensor{{\xi_1}}{^a} = \tensor{\left( \pdv*{x} \right)}{^a}$、$\tensor{{\xi_2}}{^a} = \tensor{\left( \pdv*{y} \right)}{^a}$、$\tensor{{\xi_3}}{^a} = - y \tensor{\left( \pdv*{x} \right)}{^a} + x \tensor{\left( \pdv*{y} \right)}{^a}$ 四个 Killing 矢量场,故有以下守恒量
		\begin{align*}
			E :={}& - \tensor{g}{_a_b} \tensor{\left( \pdv{\tau} \right)}{^a} \tensor{\left( \pdv{t} \right)}{^b}\\
			={} & z^{-1/2} \dv{t}{\tau},\\
			P_1 :={} & \tensor{g}{_a_b} \tensor{\left( \pdv{\tau} \right)}{^a} \tensor{\left( \pdv{x} \right)}{^b}\\
			={} & z \dv{x}{\tau},\\
			P_2 :={} & \tensor{g}{_a_b} \tensor{\left( \pdv{\tau} \right)}{^a} \tensor{\left( \pdv{y} \right)}{^b}\\
			={} & z \dv{y}{\tau},\\
			L := {} & \tensor{g}{_a_b} \tensor{\left( \pdv{\tau} \right)}{^a} \tensor{{\xi_3}}{^a}\\
			={}& z \left( -y \dv{x}{\tau} + z \dv{y}{\tau} \right),
		\end{align*}
		则
		\begin{equation}
			\begin{split}
				\dv{t}{\tau} &= \sqrt{z} E,\\
				\dv{x}{\tau} &= \frac{1}{z} P_1,\\
				\dv{y}{\tau} &= \frac{1}{z} P_2,
			\end{split}\label{eq-9-eq_txy}
		\end{equation}
		代入~\eqref{eq-9--1} 知
		\begin{equation*}
			-1 = - \sqrt{z} E^2 + \frac{1}{z} \left( P_1^2 + P_2^2 \right) - \frac{1}{\sqrt{z}} \left( \dv{z}{\tau} \right)^2,
		\end{equation*}
		由此式可解得 $z(\tau)$,再代入~\eqref{eq-9-eq_txy} 即可解得 $t(\tau)$、$x(\tau)$、$y(\tau)$。
		\begin{tcolorbox}[breakable,title=补充,fonttitle=\normalfont\bfseries]
			可以仿照命题 9-1-1 ,通过 $\mathrm{SE}(2)$ 对称性使得 $\left. y \right|_{\tau=0} = \left. \dv*{y}{\tau} \right|_{\tau=0} = 0$,则整条测地线上有 $y=0$。不过这里 $x$、$y$ 自动是解耦的,因此无需费这番口舌。
		\end{tcolorbox}
	\end{jie}

	\item 用牛顿引力论借图 9-8 直接推出式 (9-3-18)。

	\begin{figure}[!htbp]
		\centering
		\begin{tikzpicture}[scale=1.3]
			\draw[semithick] (0,0) circle (2);
			\draw[semithick] (0,0) circle (1.3);
			\draw[semithick] (0,0) circle (1.5);
			\filldraw (0,0) circle (1pt);
			\draw[semithick] (0:1.3) -- (0:1.5);
			\draw[semithick] (12:1.3) -- (12:1.5);
			\draw[thick,-{Latex[length=2pt 6,width'=0pt 0.4]}] (6:1) -- (6:1.3);
			\draw[thick,-{Latex[length=1pt 6,width'=0pt 0.4]}] (6:1.7) -- (6:1.5);
			\node[below left] (O) at (0,0) {$O$};
			\node[above] (p) at (6:1) {$p$};
			\draw (8:1.65) -- (1.8,1.6);
			\node[above] (dp) at (1.9,1.6) {$p+\dd{p}<p$};
			\draw (4:1.43) -- (2.2,-0.4);
			\node[below right] (dV) at (2.2,-0.4) {$\dd{V}$};
		\end{tikzpicture}
		\caption{正文图 9-8}\label{pic-9-9-8}
	\end{figure}

	\begin{jie}
		如图~\ref{pic-9-9-8},取厚度 $\dd{r}$ 的球壳上截面积为 $\dd{S}$ 的体积元 $\dd{V} = \dd{r} \dd{S}$,则由受力平衡知
		\begin{gather*}
			p \dd{S} = (p+\dd{p}) \dd{S} + \frac{1}{r^2} m(r) \rho \dd{V},\\
			\implies \dd{p} \dd{S} = - \frac{1}{r^2} m(r) \rho \dd{V},\\
			\implies \dv{p}{r} = - \frac{1}{r^2} m(r) \rho.
		\end{gather*}
	\end{jie}

	\item 试证 OV 流体静力学平衡方程可改写为
	\begin{equation*}
		\left[ 1- \frac{2m(r)}{r} \right]^{1/2} \dv{p}{r} = - \left( \rho + p \right) g, \tag{9-4-60}
	\end{equation*}
	其中 $g$ 代表流体质点的 4 加速 $\tensor{U}{^b} \Nabla{b} \tensor{U}{^a}$ 的大小。

	\begin{yl}{注}
		在牛顿近似下 $\left[ 1 - 2 m(r) / r \right]^{1/2} \cong 1$, $p \cong 0$, 式 (9-4-60) 成为 $\dv*{p}{r} \cong - \rho g$。而 $g \cong m(r)/r^2$,故得式 (9-3-18),即 $\dv*{p}{r} \cong - \rho m(r)/r^2$。
	\end{yl}

	\begin{zm}
		由于 $\tensor{U}{^a} = \e{-A} \tensor{\left( \pdv*{t} \right)}{^a}$,有
		\begin{align*}
			\tensor{U}{^b} \tensor{\nabla}{_b} \tensor{U}{^a} &= \tensor{U}{^b} \tensor{\left( \pdv{t} \right)}{^a} \Nabla{b} \e{-A} + \e{-2A} \tensor{\left( \pdv{t} \right)}{^b} \Nabla{b} \tensor{\left( \pdv{t} \right)}{^a}\\
			&= 0 + \e{-2A} \ChristoffelSymbol{\mu}{0}{0} \tensor{\left( \pdv{x^\mu} \right)}{^a}\\
			&= \dv{A}{r} \e{-2B} \tensor{\left( \pdv{r} \right)}{^a}\\
			&= \left( 1- \frac{2m(r)}{r} \right) \dv{A}{r} \tensor{\left( \pdv{r} \right)}{^a}\\
			&= \frac{r - 2m(r)}{r} \frac{m(r) + 4 \pi p r^3}{r \left( r - 2 m(r) \right)} \tensor{\left( \pdv{r} \right)}{^a}\\
			&= \frac{m(r) + 4 \pi p r^3}{r^2} \tensor{\left( \pdv{r} \right)}{^a},
		\end{align*}
		故
		\begin{align*}
			g &= \sqrt{\tensor{g}{_1_1} \left( \frac{m(r) + 4 \pi p r^3}{r^2} \right)^2}\\
			&= \left( 1 - \frac{2 m(r)}{r} \right)^{-1/2} \frac{m(r) + 4 \pi p r^3}{r^2},
		\end{align*}
		于是
		\begin{align*}
			\left( 1 - \frac{2 m(r)}{r} \right)^{1/2} \dv{p}{r} &= - \left( 1 - \frac{2 m(r)}{r} \right)^{1/2} \left( \rho + p \right) \frac{m(r) + 4 \pi p r^3}{r^2 \left( 1 - \frac{2 m(r)}{r} \right)}\\
			&= - \left( 1 - \frac{2 m(r)}{r} \right)^{-1/2} \left( \rho + p \right) \frac{m(r) + 4 \pi p r^3}{r^2}\\
			&= - \left( \rho + p \right) g.
		\end{align*}
	\end{zm}

	\item 试证当 $R \gg M$ 时式 (9-3-26) 近似回到牛顿引力论的式 (9-3-23)。

	\begin{zm}
		当 $M/R \rightarrow 0$ 时,
		\begin{align*}
			p_0 &= \rho \frac{1- \left( 1 - 2 M/R \right)^{1/2}}{3\left( 1- 2 M/R \right)^{1/2} - 1}\\
			&\sim \rho \frac{1 - \left( 1 - M/R \right)}{2}\\
			&= \frac{\rho}{2 R} M\\
			&= \frac{\rho}{2 R} \frac{4}{3} \pi R^3 \rho\\
			&= \frac{2}{3} \pi \rho^2 R^2.
		\end{align*}
	\end{zm}

	\item 求闵氏时空中 Rindler 坐标 $t,x$ 与洛伦兹坐标 $T,X$ 的关系。

	\begin{jie}
		\begin{align*}
			T &= \frac{U+V}{2}\\
			&= \frac{\e{v} - \e{-u}}{2}\\
			&= \frac{x \e{t} - x \e{-t}}{2}\\
			&= x \sinh{t},\\
			X &= \frac{V - U}{2}\\
			&= \frac{\e{v} + \e{-u}}{2}\\
			&= \frac{x \e{t} + x\e{-t}}{2}\\
			&= x \cosh{t},
		\end{align*}
		反解得
		\begin{align*}
			t &= \tanh^{-1} \frac{T}{X},\\
			&= \frac{1}{2} \ln \frac{X + T}{X - T},\\
			x &= \sqrt{X^2 - T^2}.
		\end{align*}
	\end{jie}

	\item Rindler 时空的类时 Killing 矢量场 $\tensor{\left( \pdv*{t} \right)}{^a}$ 是闵氏时空的哪个 Killing 矢量场?

	\begin{jie}
		由上题得
		\begin{align*}
			\tensor{\left( \pdv{t} \right)}{^a} &= \pdv{T}{t} \tensor{\left( \pdv{T} \right)}{^a} + \pdv{X}{t} \tensor{\left( \pdv{X} \right)}{^a}\\
			&= x \cosh{t} \tensor{\left( \pdv{T} \right)}{^a} + x \sinh{t} \tensor{\left( \pdv{X} \right)}{^a}\\
			&= X \tensor{\left( \pdv{T} \right)}{^a} + T \tensor{\left( \pdv{X} \right)}{^a},
		\end{align*}
		这是 boost 矢量场。事实上,这个矢量场在正文 (4-3-3) 出现过了。
	\end{jie}

	\item 求施瓦西时空中静态观者的 4 加速的长度 $A = \left( \tensor{A}{^a} \tensor{A}{_a} \right)^{1/2}$ 。提示:可借用\hyperlink{prob-8.3}{第 8 章习题 3} 的结论,即 $\tensor{A}{_a} = \Nabla{a} \ln{\chi}$。

	\begin{jie}
		对施瓦西时空,$\tensor{\xi}{^a} = \tensor{\left( \pdv*{t} \right)}{^a}$,
		\begin{equation*}
			\chi = \sqrt{- \tensor{\xi}{_a} \tensor{\xi}{^a}} = \sqrt{1 - \frac{2M}{r}},
		\end{equation*}
		知
		\begin{align*}
			\tensor{A}{_a} &= \Nabla{a} \ln{\chi}\\
			&= \frac{1}{2} \Nabla{a} \ln(1 - \frac{2M}{r})\\
			&= \frac{1}{2} \left( 1 - \frac{2M}{r} \right)^{-1} \Nabla{a} \left( 1 - \frac{2M}{r} \right)\\
			&= \left( 1 - \frac{2M}{r} \right)^{-1} \frac{M}{r^2} \tensor{\left( \dd{r} \right)}{_a},
		\end{align*}
		故
		\begin{align*}
			A &= \sqrt{\tensor{g}{^a^b} \tensor{A}{_a} \tensor{A}{_b}}\\
			&= \left( 1 - \frac{M}{r^2} \right)^{-1} \frac{2M}{r} \sqrt{\tensor{g}{^1^1}}\\
			&= \left( 1 - \frac{M}{r^2} \right)^{-1/2} \frac{2M}{r}.
		\end{align*}
		吐槽:这不是 (8-3-23) 算过的么……
	\end{jie}

	\item 把图 9-13(a) 的 $N_1$ (或 $N_2$) 所代表的径向类光测地线简称为 $N_1$ (或 $N_2$) ,试证:(1) 坐标 $V$ (或 $U$ )是类光测地线 $N_1$ (或 $N_2$) 的仿射参数;(2) 坐标 $r$ 是除 $N_1$ 和 $N_2$ 外的径向类光测地线的仿射参数。

	\begin{figure}[htbp!]
		\centering
		\begin{tikzpicture}[decoration={snake,segment length=5,amplitude=1}, semithick]
			\draw[\myarrow, very thick] (-4,0) -- (4,0);
			\draw[\myarrow, very thick] (0,-4) -- (0,4);
			\draw[\myarrow, very thick] (-2.828427,-2.828427) -- (2.828427, 2.828427);
			\draw[\myarrow, very thick] (2.828427, -2.828427) -- (-2.828427, 2.828427);
			% \draw[domain=-1:1,variable=\t] plot ({2*cosh(\t)},{2*sinh(\t)});
			\draw[domain=-1.1:1.1,variable=\t,decorate] plot ({1.7*sinh(\t)},{1.7*cosh(\t)});
			\draw[domain=-1.1:1.1,variable=\t,decorate] plot ({1.7*sinh(\t)},{-1.7*cosh(\t)});
			% \draw[domain=-1.3:1.3,variable=\t] plot ({1*sinh(\t)},{1*cosh(\t)});
			\node[right] (X) at (4, 0) {$X$};
			\node[left] (T) at (0, 4) {$T$};
			\node[above left] (U) at (-2.828427, 2.828427) {$U$};
			\node[above right] (V) at (2.828427, 2.828427) {$V$};
			\node (A) at (2,0) {\colorbox{white}{$\mathrm{A}$}};
			\node (Ap) at (-2, 0) {\colorbox{white}{$\mathrm{A'}$}};
			\node (B) at (0,1) {\colorbox{white}{$\mathrm{B}$}};
			\node (W) at (0, -1) {\colorbox{white}{$\mathrm{W}$}};
			\node (N1p) at (1.4, 1.4) {\colorbox{white}{$N_1^+$}};
			\node (N1m) at (-1.4, -1.4) {\colorbox{white}{$N_1^-$}};
			\node (N2p) at (-1.4, 1.4) {\colorbox{white}{$N_2^+$}};
			\node (N2m) at (1.4, -1.4) {\colorbox{white}{$N_2^-$}};
			\draw (2,2) -- (2.5,1.7);
			\node[right] (U0) at (2.4,1.7) {\colorbox{white}{$U=0$}};
			\draw (2,-2) -- (2.5,-1.7);
			\node[right] (V0) at (2.4,-1.7) {\colorbox{white}{$V=0$}};
			\draw (0,0) -- (-1.5, -0.75);
			\node[below] (UV0) at (-1.9, -0.55) {\colorbox{white}{\small $V=U=0$}};
			\draw (0.7, 1.83848) -- (0.6, 2.1);
			\node[above] (r1) at (0.7, 2.05) {\footnotesize $r=0$};
			\draw (0.7, -1.83848) -- (0.6, -2.1);
			\node[below] (r2) at (0.7, -2.05) {\footnotesize $r=0$};
		\end{tikzpicture}
		\caption{正文图 9-13(a)}
	\end{figure}
	\begin{zm}
		\begin{enumerate}[label = (\arabic*)]
			\item 对 $N_1$,
				\begin{align*}
					\tensorp{V}{^a} \Nabla{a} \tensorp{V}{^b} &= \frac{1}{4} \left( \tensorp{T}{^a} + \tensorp{X}{^a} \right) \Nabla{a} \left( \tensorp{T}{^b} + \tensorp{X}{^b} \right)\\
					&= \frac{1}{4} \left( \ChristoffelSymbol{\mu}{T}{T} + 2 \ChristoffelSymbol{\mu}{T}{X} + \ChristoffelSymbol{\mu}{X}{X} \right) \tensorp{x^\mu}{^b},
				\end{align*}
				显然 $x^\mu$ 只能取 $T$, $X$ 而不能取 $\theta, \varphi$。易从 Kruskal 度规算得\footnote{如果懒得手搓,请参考第三章习题解答中的 Mathematica 代码}:
				\begin{align*}
					\ChristoffelSymbol{T}{T}{T} &= - \frac{r + 2M}{4Mr} \pdv{r}{T}, & \ChristoffelSymbol{T}{T}{X} &= - \frac{r + 2M}{4Mr} \pdv{r}{X}, & \ChristoffelSymbol{T}{X}{X} &= - \frac{r + 2M}{4Mr} \pdv{r}{T},\\
					\ChristoffelSymbol{X}{T}{T} &= - \frac{r + 2M}{4Mr} \pdv{r}{X}, & \ChristoffelSymbol{X}{T}{X} &= - \frac{r + 2M}{4Mr} \pdv{r}{T}, & \ChristoffelSymbol{X}{X}{X} &= - \frac{r + 2M}{4Mr} \pdv{r}{X},
				\end{align*}
				故
				\begin{align*}
					\tensorp{V}{^a} \Nabla{a} \tensorp{V}{^b} &= -\frac{r + 2M}{8Mr} \left( \pdv{r}{T} + \pdv{r}{X} \right) \left( \tensorp{T}{^b} + \tensorp{X}{^b} \right)\\
					&= - \frac{r + 2M}{4Mr} \left( \pdv{r}{T} + \pdv{r}{X} \right) \tensorp{V}{^a},
				\end{align*}
				而 $r = f(X^2 - T^2)$,故
				\begin{align*}
					\pdv{r}{T} + \pdv{r}{X} &= -2 T f' + 2X f' = - 2 U f'
				\end{align*}
				在 $N_1$ 上为零,故得证。同理可知 $U$ 是 $N_2$ 的仿射参数。

			\item 其余径向类光测地线在 Kruskal 坐标图示下也必须为倾斜 $\frac{\pi}{2}$ 的直线,即 $U$ 坐标线或 $V$ 坐标线。不妨考虑直线 $U = U_0$,其中 $U_0 \neq 0$,则以 $\lambda=r$ 为参数\footnote{这里多设一个参数 $\lambda$ 而不是直接写作 $r$ 是为避免测地线切矢的记号与 $r$ 坐标基矢引起混淆。},由于 $r = f(X^2 - T^2) = f(UV)$,其中 $f$ 的逆是
			\begin{equation*}
				f^{-1}(r) = \frac{r-2M}{2M} \e{r/2M},
			\end{equation*}
			则在测地线上有
			\begin{equation*}
				V = \frac{1}{U_0} f^{-1}(r),
			\end{equation*}
			切矢为
				\begin{align*}
					\tensorp{\lambda}{^a} &= \dv{V}{\lambda} \tensorp{V}{^a}\\
					&= \frac{r}{4M^2} \e{r/2M} \tensorp{V}{^a},
				\end{align*}
				故在曲线上,
				\begin{align*}
					\tensorp{\lambda}{^a} \Nabla{a} \tensorp{\lambda}{^b} &= \dv[2]{V}{\lambda} \tensorp{V}{^b} + \left( \dv{V}{\lambda} \right)^2 \tensorp{V}{^a} \Nabla{a} \tensorp{V}{^b}\\
					&= \left( \frac{r+2M}{8M^3} \e{r/2M} - \left( \dv{V}{\lambda} \right)^2 \frac{r+2M}{4Mr} \left( \pdv{r}{T} + \pdv{r}{X} \right) \right) \tensorp{V}{^b}\\
					&= \left( \frac{r+2M}{8M^3} \e{r/2M} - \left( \dv{V}{\lambda} \right)^2 \frac{r+2M}{2Mr} \dv{\lambda}{V} \right) \tensorp{V}{^b}\\
					&= \left( \frac{r+2M}{8M^3} \e{r/2M} - \dv{V}{\lambda} \frac{r+2M}{2Mr} \right) \tensorp{V}{^b}\\
					&= \left( \frac{r+2M}{8M^3} \e{r/2M} - \frac{r}{4M^2} \e{r/2M} \times \frac{r+2M}{2 Mr} \right) \tensorp{V}{^b}\\
					&= 0.
				\end{align*}
				若为测地线 $V = V_0$,同理可证。
		\end{enumerate}
	\end{zm}

	\item 引入与 Kruskal 坐标类似的坐标消除下列线元的坐标奇性 $r = R$:
	\begin{equation*}
		\dd s^2 = - \left( 1 - r^2/R^2 \right) \dd t^2 + \left( 1 - r^2/R^2 \right)^{-1} \dd{r}^2 + r^2 \left( \dd{\theta}^2 + \sin^2 \theta \dd{\varphi}^2 \right) \qc R=\text{常数}.
	\end{equation*}
	\begin{jie}
		只考虑前两维,并取 $0<r<R$,
		\begin{equation*}
			\dd \hat{s}^2 = - \left( 1 - r^2/R^2 \right) \left( \dd t^2 + \left( 1 - r^2/R^2 \right)^{-2} \dd{r}^2 \right),
		\end{equation*}
		令
		\begin{equation*}
			r_* = \int_0^r \left( 1 - \frac{{r'}^2}{R^2} \right)^{-1} \dd{r'} = R \arctanh\left(\frac{r}{R}\right) \in (0, \infty),
		\end{equation*}
		则
		\begin{equation*}
			\dd{\hat{s}}^2 = \left( 1 - r^2/R^2 \right) \left( - \dd t^2 + \dd{r_*}^2 \right),
		\end{equation*}
		再令
		\begin{equation*}
			u := t - r_* \qc v := t + r_*,
		\end{equation*}
		其取值范围为
		\begin{equation*}
			v > u,
		\end{equation*}
		得
		\begin{equation*}
			\dd \hat{s}^2 = - \left( 1 - \frac{r^2}{R^2} \right) \dd{u} \dd{v},
		\end{equation*}
		令
		\begin{equation*}
			U := \e{-\beta u} \qc V := \e{\beta v},
		\end{equation*}
		其中 $\beta$ 是待定常数,则取值范围是
		\begin{equation*}
			U,V > 0 \qc UV > 1,
		\end{equation*}
		且
		\begin{align*}
			\dd \hat{s}^2 &= - \beta^{-2} \left( 1 - \frac{r^2}{R^2} \right) \e{\beta(u-v)} \dd{U}\dd{V}\\
			&= - \beta^{-2} \left( 1 - \frac{r^2}{R^2} \right) \left( \frac{R+r}{R-r} \right)^{-\beta R} \dd{U} \dd{V},
		\end{align*}
		若选择 $\beta = -1/R$,则
		\begin{align*}
			\dd \hat{s}^2 = - \left( R+r \right)^2 \dd{U} \dd{V},
		\end{align*}
		坐标奇性已被消除,只需将 $r$ 视作函数
		\begin{equation*}
			r = \frac{1 - U V}{1 + U V} R,
		\end{equation*}
		即可允许 $U, V$ 除 $U V = - 1$ 外任意取值,则 $U V > -1$ 定义了一个连通的延拓。还可进一步令
		\begin{equation*}
			T := \frac{1}{2} \left( V + U \right) \qc X := \frac{1}{2} \left( V - U \right),
		\end{equation*}
		则得到完整的延拓后度规
		\begin{equation*}
			\dd s^2 = \left( R + r \right)^2 \left( - \dd{T}^2 + \dd{X}^2 \right) + r^2 \left( \dd \theta^2 + \sin^2 \theta \dd \varphi^2 \right),
		\end{equation*}
		其中
		\begin{equation*}
			r = \frac{1 - T^2 + X^2}{1 + T^2 - X^2} R.
		\end{equation*}
	\end{jie}

	\item 试证最大延拓施瓦西时空有 s.p. 曲率奇性。提示:利用式 (8-3-21)\footnote{正文(8-3-21)为
	\begin{equation}
		\left.
			\begin{gathered}
				\tensor{R}{_0_1_0_1} = -\frac{2M}{r^3} \qc \tensor{R}{_0_2_0_2} = \frac{M}{r} \left( 1 - 2M/r \right) \qc \tensor{R}{_0_3_0_3} = \frac{M}{r}\left( 1 - 2M/r \right) \sin^2\theta,\\
				\tensor{R}{_1_2_1_2} = - \frac{M}{r} \left( 1 - 2M/r \right)^{-1} \qc \tensor{R}{_1_3_1_3} = - \frac{M}{r} \left( 1 - 2M/r \right)^{-1} \sin^2 \theta \qc \tensor{R}{_2_3_2_3} = 2 M r \sin^2 \theta.
			\end{gathered}
		\right\}
		\tag{8-3-21}
	\end{equation}}。
	\begin{zm}
		由于 $B$ 区仍可使用施瓦西坐标,直接利用 (8-3-21) 得
		\begin{align*}
			\tensor{R}{^a^b^c^d} \tensor{R}{_a_b_c_d} &= \tensor{g}{^\mu^\alpha} \tensor{g}{^\nu^\beta} \tensor{g}{^\sigma^\gamma} \tensor{g}{^\rho^\delta} \tensor{R}{_\mu_\nu_\sigma_\rho} \tensor{R}{_\alpha_\beta_\gamma_\delta}\\
			&= \sum_{\mu,\nu,\sigma,\rho} \tensor{g}{^\mu^\mu} \tensor{g}{^\nu^\nu} \tensor{g}{^\sigma^\sigma} \tensor{g}{^\rho^\rho} \left( \tensor{R}{_\mu_\nu_\sigma_\rho} \right)^2\\
			&= \frac{4M^2}{r^6} + \frac{M^2}{r^6} + \frac{M^2}{r^6} + \frac{M^2}{r^6} + \frac{M^2}{r^6} + \frac{4M^2}{r^6}\\
			&= \frac{12 M}{r^6},
		\end{align*}
		故当不完备测地线(例如题8.8中以 $r$ 为仿射参数的径向类光测地线)趋于 $r=0$ 时,具有 s.p. 曲率奇性。
	\end{zm}

	\item 试证图 9-13(a) 的 $N_1$ 是类光超曲面。提示:只须证明其法矢 $\tensor{n}{^a}$ 类光。请注意 $N_1$ 的方程为 $U=0$,其法余矢为 $\tensor{n}{_a} = \Nabla{a} U$。
	\begin{zm}
		\begin{align*}
			\tensor{g}{^a^b} \tensor{n}{_a} \tensor{n}{_b} &= \tensor{g}{^a^b} \tensord{U}{_a} \tensord{U}{_b}\\
			&= \tensor{g}{^U^U}\\
			&= 0,
		\end{align*}
		其中 $\tensor{g}{^U^U}$ 指 $\tensor{g}{^a^b}$ 在 $\{ U, V, \theta, \varphi \}$ 坐标系中的 $UU$ 分量。
	\end{zm}

	\item 试由式 (9-4-50)\footnote{
		正文 (9-4-50) 为
		\begin{equation}
			\dd \hat{s}^2 = - \left( 1 - 2M/r \right) \dd t^2 + \left( 1 - 2M/r \right)^{-1} \dd r^2 \tag{9-4-50}
		\end{equation}
	} 推出式 (9-4-51)\footnote{
		正文 (9-4-51) 为
		\begin{equation}
			\dd \hat{s}^2 = - \left( 1 - 2M/r \right) \dd v^2 + 2 \dd v \dd r. \tag{9-4-51}
		\end{equation}
	},再推出式 (9-4-54)\footnote{
		正文 (9-4-54) 为
		\begin{equation}
			\dd \hat{s}^2 = - \left( 1 - 2M/r \right) \dd{\tilde{t}}^2 + \left( 4M/r \right) \dd{\tilde{t}} \dd{r} + \left( 1 + 2M/r \right) \dd{r}^2. \tag{9-4-54}
		\end{equation}
	}。
	\begin{zm}
		\begin{enumerate}
			\item $t,r$ 与 $v,r$ 的变换为
				\begin{equation*}
					t = v-r_* \qc r = r,
				\end{equation*}
				故
				\begin{align*}
					\dd \hat{s}^2 &= - \left( 1 - \frac{2M}{r} \right) \dd t^2 + \left( 1 - \frac{2M}{r} \right)^{-1} \dd{r}^2\\
					&= - \left( 1 - \frac{2M}{r} \right) \left( \dd{v} - \dd{r_*} \right)^2 + \left( 1 - \frac{2M}{r} \right)^{-1} \dd{r}^2\\
					&= - \left( 1 - \frac{2M}{r} \right) \left( \dd{v} - \left( 1 - \frac{2M}{r} \right)^{-1} \dd{r} \right)^2 + \left( 1 - \frac{2M}{r} \right)^{-1} \dd{r}^2\\
					&= - \left( 1 - \frac{2M}{r} \right) \dd{v}^2 + 2 \dd{v} \dd{r}.
				\end{align*}
			\item $v, r$ 与 $\tilde{t}, r$ 的变换为
				\begin{equation*}
					v = \tilde{t} + r \qc r =r,
				\end{equation*}
				故
				\begin{align*}
					\dd \hat{s}^2 &= - \left( 1 - \frac{2M}{r} \right) \dd{v}^2 + 2 \dd{v} \dd{r}\\
					&= - \left( 1 - \frac{2M}{r} \right) \left( \dd{\tilde{t}} + \dd{r} \right)^2 + 2 \left( \dd{\tilde{t}} + \dd{r} \right) \dd{r}\\
					&= - \left( 1 - \frac{2M}{r} \right) \dd{\tilde{t}}^2 + \frac{4M}{r} \dd{\tilde{t}} \dd{r} + \left( 1 + \frac{2M}{r} \right) \dd{r}^2.
				\end{align*}
		\end{enumerate}
	\end{zm}

	\item 写出施瓦西度规在外向 Eddington 坐标系 $\{ u,r, \theta, \varphi \}(u \equiv t - r_*)$ 的线元表达式。

	\begin{jie}
		先考察前两维,$t,r$ 与 $u,r$ 的变换为
		\begin{equation*}
			t = v + r_* \qc r = r,
		\end{equation*}
		故
		\begin{align*}
			\dd \hat{s}^2 &= - \left( 1 - \frac{2M}{r} \right) \dd t^2 + \left( 1 - \frac{2M}{r} \right)^{-1} \dd{r}^2\\
			&= - \left( 1 - \frac{2M}{r} \right) \left( \dd{u} + \dd{r_*} \right)^2 + \left( 1 - \frac{2M}{r} \right)^{-1} \dd{r}^2\\
			&= - \left( 1 - \frac{2M}{r} \right) \left( \dd{u} + \left( 1 - \frac{2M}{r} \right)^{-1} \dd{r} \right)^2 + \left( 1 - \frac{2M}{r} \right)^{-1} \dd{r}^2\\
			&= - \left( 1 - \frac{2M}{r} \right) \dd{u}^2 - 2 \dd{u} \dd{r},
		\end{align*}
		补全得
		\begin{equation*}
			\dd s^2 = - \left( 1 - \frac{2M}{r} \right) \dd{u}^2 - 2 \dd{u} \dd{r} + r^2 \left( \dd{\theta}^2 + \sin^2 \theta \dd{\varphi}^2 \right).
		\end{equation*}
	\end{jie}

	\item 试证用 $\tensor{\left( \pdv*{V} \right)}{^a}$ 和 $\tensor{\left( \pdv*{U} \right)}{^a}$ 定义的 $\tensor{\xi}{^a}$ [见式 (9-4-40)\footnote{
		正文 (9-4-40) 为
		\begin{equation}
			\tensor{\xi}{^a} = \tensor{\left( \pdv*{t} \right)}{^a} = \frac{1}{4M} \left[ V \tensor{\left( \pdv*{V} \right)}{^a} - U \tensor{\left( \pdv*{U} \right)}{^a} \right] \tag{9-4-40}
		\end{equation}
	}] 在 $N_1$ 和 $N_2$ 上是类光 Killing 矢量场。
	\begin{zm}
		\begin{enumerate}
			\item 类光:
			\begin{align*}
				\tensor{g}{_a_b} \tensor{\xi}{^a} \tensor{\xi}{^b} &= \frac{1}{16M^2} \tensor{g}{_a_b} \left( V \tensorp{V}{^a} - U \tensorp{U}{^a} \right) \left( V \tensorp{V}{^b} - U \tensorp{U}{^b} \right)\\
				&= \frac{1}{16 M^2} \left( V^2 \tensor{g}{_V_V} -2UV \tensor{g}{_U_V} + U^2 \tensor{g}{_U_U} \right)\\
				&= - \frac{1}{8 M^2} UV \tensor{g}{_U_V},
			\end{align*}
			然而 $N_1$, $N_2$ 由 $U=0$ 或 $V=0$ 定义,故在 $N_1$, $N_2$ 上有 $UV=0$, $\tensor{\xi}{^a}$ 类光。
			\item Killing:
			\begin{align*}
				\Nabla{a} \tensor{\xi}{^b} &= \frac{1}{4M} \Nabla{a} \left( V \tensorp{V}{^b} - U \tensorp{U}{^b} \right)\\
				&= \frac{1}{4M} \left( \tensord{V}{_a} \tensorp{V}{^b} + V \ChristoffelSymbol{\mu}{\nu}{V} \tensord{x^\nu}{_a} \tensorp{x^\mu}{^b} \right.\\
				&\qquad\qquad \left. {}- \tensord{U}{_a} \tensorp{U}{^b} - U \ChristoffelSymbol{\mu}{\nu}{U} \tensord{x^\nu}{_a} \tensorp{x^\mu}{^b} \right),
			\end{align*}
			在 $\{U,V,\theta,\varphi\}$ 系下,易算得上式中涉及的非零克氏符为
			\begin{gather*}
				\ChristoffelSymbol{U}{U}{U} = - \frac{1+2M/r}{2M} \pdv{r}{U} \qc \ChristoffelSymbol{V}{V}{V} = - \frac{1+2M/r}{2M} \pdv{r}{V},\\
				\ChristoffelSymbol{\theta}{\theta}{U} = \ChristoffelSymbol{\varphi}{\varphi}{U} = \frac{1}{r} \pdv{r}{U} \qc \ChristoffelSymbol{\theta}{\theta}{V} = \ChristoffelSymbol{\varphi}{\varphi}{V} = \frac{1}{r} \pdv{r}{V},
			\end{gather*}
			由于
			\begin{equation*}
				r = f(X^2 - T^2) = f(UV),
			\end{equation*}
			其中 $f$ 的逆是
			\begin{equation*}
				f^{-1}(r) = \frac{r-2M}{2M} \e{r/2M},
			\end{equation*}
			可以算出
			\begin{align*}
				\pdv{r}{U} &= V \times \frac{1}{{\left(f^{-1}\right)}'(r)}\\
				&= \frac{4M^2}{r} V \e{-r/2M},\\
				\pdv{r}{V} &= U \times \frac{1}{{\left( f^{-1} \right)}'(r)}\\
				&= \frac{4M^2}{r} U \e{-r/2M},
			\end{align*}
			故以上 $V \ChristoffelSymbol{\mu}{\nu}{V}$ 项及 $U \ChristoffelSymbol{\mu}{\nu}{U}$ 项均含 $UV$,于是在 $N_1$ 和 $N_2$ 上为零。于是得
			\begin{align*}
				4M \Nabla{a} \tensor{\xi}{_b} &= \tensor{g}{_b_c} \left( \tensord{V}{_a} \tensorp{V}{^c} - \tensord{U}{_a} \tensorp{U}{^c} \right)\\
				&= \tensor{g}{_U_V} \left( \tensord{V}{_a} \tensord{U}{_b} - \tensord{U}{_a} \tensord{V}{_b} \right)
			\end{align*}
			为反称张量,故 $\Nabla{(a} \tensor{\xi}{_{b)}} = 0$。
		\end{enumerate}
	\end{zm}

	\item 把图 9-21 改画为图 9-23。 试通过计算图中的 $\Delta \tau'/\Delta \tau$ 给出式 (9-4-58) 的另一推导。提示:(1) $U \equiv -\e{(r_* - t)/4M}$ 在每条外向类光测地线上为常数。先后沿外部静态观者世界线和星面自由下落观者世界线求得同一 $\dd{U}$ 的两个表达式(分别含 $\dd{\tau'}$ 和 $\dd{\tau}$ ),在两式之间画等号便得式 (9-4-58)。(2) 在写出用 $\dd{\tau}$ 表出 $\dd{U}$ 的式子时要用到以能量 $E$ 表达 $\dv*{t}{\tau}$ 和 $\dv*{r}{\tau}$ 的公式,这可借 \S 9.1 的手法求得。
	\begin{figure}[htbp!]
		\centering
		\tikzexternalenable
		\begin{tikzpicture}[scale=0.4]
			\draw[very thick] (1,0) -- (0,0) -- (0,1);
			\draw[\myarrow, very thick] (0,0) -- (12,0);
			\draw[\myarrow, very thick] (0,0) -- (0,29.5);
			\node[right] at (12,0) {$r$};
			\node[above] at (0,29.5) {$\tilde{t}$};
			\fill[pattern=horizontal lines] (0,0) -- plot[domain=0:2.6] (\x^2,{10.08 - 2.82843*\x - 0.471405*\x^3 + 4*ln(sqrt(2)+\x)}) -- (6.76,0) -- cycle;
			\draw[very thick,postaction={decorate,decoration={raise=5,text along path,
			text={星\ 表\ 面\ \ },text align=right}}] plot[domain=0:2.6] (\x^2,{10.08 - 2.82843*\x - 0.471405*\x^3 + 4*ln(sqrt(2)+\x)});
			\draw[very thick] (2,8.9055) -- (2,29) node[pos=0.85,right] {\parbox{1em}{事件视界}};
			\draw[semithick,blue] plot[domain=2.4:10.6] (\x,{9.55598 + \x + 4*ln(\x-2)});
			\draw[semithick,blue,postaction={decorate,decoration={raise=4,text along path,
			text={|\color{blue}|外\ 向\ 光\ 子},text align=center}}] plot[domain=3.4:10.6] (\x,{1.88793 + \x + 4*ln(\x-2)});
			\draw[semithick,blue] plot[domain=4.4:10.6] (\x,{-3.08117 + \x + 4*ln(\x-2)});
			\draw[semithick,blue] (12, 13.58) -- (0,24.53);
			\draw[semithick] (10.6,28.763) -- +(2,0);
			\draw[semithick] (10.6,21.095) -- +(2,0);
			\draw[semithick] (10.6,16.1259) -- +(2,0);
			\draw[{Latex[length=5pt 6,width'=0pt 0.3]}-{Latex[length=5pt 6,width'=0pt 0.3]}] (11.6,28.763) -- (11.6,21.095) node[midway,sloped,fill=white] {$\Delta \tau_2^\prime$};
			\draw[{Latex[length=5pt 6,width'=0pt 0.3]}-{Latex[length=5pt 6,width'=0pt 0.3]}] (11.6,21.095) -- (11.6,16.1259) node[midway,sloped,fill=white] {$\Delta \tau_1^\prime$};
			\draw[very thick,purple] (10.6,0) -- (10.6,29) node[fill=white,pos=0.2,right] {\parbox{1em}{外部观者}};
			\node[rotate=-55] (t2) at (3.2,8.1) {$\Delta \tau_2$};
			\node[rotate=-57] (t1) at (4.3,6.2) {$\Delta \tau_1$};
			\draw (9.97,16) circle [x radius=0.55, y radius=0.1, rotate=5];
			\draw (7.95,17.85) circle [x radius=0.55, y radius=0.1, rotate=7];
			\draw (5.16,20.45) circle [x radius=0.55, y radius=0.1, rotate=11];
			\draw (1.4,23.93) circle [x radius=0.55, y radius=0.1, rotate=15];
			\node[fill=white] at (2,4) {\parbox{1em}{坍缩星}};
			\draw (2,0) -- (2,0.3);
			\node[below] at (2,0) {$2M$};
			\node[below left] at (0,0) {$0$};
		\end{tikzpicture}
		\tikzexternaldisable
		\caption{正文图 9-21}
	\end{figure}
	\begin{figure}[htbp!]
		\centering
		\tikzexternalenable
		\begin{tikzpicture}[scale=0.4]
			\draw[\myarrow, very thick] (0,0) -- (12,0);
			\draw[\myarrow, very thick] (0,-5) -- (0,12);
			\draw[very thick] (0,0) -- (10,10);
		\end{tikzpicture}
		\tikzexternaldisable
		\caption{正文图 9-23}
	\end{figure}
\end{xiti}
