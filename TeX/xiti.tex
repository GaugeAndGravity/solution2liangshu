\newcounter{xiti}[chapter]
\setcounter{xiti}{0}
\newcounter{subxiti}[xiti]
\setcounter{subxiti}{0}
\renewcommand{\thesubxiti}{\alph{subxiti}}
%\renewcommand{\thexiti}{\arabic{xiti}}
%\newcommand{\xit}{\stepcounter{xiti}\ensuremath{\mathbf{\thexiti}}\ }

\newenvironment{xiti}[1][0em]{%
\heiti\section*{习题}\begin{list}{\stepcounter{xiti}\textbf{\thexiti .}}{%
\itemindent=#1 \listparindent=\itemindent%
\leftmargin=0em \rightmargin=0pt%
\parsep=0pt}\normalfont}{\end{list}}

%\newenvironment{subxiti}{%
%\begin{list}{\stepcounter{subxiti}\textbf{(\thesubxiti)}}{\itemindent=0em \listparindent=2em%
%\parsep=0em \leftmargin=2em \rightmargin=0em}%
%\itemsep=0em \topsep=0em \parskip=0em%
%}{\end{list}}

\newenvironment{zm}{%
\begin{list}{\textbf{证明}}{%
\itemindent=0em \listparindent=2em%
\leftmargin=2.5em \rightmargin=0em%
\parsep=0pt \labelwidth=2em}\item \kaishu}{ \end{list}}

\newenvironment{jie}{%
\begin{list}{\textbf{解}}{%
\itemindent=0em \listparindent=2em%
\leftmargin=2em \rightmargin=0em%
\parsep=0pt \labelwidth=1em}\item \kaishu}{\end{list}}

\newenvironment{da}{%
\begin{list}{\textbf{答}}{%
\itemindent=0em \listparindent=2em%
\leftmargin=2em \rightmargin=0em%
\parsep=0pt \labelwidth=1em}\item \kaishu}{\end{list}}

\newenvironment{yl}[1]{%
\begin{list}{\textbf{#1}}{%
\itemindent=0em \listparindent=2em%
\leftmargin=2em \rightmargin=1em%
\parsep=0pt \labelwidth=3em}\item }{ \end{list}}
